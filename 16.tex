\documentclass[]{article}
\usepackage{lmodern}
\usepackage{amssymb,amsmath}
\usepackage{ifxetex,ifluatex}
\usepackage{fixltx2e} % provides \textsubscript
\ifnum 0\ifxetex 1\fi\ifluatex 1\fi=0 % if pdftex
  \usepackage[T1]{fontenc}
  \usepackage[utf8]{inputenc}
  \usepackage{eurosym}
\else % if luatex or xelatex
  \ifxetex
    \usepackage{mathspec}
  \else
    \usepackage{fontspec}
  \fi
  \defaultfontfeatures{Ligatures=TeX,Scale=MatchLowercase}
  \newcommand{\euro}{€}
\fi
% use upquote if available, for straight quotes in verbatim environments
\IfFileExists{upquote.sty}{\usepackage{upquote}}{}
% use microtype if available
\IfFileExists{microtype.sty}{%
\usepackage{microtype}
\UseMicrotypeSet[protrusion]{basicmath} % disable protrusion for tt fonts
}{}
\usepackage[unicode=true]{hyperref}
\hypersetup{
            pdfborder={0 0 0},
            breaklinks=true}
\urlstyle{same}  % don't use monospace font for urls
\IfFileExists{parskip.sty}{%
\usepackage{parskip}
}{% else
\setlength{\parindent}{0pt}
\setlength{\parskip}{6pt plus 2pt minus 1pt}
}
\setlength{\emergencystretch}{3em}  % prevent overfull lines
\providecommand{\tightlist}{%
  \setlength{\itemsep}{0pt}\setlength{\parskip}{0pt}}
\setcounter{secnumdepth}{0}
% Redefines (sub)paragraphs to behave more like sections
\ifx\paragraph\undefined\else
\let\oldparagraph\paragraph
\renewcommand{\paragraph}[1]{\oldparagraph{#1}\mbox{}}
\fi
\ifx\subparagraph\undefined\else
\let\oldsubparagraph\subparagraph
\renewcommand{\subparagraph}[1]{\oldsubparagraph{#1}\mbox{}}
\fi

% set default figure placement to htbp
\makeatletter
\def\fps@figure{htbp}
\makeatother


\date{}

\begin{document}

Igiene,Sanità pubblica e Politiche della salute

11/11/2016

relatore Prof. Simone Fanelli

\emph{DOMANDA E OFFERTA DEI SERVIZI SANITARI (sezione)}

\emph{Criteri di costruzione del fondo sanitario nazionale
(sottosezione)}

Vediamo ora come funziona il nostro sistema sanitario nazionale sotto il
punto di vista del finanziamento.

Inutile ripetere il tema che il servizio sanitario nazionale è un
sistema pubblico e questo comporta un sistema di finanziamento che
deriva da noi ,sostanzialmente dalle tasse e dai contributi che noi
versiamo allo Stato quindi, dato per scontato questo( anche se non è
scontato perché il nostro sistema nazionale prima della riforma del "78
non era pubblico), in realtà noi lo diamo per scontato oggi però ,tant'è
vero che in altri stati il sistema non è pubblico ma è privato; il
sistema di aziendalizzazione, la legge 592 nel "92 ,tutte le riforme
ancora adesso che si susseguono nascono per continuare a garantire un
sistema sanitario nazionale che sia pubblico.

Quindi in realtà noi dobbiamo capire effettivamente quanto e come
distribuire le risorse all'interno del sistema. Quindi c'è il problema
di definire un fondo che sia a disposizione delle aziende e del
personale e di tutti coloro che operano nel sistema.

Quindi bisogna definire esattamente di quante risorse ha necessità il
nostro sistema per funzionare in maniera efficiente ed efficace, per
perseguire gli obbiettivi per il quale è nato: quello di soddisfare il
bisogno di salute di tutti i cittadini.

Come viene costituito questo fondo ?

In realtà c'è una legge che è la legge finanziaria( ex legge finanziaria
del patto di stabilità) che ogni anno va a definire sostanzialmente
quale deve essere questo ammontare di risorse del settore.

E' proprio la legge della quale in questi giorni sentite parlare nei
telegiornali dove si stanno discutendo quali sono, qual è la
programmazione , quali sono le risorse da destinare ai diversi ambiti in
cui il pubblico interviene . Si discute quanto dare alla sanità ,alla
difesa, alla giustizia ,ai servizi sociali ,all'istruzione.

Quindi la legge finanziaria va a definire l'ammontare delle risorse .

Su che base lo fa? Su una base di una stima preventiva delle risorse che
devono essere impiegate e necessarie per garantire i L.E.A (livelli
essenziali di assistenza).

Il nostro settore non solo è pubblico cioè siamo noi a contribuire con
le nostre tasse e tributi al sistema .In realtà il sistema italiano è un
sistema fortemente accentrato cioè di finanza derivata , in realtà noi
paghiamo le tasse ma la maggior parte di queste tasse non le paghiamo al
singolo ente pubblico, noi non paghiamo le tasse direttamente
all'ospedale o alle ASL. Noi paghiamo le tasse ; poi lo Stato a livello
centrale raccoglie queste entrate, questi contributi dei cittadini per
poi distribuirle a chi opera nel settore ed eroga i servizi.

In realtà noi paghiamo le tasse per finanziare l'ospedale , le A.S.L
(azienda sanitaria locale) però non eroghiamo direttamente questo
contributo, le nostre tasse non le paghiamo direttamente a questi enti
ma le diamo allo Stato che poi le distribuisce.

Quindi questo è un sistema di finanza derivata: vuol dire che è
fortemente accentrata ed è lo Stato che raccoglie queste risorse per poi
distribuirle. In Stati in cui c'è una finanza meno accentrata sono
direttamente gli enti territoriali soprattutto o le regioni che prendono
nomi diversi ,o gli enti locali le province e i comuni che raccolgono in
maniera diretta le tasse.

In Italia solo una piccola parte deriva dai comuni (tasse comunali) ma
il grosso dei contributi deriva da trasferimenti finanziari da parte
dello Stato.

Questo è lo scenario nel quale noi ci muoviamo ,il sistema di finanza si
dice derivata.

Quindi noi abbiamo la fiscalità generale ;quindi le tasse che noi
paghiamo contribuiscono a creare questo fondo, nel senso ,contribuiscono
ad alimentare questo fondo sanitario nazionale. Quali sono le tasse che
noi paghiamo allo stato e che vanno a finire nel fondo?

Sono l'addizionale IRPEF(imposta sul reddito delle persone fisiche) che
è una tassa a livello regionale cioè ogni regione va a stabilire
l'addizionale rispetto all'IRPEF statale,l'IRAP (imposta regionale
attività produttive) che è un'altra tassa che pagano le imprese sulle
loro attività pubbliche, poi abbiamo la compartecipazione all'IVA: anche
una parte dell'IVA dei consumi che noi facciamo va ad alimentare il
fondo ,e poi la accisa sulla benzina, queste sono le macro voci che
impattano e vanno a creare questo fondo.

Su che base quindi viene definito il fondo complessivo ,considerato che
queste sono le entrate? Vengono utilizzati diversi criteri.

In sanità per definire quanto deve essere l'ammontare di questo fondo
vengono utilizzati il criterio di una percentuale del PIL. Che cos'è il
PIL? Il PIL è il prodotto interno lordo ma sostanzialmente rappresenta
dal punto di vista economico la ricchezza di un paese .

Anche se qui ci sono tante critiche che non considerano il PIL un buon
indicatore della ricchezza tanto vero anche qui spesso sentite parlare
del PIL la necessità che questo cresca perché rappresenta la nostra
ricchezza , più cresce , più vuol dire che il nostro paese è ricco
.Sostanzialmente allora,tornando a prima, per capire il primo elemento
che entra in gioco è importante capire quanto della nostra ricchezza
deve essere destinata alla sanità .

E qui vengono fatte delle scelte a livello politico , voi sapete più o
meno quanto del PIL ogni anno viene destinato al sistema sanitario
nazionale ? Siamo attorno al 7 \%.

Ogni anno varia di poco ma negli ultimi anni è stabile al 7\% in realtà
in linea con quanto spendono i paesi industrializzati soprattutto
occidentali fatta eccezione degli USA che spendono il 14\% noi siamo più
o meno in linea con gli altri paesi poi se è tanto o poco non siamo qui
a fare valutazioni. Ogni anno si parla di aumentare il fondo ,in realtà
in termini assoluti aumenta dal punto di vista il valore assoluto,
l'anno scorso era circa 110 miliardi ma come percentuale al PIL siamo
stabili .

Naturalmente questo PIL, questo fondo deve tenere in considerazione le
esigenze per erogare le prestazioni ,quindi un'altro criterio è quello
di capire il fabbisogno di spesa dei beneficiari .Quanto hanno bisogno
le regioni, le ASL ,le aziende ospedaliere per erogare queste
prestazioni? Si fa una stima secondo la quota forfettaria ponderata
,vuol dire che si prendono in considerazione le caratteristiche degli
utenti che questi beneficiari devono soddisfare . Ponderata vuol dire
che vado a capire esattamente ,innanzi tutto il numero dell'utenza di
riferimento,poi anche quelle che sono le caratteristiche .Una
popolazione anziana avrà più bisogno di maggiori risorse ,una ASL che
deve soddisfare questo bacino di utenza ,avrà bisogno di più risorse
,quindi questo è un altro criterio che entra.

\emph{Introduzione all'economia sanitaria (sottosezione)}

Chi è che in sanità ha il compito, dal punto di vista dell'ente
territoriale più vicino, ad erogare le prestazioni sanitarie? Quale
organo di governo?

In Italia abbiamo tre organi di governo : lo Stato ,le Regioni,gli enti
locali.

Negli enti locali abbiamo le province e i comuni, l'unione di comuni.

Quale in sanità ,ad oggi, ha il massimo potere di definire le politiche
di sanità? Le regioni, almeno fino alla nuova riforma costituzionale che
va ha ridisegnare le competenze tra stato e regioni.

Le Regioni in qualche modo sono le responsabili di raggiungere e
soddisfare la domanda dei servizi sanitari della propria popolazione di
riferimento.

Il problema è che le regioni, per le tasse che alcune sono raccolte a
livello accentrato fanno riferimento a quanto viene prodotto all'interno
della regione: abbiamo detto che l'addizionale IRPEF è una tassa
"regionale" è una addizionale rispetto all'IRPEF statale anche l'IRAP è
una tassa che pagano le imprese che fanno riferimento al territorio,
quindi le pagano alla regione e poi la regione le trasferisce ,sono
contributi ,entrate dirette dalla regione.

Il problema è che le regioni italiane non sono tutte uguali non hanno
tutte la stessa attività produttiva al proprio interno , ci sono regioni
più ricche e regioni meno ricche e quindi regioni che riescono in
maniera più autonoma a soddisfare i bisogni sanitari dei propri
cittadini e regioni che invece trovano maggiore difficoltà.

Quindi qual è il sistema che permette anche alle regioni "meno ricche"
di contribuire in maniera efficace al soddisfacimento di bisogno di
salute da parte dei cittadini?

E' quello della creazione di un fondo nazionale che viene alimentato
sostanzialmente dalla compartecipazione all'IVA.

Quindi dove la regione non riesce ad arrivare con le risorse proprie, lo
Stato trasferisce ulteriori fondi ad integrazione di quelle proprie per
raggiungere l'ammontare delle risorse necessarie per soddisfare i
bisogni dei beneficiari e quindi degli utenti.

Quindi ,abbiamo la fiscalità generale con tasse regionali ,tasse statali
e il fondo,che viene determinato sulla base della percentuale del PIL
che però tiene in considerazione la spesa dei beneficiari proprio perché
l'IRPEF e l'IRAP sono fortemente legati alla capacità produttiva del
territorio e quindi della regione nel caso della sanità, essendo
l'organo di livello di governo in cui è competenza il tema della sanità,
esiste un fondo che viene alimentato dalla partecipazione all' IVA che
aiuta queste queste regioni che sono in difficoltà.

Che cosa fanno le regioni con questo fondo ? Devono soddisfare i LEA(
livello essenziale di assistenza) sono quelle prestazioni che su tutto
il territorio nazionale devono essere garantite a tutti i cittadini.

In realtà le regioni , quindi i LEA , sono definiti a livello statale
nazionale anche se in concerto con le regioni (è la conferenza stato-
regioni che definisce quali sono i LEA e le caratteristiche che devono
avere) .In realtà proprio perché siamo in un paese in cui le politiche
sanitarie sono poi sviluppate dalle regioni , le regioni hanno la
possibilità non di di ridurre i LEA ma di potenziarli , quindi di
aumentare le prestazioni che rientrano in quelle garantite a tutti.

Il momento in cui una regione fa questa scelta di aumentare i LEA , deve
soddisfare in maniera autonoma , quindi con risorse proprie, la propria
capacità di garantire tutte le prestazioni in più . Quindi tutte le
ulteriori esigenze finanziarie delle regioni devono essere soddisfatte
con risorse proprie! Eventuali disavanzi sono coperti dalle regioni
quindi anche qui la regione non agisce direttamente eroga la prestazione
sanitaria ma lo fa attraverso le ASL , le aziende ospedaliere
principalmente e nel momento in cui una ASL o azienda ospedaliera non
riesce con le proprie risorse ,è in deficit ed entra in perdita, deve
intervenire la Regione per appianare questo stato di disavanzo.

Questo è uno schema che fa vedere in realtà come tutte le regioni sono
autonome sotto il profilo finanziario quindi in rosso le entrate proprie
, quelle che derivano dai ticket o da attività di ricerca o donazioni, o
entrate proprie delle regioni.

Poi abbiamo l'IRAP, quella tassa regionale che è espressione della
produttività del territorio ,poi abbiamo anche qui l'addizionale IRPEF ,
però vedete che per arrivare al 100\% interviene in maniera forte il
fondo perequativo in modo che tutte le regioni abbiano lo stesso livello
di risorse necessario per soddisfare le proprie esigenze sanitarie .

Evidente come la provincia autonoma di Bolzano è autonoma al 50\% ,la
Calabria invece dipende dal fondo perequativo per il 93/\%.

E' logico che la Calabria ha una quantità di aziende di attività
produttive minori rispetto alla Lombardia ,Emilia Romagna e altre
regioni più produttive e quindi lo stato interviene grazie al fondo
perequativo nazionale per garantire anche ai cittadini della Calabria
,Puglia, Sicilia che sono in difficoltà le risorse necessarie per
erogare i servizi.

La regione in realtà può agire , non sull' IRAP che è fissata a livello
nazionale , su quella non ha manovre, l'IRAP è quella è competenza della
regione non la può alzare o abbassare ;l'unica tassa su cui può agire è
sull' addizionale IRPEF nel senso di quella quota in più rispetto
all'IRPEF pagata a livello nazionale per la quota della regione.

La regione può farlo, può cambiarlo, a meno che la regione sia in stato
in piano di rientro .

Sapete cosa vuol dire piano di rientro?

Le regioni che non riescono in maniera autonoma a soddisfare i propri
bisogni, cioè che abbiano una quantità di entrate inferiori alle uscite
o costi e ricavi a basso livello economico , entrano in piano di
rientro: è una situazione in cui la regione deve nominare un commissario
per la gestione delle politiche sanitarie perché non è in grado in
maniera autonoma di essere in equilibrio.

Per raggiungere gli equilibri visti l'altra volta non a livello di
azienda ma a livello di regione quindi se faccio la somma delle entrate
di tutte le aziende sanitarie e le uscite di tutte le uscite delle
aziende sanitarie, quella regione è in disavanzo quindi entra in piano
di rientro e in più in quella situazione deve per legge mettere tutte
queste asticelle al massimo , in realtà lo può fare ma solo se è una
situazione di equilibrio quindi non ha bisogno di attingere in maniera
forte dal fondo perequativo nazionale. Se io abbasso una aliquota, però
questo non mi permette di essere in equilibrio, per legge entro nel
piano di rientro quindi devo aumentare per forza le aliquote al livello
massimo.

Succede anche ai comuni in dissesto che devono aumentare le tasse al
massimo pur avendo un margine di manovra per rientrare dal disavanzo.

Sono in piano di rientro quasi tutto il sud , la Sicilia ,la Puglia , la
Calabria ad esclusione

della Basilicata , il Lazio , Campania , al nord abbiamo la Liguria e il
Piemonte , una delle due la Liguria mi sembra sia rientrata nel piano
triennale.

Poi in realtà cosa è successo con l'ultima legge finanziaria? con la
legge di stabilità del 2015 - 2016 questa situazione di squilibrio che
prima veniva registrata a livello regionale, quindi quando una regione
rientrava nel piano di rientro, nel momento in cui aveva un disavanzo a
livello complessivo, le 20 aziende che aveva all'interno se le entrate
erano inferiori alle uscite a livello complessivo ,la regione entrava
nel piano di rientro.

Con la nuova norma questo vale a livello di ogni singola azienda, in
realtà dal 2016 le singole aziende ,la singola azienda ospedaliera ,per
adesso sono partite solamente le aziende ospedaliere, quelle
universitarie ,le IRS le ASL invece l'anno prossimo ,entrano le singole
aziende sanitarie un piano di rientro, quindi l'equilibrio che prima
veniva visto solo a livello complessivo di regione adesso si vede nel
dettaglio in ogni singola azienda regioni come la Lombardia non sono in
piano di rientro hanno all'interno aziende che in realtà sono in deficit
,sono in squilibrio , esempio il Niguarda di Milano che è un grande
ospedale che è in piano di rientro perché non raggiunge l'equilibrio con
la legge finanziaria , questa normativa che sta entrando insomma,per la
quale i piani di rientro devono essere in qualche modo definiti entro la
fine dell'anno e le azioni di miglioramento devono durare tre anni e
devono portare nel triennio al recupero della situazione di squilibrio
,perché prima interveniva la regione, prima per il Niguarda anche se era
in disavanzo e non raggiungeva l'equilibrio ,interveniva la regione a
ripianare il deficit e riportare l'equilibrio.

Adesso non più , interviene lo stato e impone questi limiti a livello di
ogni singola azienda .

A chi è interessato a questo tema con il prof. Signorelli abbiamo
scritto un lavoro su questo tema, sulla regionalizzazione e sui piani di
rientro e nuove modifiche al sistema.

Una volta che ho definito il fondo , la regione abbiamo capito che ha a
disposizioni un tot di risorse per soddisfare il bisogno di salute dei
cittadini ,abbiamo anche detto che la regione non lo fa in modo diretto
nel senso che sono poi le aziende sanitarie ASL e aziende ospedaliere
principalmente ad erogare le prestazioni.

Come fa quindi la regione a distribuire queste risorse?

Anche qui entrano in gioco diversi criteri , io ho, visto i criteri per
determinare il fondo era il PIL ,la capacità di entrate, i fabbisogni.

Questi sono i criteri che esistono in sanità e in generale nel settore
pubblico per trasferire le risorse finanziarie.

Quindi i sistemi che possono utilizzare le regioni per distribuire i
fondi:

Il primo è il criterio del piè di lista.

Cosa prevede questo criterio? dal punto di vista sostanziale che si
intervenga a posteriori , un criterio di finanziamento ex post alla
gestione ,che prevede di rimborsare l'azienda di tutti i costi
sostenuti.

Ripeto, il criterio del piè di lista è un criterio di finanziamento ex
post alla gestione cioè a posteriori che interviene andando a trasferire
una quantità di risorse pari ai costi sostenuti dall'azienda nell'anno
precedente ex post.

Secondo voi questo criterio è un criterio che mira a raggiungere quel
principio di economicità che avete visto con Ferretti? No! Perché è un
criterio che incentiva la spesa,perché io più spendo più mi riconoscono
risorse e quindi sono incentivato ad aumentare la spesa, anzi a maggior
ragione sono efficiente e utilizzo un numero minore di risorse
paradossalmente vedo il mio ammontare di risorse destinate minore.

Io sono una azienda di Parma che faccio le stesse attività che fa Modena
ma le fa con meno risorse, mi trasferiscono 100 anziché 150 che danno a
Modena perché ha speso di più. Questo criterio è fortemente
deresponsabilizzante a livello aziendale che però è un criterio che
veniva utilizzato prima della 502/92 quindi non è cosi strano e cosi
impensabile di trovare questo criterio ,tanto vero che la 502 che pone
l'obbiettivo dell'equilibrio della economicità e quindi
dell'efficienza,cancella questo criterio come criterio guida
nell'assegnare le risorse proprio perché il processo di
aziendalizzazione mirava le economicità ,migliore gestione efficienza ed
efficacia dei servizi, questo è un criterio che non va in quella
direzione.

Resta oggi però un criterio utilizzato nel momento in cui dicevo prima,
un' azienda è in deficit e interviene la regione per ripianare il
deficit di quella azienda. Anche se adesso con i piani di rientro
bisogna capire esattamente che cosa succede fino ad oggi, fino a quando
i piani di rientro a livello aziendale non sono attuativi , questo
criterio viene utilizzato per ripianare ex post la gestione o situazioni
di squilibrio delle singole aziende.

Il criterio della spesa storica è un criterio che invece si basa sullo
storico quindi è un criterio di finanziamento ex ante: io ti do prima un
ammontare di risorse sulla base dello storico cioè quanto hai speso nel
passato.

Io ti do esattamente la stessa cifra ma sono delle valutazioni che
vengono fatte sullo storico.

Si prende in considerazione o l'ultimo anno o il trend degli ultimi
anni, degli ultimi cinque anni , ti vado a dare un ammontare del 5\%
superiore o in linea o con un taglio del 5\%, un criterio che può pagare
quello in comune che si va a vedere la spesa storica ,cioè quanto hai
speso in passato per erogare un servizio o una prestazione.

Quest'altro criterio è un criterio che agevola il raggiungimento
dell'economicità o anche questo è un criterio di deresponsabilizzante
come il piè di lista ?

E' un criterio deresponsabilizzante anche qui ,se storicamente ho speso
tanto e ho speso male comunque mi vengono riconosciute delle risorse in
linea con il passato .

Questo non sarebbe un criterio che in qualche modo agevola il
raggiungimento dell'economicità.

Eppure anche questo è un criterio che in sanità viene utilizzato nel
momento in cui devono essere finanziate le funzioni che sono quelle che
dicevo prima ,ci sono alcune attività che l'azienda mette in campo ,
come può essere il pronto soccorso ( come facevamo esempio l'altra volta
per il solo fatto di avere un pronto soccorso e un insieme di risorse e
attività in qualsiasi momento da destinare al soddisfazione del bisogno,
lo Stato lo riconosce e lo fa sulla base dello storico).

La spesa storica può essere , è ,un criterio per il
finanziamento,soprattutto delle funzioni, le attività che vengono svolte
non producono attività di ricovero quindi direttamente valutabili in
termini di agio del prodotto.

Il criterio dei parametri espressivi del fabbisogno sono simili a quelli
visti prima, tengono conto, in considerazione dei servizi che deve
erogare l'azienda sanitaria ,la quale deve soddisfare determinati
bisogni.

Ogni azienda sanitaria ha un bacino di riferimento con determinati
bisogni di salute e quindi questo criterio va a stabilire ex ante più o
meno quale sarà il fabbisogno di risorse per soddisfare questi cittadini
questa utenza di riferimento.

Uguale a quello che abbiamo visto prima per la creazione del fondo, si
prendono in considerazione dei parametri ,il numero di abitanti ,il
sesso, l'età ,le condizioni morfologiche del territorio, una serie di
elementi che mi permettono di stabilire qual è il loro fabbisogno di
prestazioni sanitarie.

E qui di conseguenza assegnano un fondo appropriato: è questo qui
che,insieme alla spesa storica ,rientra per funzionamento del
finanziamento. Il pronto soccorso si basa sullo storico ,naturalmente
deve tenere in considerazione qual è il bacino di utenza dell'azienda
sanitaria per capire esattamente di quanto hanno bisogno.

Ultimo è il criterio del prezzo di trasferimento .

Questo criterio prevede che ex post vengano riconosciuti i costi sulla
base di quanto ha prodotto , nel senso che l'azienda produce una
prestazione , questa prestazione ha una determinata tariffa, che prende
il nome di prezzo di trasferimento, ed ex post io ti riconosco quel
prezzo di trasferimento ,perché tu hai erogato quella prestazione .

Nello specifico parliamo dei DRG (Diagnosis Related Groups): sono le
tariffe DRG, sono indicazione di quanto l' azienda avrebbe dovuto
spendere oppure i costi che avrebbe aver dovuto sostenere per erogare
quella prestazione sanitaria quindi ex post ti riconosco esattamente
quel costo quella tariffa .

La tariffa DRG è un prezzo di trasferimento che dovrebbe andare a
coprire esattamente il costo che l' azienda dovrebbe sostenere in
condizioni di efficienza per erogare quella prestazione.

Domanda! Questo criterio è un criterio che agevola il raggiungimento
dell'economicità o anche questo è un criterio deresponsabilizzante
secondo voi?

Vado a riconoscere le prestazioni effettivamente effettuate.

Le tariffe DRG cambiano in base alla attività , se io faccio una
operazione che in condizioni di efficienza dovrebbe costare 10000
\euro{} , ti do 10000 \euro{} ,se è una operazione semplice che richiede
solo l'assorbimento di risorse per 1000 \euro{} io ti riconosco 1000
\euro{},tiene in considerazione quanti elementi ,e naturalmente la
complessità della prestazione effettuata ,della complessità della
prestazione ,dei costi per erogare quella prestazione; quindi dentro c'è
tutto dentro il DRG :c'è il personale ,i costi generali, i beni di
consumo ,protesi,dispositivi medici. Il problema è questo qui: se io
finanzio le attività che possono essere incentivate ad aumentare la
produzione , più aumento e più mi riconoscono dei ricavi, in realtà se
io faccio questa prestazione mi riconoscono 10000\euro{} a prescindere
perché in condizioni di efficienza dovrei spendere per sostenere i costi
per 10000\euro{}.

Se io poi ne consumo di risorse 15000\euro{} e me ne danno 10000\euro{}
,quindi in qualche modo non sono stato efficiente ,allora in qualche
modo genero una perdita.

Se invece a fronte di 10000\euro{} che mi vengono riconosciuti dalla
regione, io sostengo costi per 8000\euro{} ho un "utile" di 2000\euro{}
perché sono stato efficiente.

Questo incentiva chi ha speso 15000\euro{} ad avvicinarsi allo standard
e alla tariffa stabilita. Questo però può spingere le aziende ad
aumentare la produzione facendo anche prestazioni non richieste ,perché
poi ho un ritorno.

Ci sono due strategie per limitare questo fenomeno, uno è quello di
mettere dei tetti , io ti riconosco un tot di prestazioni se tu ne fai
di più non te le riconosco e non ti rimborso, deve essere appropriato
per fare una prestazione piuttosto che un'altra.

Un altro modo è quello di agire sulle tariffe,problema proprio pratico :
esempio quella del parto , parto cesareo e parto naturale , sono due
attività che generano DRG naturalmente uno è un DRG chirurgico ,quindi
il parto cesareo ha un riconoscimento e una tariffa superiore rispetto
ad un parto naturale, quindi questo portava le aziende ad aumentare i
parti cesarei per avere un riconoscimento maggiore anche a rischio di
appropriatezza: anche qui, o pongo un limite( il parto cesareo deve
essere un 15\% dei parti quindi se tu ne fai di più io non ti li
riconosco) oppure quello di abbassare la tariffa: provi ad abbassare la
tariffa, tu il parto cesareo lo fai con parto naturale,hai la stessa
tariffa ,quindi sei incentivato a fare un parto cesareo solo dove è
necessariamente richiesto.

Quindi sicuramente il problema di aumentare la produzione tende a far sì
che l'azienda aumenti la produzione, a lato ci sono questi meccanismi
che generano,in qualche modo, di porre dei limiti alla produzione.

Le ASL sono sostanzialmente ,principalmente finanziate a quota capitaria
sulla base del fabbisogno , mentre le aziende ospedaliere sulla base dei
prezzi di trasferimento, per le attività sono legate al DRG e
all'attività di recupero o comunque la spesa stessa ,spesa storica per
tutte le funzioni che lì azienda comunque produce, per le quali non
esiste un prezzo di trasferimento (esempio del pronto soccorso che ho
fatto prima).

Naturalmente anche le ASL possono avere una parte dei prezzi di
trasferimento,nel momento in cui sono produttrici, diciamo che a livello
generale il metodo generale principale del finanziamento delle ASL è la
quota capitaria, le aziende ospedaliere sono i prezzi di trasferimento.

Ogni ASL ha un ammontare di risorse, che tiene in considerazione quelli
che potrebbero essere i bisogni dei cittadini; io do alle ASL 100 perché
ha una composizione di popolazione di un certo tipo; si va ha vedere il
sesso,l'età, genere ,la composizione del territorio : montagna
,pianura,..una serie di parametri che vengono tenuti in considerazione
per definire la quota da destinare al finanziamento di quella ASL.

Poi le regioni: per chiudere il cerchio , abbiamo detto che in Italia
più che sistema sanitario nazionale sarebbe più appropriato parlare di
21 sistemi regionali perché le politiche che possono adottare le singole
regioni nel campo sanitario sono importanti e portano a delle differenze
notevoli tra le diverse regioni . Questo vale anche sotto il punto di
vista non solo dell'organizzazione,ma anche del finanziamento.

Abbiamo visto che anche le prestazioni possono cambiare i LEA da regione
a regione.

Esistono quindi tre macro sistemi che possono essere accumunati ,che
vedono come elemento centrale un sistema ,un soggetto piuttosto che un
altro ,esempio in Emilia Romagna dove ci troviamo noi, le ASL (diciamo
un modello di centralità dell'ASL perché la ASL fa una doppia attività,
sia attività di acquisire le prestazioni dalle aziende ospedaliere
pubbliche e dal privato accreditato).

In realtà in Emilia Romagna le ASL producono anche direttamente le
prestazioni attraverso i propri presidi ospedalieri.

Quindi la regione finanza quelle ASL sulla base della quota capitaria
.Le ASL finanziano il privato accreditato e l'azienda ospedaliera, sulla
base delle tariffe di DRG sostanzialmente, però la ASL stessa è un
produttore, quindi non è solo un acquirente dei servizi sanitari .

In Lombardia invece abbiamo la netta la separazione tra il ruolo di chi
eroga il servizio, quindi il privato accreditato dalla azienda
ospedaliera, e la ASL.

Quindi la ASL non eroga direttamente servizi attraverso i presidi, ma si
limita alla attività di programmazione e finanziamento,cioè acquista
direttamente i servizi a delle aziende pubbliche e private attraverso il
sistema PRG ,non interviene direttamente nella produzione(quindi c'è una
netta separazione dei due soggetti).

Ultimo ,vede la regione come modello centrale del sistema cioè nel senso
non è più l'ASL che va a finanziare il privato accreditato e la azienda
ospedaliera pubblica, ma lo fa direttamente la regione ,e quindi le ASL
sono dei produttori e non sono più degli acquirenti.

Quindi il sistema di finanziamento poi cambia anche a seconda del
sistema regionale di riferimento proprio perché non esiste un sistema
unico ma siamo qui in presenza di 21 sistemi sanitari regionali .

Chiudiamo quindi questo capitolo sul sistema di finanziamento.

\emph{Il bisogno di salute -variabili che influiscono sulla domanda
(sottosezione)}

Vediamo adesso alla domanda e all'offerta dei servizi sanitari.

L'economia sanitaria si occupa proprio di questi temi cioè di studiare
il bisogno di domanda di salute e studiare la produzione e l'offerta di
prestazioni sanitarie, e naturalmente si occupa anche della parte di
valutare i costi che vengono sostenuti dal sistema ,e di fare
programmazione e controllo all'interno del sistema , criterio di
finanziamento,rapporto con il pubblico e privato ..ora ci concentriamo
sull'ultimo, in parte sui costi abbiamo già discusso ampliamente la
volta scorsa e sulla parte di programmazione soprattutto sul
finanziamento abbiamo appena visto. Ci concentriamo sulla domanda e
sull'offerta di prestazioni sanitarie.

Abbiamo detto che il sistema in Italia è un sistema pubblico ,questo
cosa vuol dire ,che sostanzialmente la maggior parte delle scelte nel
campo sanitario vengono svolte dal soggetto pubblico ,e ripeto questo
non è una cosa scontata nel senso che ci sono paesi in cui le scelte in
campo sanitario sono delegate al privato.

Almeno in Italia si è scelto di istituire questo sistema sanitario
pubblico per due motivi :

Il primo è quello dell'efficienza locativa , cosa vuol dire?

Vuol dire che dal momento in cui il cittadino singolo , l'individuo,
delega ad un organo superiore che può essere l'ente pubblico ,quindi
espressione in qualche modo della comunità di riferimento, lo fa perché
il pubblico è in grado di allocare meglio le risorse.

Quindi se la scelta fosse lasciata al singolo,la locazione delle risorse
non sarebbe ottimale: se io delego al pubblico, avviene una
distribuzione ottimale delle risorse ,poi ci ritorniamo nel dettaglio
,però questo è il primo principio.

Secondo è motivi di equità .

In Italia non si ritiene giusto che alcuni cittadini siano esclusi dal
beneficiare di determinati servizi e cure sanitarie. Quindi si ritiene
non sia equo e quindi devono essere garantiti a tutti, questi sono i due
principi che spingono in qualche modo il sistema ad essere pubblico:
questo fa sì che noi deleghiamo a qualcuno, "soggetto pubblico'', la
scelta di fare tutte le scelte in campo sanitario come la maggior parte
delle scelte che poi fa il singolo: la spesa farmaceutica ,se vuole fare
una assicurazione sanitaria,...sono scelte che sono delegate al singolo.

Per capire esattamente il concetto di efficienza locativa ,dobbiamo
spiegare il concetto di esternalità ..sapete cos' è un'esternalità? è
quell'impatto positivo o negativo che la scelta del singolo ha sugli
altri. Nel senso, è una scelta che fa l'individuo ,una scelta privata
che però può avere un impatto o positivo o negativo sugli altri .Nel
momento in cui c'è l'azione di esternalità, sia questa positiva o
negativa,avviene che questa scelta spetta al singolo ,ma spetta al
pubblico! Per questo c'è l'intervento pubblico nel settore sanitario.
Andiamo a fare degli esempi\ldots{}esempio di una esternalità positiva:
c'è una scelta che il singolo fa e ha un impatto positivo anche sugli
altri.

Facciamo un esempio ,visto che negli ultimi mesi si è parlato poco di
questo tema ,il tema dei vaccini,ma se io mi vaccino è un vantaggio per
me? Si! Ma questa mia scelta individuale di vaccinarmi ha un impatto
positivo anche sugli alti? Si! Nel momento in cui io non mi ammalo evito
quindi di contagiare altre persone.

Sicuramente il tema delle vaccinazioni è una scelta che non dovrebbe
essere lasciata al singolo ,perché magari l'intervento pubblico potrebbe
prevedere quel tipo di vaccinazione per tutti e quindi massimizzare il
beneficio che ne deriva.

Perché se io delegassi al singolo individuo questa scelta potrei avere
un numero di vaccinazioni inferiori rispetto al punto ottimale.

Quindi viene in qualche modo internalizzata l'esternalità. E un esempio
di scelta individuale che può avere un impatto negativo su una scelta
che faccio io ,e portare uno svantaggio agli altri sotto il profilo
sanitario? Per esempio quella del fumo .Se io fumo ho un impatto
negativo su di me sicuramente, ma un impatto negativo anche per chi mi
sta intorno.

Quindi anche in questo caso si delega al pubblico la scelta di vietare
il fumo nei luoghi pubblici ,voi qui vedete il cartello vietato fumare:
siamo in un luogo pubblico ,quindi il pubblico ritiene giusto vietare
per internalizzare questa esternalità . Perché è una scelta individuale
che però ha un impatto negativo anche sugli altri .Quindi nel momento in
cui esistono queste esternalità ,siano positive o negative si ritiene
giusto,delegare al pubblico la capacità di prendere queste decisioni.

Secondo motivo è quello dell'equità: non si ritiene giusto escludere
qualcuno dal beneficiare di un servizio come quello sanitario,dalle
prestazioni sanitarie solo perché non se lo può permettere
sostanzialmente. Questo è molto più legato all'etica e alla cultura di
un paese naturalmente,ci sono paesi in cui non c'è questo principio di
equità , tant'è vero che vengono esclusi tutti coloro che non hanno la
possibilità di usufruire del servizio.

In Italia alcune prestazioni di assistenza sanitaria vengono considerati
beni meritori, beni che devono essere garantiti in maniera gratuita a
tutti.

Quali sono questi beni meritori? sono i LEA (livelli essenziali di
assistenza) che sono quelle prestazioni che sono garantite sul
territorio nazionale in maniera gratuita a tutti i cittadini .Ecco
perché possono permetterselo, e non perché non hanno a disposizione
risorse finanziarie ed economiche per comperare quel bene,semplicemente
perché hanno bisogno di quel bene di quel servizio. Quindi la logica che
è al di sotto del sistema è il concetto di bisogno di guidare le scelte
in sanità ,e non quello della domanda dei servizi privati. Nel settore
privato c'è domanda e offerta ,io ho bisogno della mia auto faccio la
domanda di un'auto ma il fornitore ,la Fiat ,mi da l'auto solo se il
prezzo della sua automobile è coerente con la mia disponibilità a
spendere per quel bene.

In sanità no, anche se non ho una disponibilità economica si ritiene
giusto che quella prestazione sia erogata al soggetto.

Nel momento in cui esistono questi LEA ,questi beni meritori che sono
resi in maniera gratuita, nascono due criticità principali ,quali
possono essere queste due criticità?

Siamo in presenza di beni meritori : tutti i costi legati al
finanziamento,come erogare prestazioni,ha un costo e qualcuno deve
pagare queste prestazioni, e quindi il problema del finanziamento,chi
finanzia questa LEA ?di questo abbiamo già parlato del sistema di
finanziamento, accentramento ,distribuzione delle risorse.

Secondo problema da affrontare, è capire esattamente quali devono essere
questi LEA, questi beni meritori, anche qui bisogna definire chi deve
pagare le prestazioni e come definire se una prestazione rientra o meno
nei LEA.

Qui sono delle prestazioni in evoluzione ,che devono tener conto dei
bisogni dei cittadini e della comunità , quindi si modificano e si
basano su una base di parametri oggettivi che fanno sì che una
prestazione rientri o meno nei LEA.

Se entriamo nel dettaglio sono principi di efficacia, di efficienza ,di
equità quindi fanno sì che una prestazione rientri nei LEA definiti
dalla conferenza stato-regioni.

Abbiamo detto che in sanità più che dalla domanda ,bisogna partire dal
bisogno perché è il bisogno a guidare le scelte a far funzionare tutto
il sistema .

Si parla di bisogno di salute quando c'è uno stato di insoddisfazione
percepito dal singolo individuo.

Questo bisogno ,affinchè si trasformi in una domanda ,nell'erogazione
del servizio occorre innanzitutto conoscere un mezzo per soddisfare
questo bisogno,questo stato di insoddisfazione e poi desiderare di avere
a disposizione quel mezzo in grado di soddisfare il bisogno.

Questi sono i tre elementi che fanno si che si passi dal bisogno al
richiedere e poi al formulare la domanda .

Ci sono delle variabili individuali che impattano sul bisogno :
età,sesso, stato civile,valori con certe esperienze..l'elemento
determinante è quello dell'età: esistono forti legami tra età e il
consumo di prestazioni sanitarie ?E' un andamento un po' particolare,che
senso assume questa forma di y? Più alto nei primi mesi, nei primi anni
di vita dell'individuo,poi decresce per poi aumentare in maniera
esponenziale dopo i 50-60 anni .

Questo è il legame che unisce l'età ,che vedete sull'asse delle x ,con
il consumo medio annuo di prestazioni sanitarie.

Il bisogno abbiamo detto che per arrivare alla prestazione vera e
propria deve trasformarsi in domanda altrimenti resta un bisogno
inespresso. Bisogna attivare tutto il sistema sanitario ,occorre che
questo bisogno si trasformi in domanda.

Sulla domanda abbiamo innanzitutto fattori variabili predisponenti
:quello visto prima, l'età ,sulle variabili diciamo oggettive che
influenzano la predisposizione a richiedere un servizio sanitario, anche
il fattore abilitante ,condizioni monetarie o anche non monetarie perché
il soggetto poi decida di rivolgersi al sistema.

Poi ci sono delle condizioni del sistema offerta che possono influenzare
il comportamento degli individui ,quindi sul comportamento e sulla
domanda influenzano sia le condizioni di bisogno , sia quelle dei
fattori legati fortemente alla domanda, ma anche il sistema dell'
offerta è più articolato quindi dipende un po' dalle risorse,
dall'organizzazione, dal comportamento dei medici ,se la mia domanda, il
mio bisogno si trasformi in domanda.

Io non sto bene ,l'ospedale di Parma non è dei migliori ,non ho tempo e
modo di spostarmi e allora decido sostanzialmente di non ricorrere al
sistema offerto, può influenzare il comportamento dell' individuo e il
ricorso o meno alle prestazioni.

\emph{Dallo stato di salute al consumo della prestazione (sottosezione)}

Vediamo adesso qual è il processo che porta dallo stato di salute al
consumo della prestazione.

Si parte con lo stato di salute :non c'è neanche bisogno,è la situazione
iniziale .

La fase successiva è quella che Mabelli, questo studioso, chiama come
bisogno sistematico,bisogno oggettivo che però non è ancora noto al
soggetto,all'individuo,cioè si è venuto a modificare lo stato di
salute,però il soggetto non è ancora consapevole di questa modificazione
del suo stato di salute .Esistono processi di diagnosi precoce, di
screening, proprio con l'obbiettivo di andare a catturare ed individuare
situazioni di bisogno sintomatico dove il soggetto non è ancora a
conoscenza del bisogno di salute. C'è quindi il processo di screening
,la diagnosi precoce per arrivare ad un bisogno che il soggetto avverte
,è un bisogno soggettivo ,che viene valutato dal singolo :quindi adesso
c'è lo stato di bisogno.

Cosa può fare adesso il nostro individuo,una volta che capisce che non
sta bene e si è modificato il suo stato di salute? Ha diverse opzioni
davanti a se', o far finta di nulla e con un po' di riposo gli passa ,o
ricorrere "all'auto cura": si auto prescrive l'aspirina perché ha mal di
testa, quindi non impatta sul sistema in maniera diretta ,oppure va dal
medico.

Posso avere un bisogno non espresso ,torno al mio stato di salute oppure
decido di andare da un operatore specializzato del sistema ,e quindi
vado dal medico.

Anche qui poi il medico ,una volta che l'individuo arriva da lui, può
una volta visitato il soggetto, fare diverse scelte. Quali sono le
scelte che può fare il medico?

Può prescrivere lui direttamente la cura , può mandarlo per accertamenti
a ulteriori visite specialistiche,può rassicurare il paziente dicendo
che non ha niente : ``non hai niente, vai a casa e riposati'', oppure
può attivare attraverso l'ospedale la procedura di ricovero.

Quindi una volta che vado dal medico il bisogno diventa una domanda,
questa domanda può essere soddisfatta in maniere diverse prima di
arrivare al consumo della prestazione, quindi servizio di base
specialistico e servizio ospedaliero .

Che poi può far tornare il soggetto al suo stato di salute .Questo è il
percorso che passa dallo stato di salute al bisogno alla domanda fino al
consumo della prestazione.

Quando parliamo poi di domanda dei servizi sanitari (facciamo lo step
successivo al bisogno!) abbiamo ,almeno nel nostro paese
industrializzato ,una serie di problemi che portano ad un aumento della
domanda dei servizi sanitari .

\emph{L'aumento della domanda : governare la domanda per migliorare la
risposta (sottosezione)}

Vediamo un crescere di prestazioni richieste dai singoli al
sistema.Quali sono questi fattori che portano ad un aumento della
domanda di prestazioni sanitarie ? La domanda di prestazioni è in
continuo aumento,perché? E' in aumento l'invecchiamento della
popolazione, che naturalmente comporta poi che sia legato fortemente
all'età il consumo di prestazioni sanitarie. Naturalmente
l'invecchiamento della popolazione porta ad un maggior ricorso alle
prestazioni con una maggior domanda.

Poiché le condizioni sono cambiate ,sempre più persone che vivono in
situazioni di malattie croniche quindi ,comportano non che nel momento
acuto faccio domanda e finisce qui, ma ad una domanda continua di
prestazioni e di cure ,e non solo l'invecchiamento,ma anche le cure che
vengono richieste sempre più condizioni di cronicità. Le nuove
tecnologie portano ad individuare in maniera precoce determinate
malattie, quindi il progresso tecnologico porta ad aumento della domanda
, aumento di richieste di questo tipo e di conseguenza il progresso
tecnologico porta ad un miglioramento della condizione fisica
,miglioramento delle cure ma anche vita più lunga grazie al progresso
tecnologico: è un po' collegato all'andamento del grafico anche un
elemento che parte dal progresso tecnologico , gli screening e cosi via,
sono tutte tecnologie che in qualche modo migliorano sicuramente lo
stato di salute ,però fanno sì che ci siano maggiore richiesta di
servizi.

Altro elemento è che non si parla più di sanità ma di salute in
generale, lo stesso Ministero che si chiamava della sanità ora si chiama
Ministero della salute .Il concetto di salute è molto più ampio ,è il
benessere generale quindi diventa più difficile capire esattamente quali
sono quei bisogni per stare bene,e cosa vuol dire effettivamente essere
in salute e avere benessere. Questo ampliamento del concetto di salute
porta ad un aumento delle prestazioni. Altro elemento importante che non
avete citato, è quello del terzo pagante.In un sistema pubblico dove non
sono io a pagare per le prestazioni che mi vengono fornite, io tendo ad
aumentare la domanda ,giusto? Il terzo pagante(con il fatto di non
dovere pagare io l'azienda ospedaliera ma la regione ),mi porta ad una
distorsione della domanda,a chiedere più prestazioni sanitarie anche se
non dovute.Anche l'azienda ospedaliera ha poi ha il vantaggio di erogare
queste prestazioni (come diceva il vostro collega), io sono stimolato
attraverso il sistema delle DRG e delle tariffe ad aumentare la
produzione perché mi arrivano più risorse anche per quanto riguarda il
lato dell'offerta, le aziende producono più prestazioni perché questo le
porta ad un ritorno economico ,quindi c'è una distorsione legata al
sistema del terzo pagante. Ultimo elemento è quello delle case
farmaceutiche che investono più in marketing ,il doppio di quanto
investono in ricerca e sviluppo ,quindi anche loro hanno un impatto
:vedere pubblicità in continuazione su un farmaco magari induce le
persone ad acquistare il farmaco quindi a ricorrere alle prestazioni per
le cure .Questo problema di aumento della domanda dei servizi sanitari
porta poi un problema di governo di queste domande, cioè cercare una
serie di soluzioni ed innovazioni in grado di gestire questo aumento di
domanda . Come abbiamo appena detto ci sono relazioni tra domanda e
offerta , quindi nell'azienda ospedaliera sappiamo che il numero di
ricoveri è lineare alla disponibilità di posti letto, più le aziende
sanitarie hanno posti letto più tendono ad occuparli, quindi ad
aumentare la produzione e quindi la domanda, tant'è vero che si parla
spesso di riduzione del numero dei posti letto, riorganizzare il sistema
di offerta perché l'offerta guida la domanda più io offro servizi più
persone tendono a fare domanda. Poi c'è il tema della farmaceutica, il
consumo dei farmaci è legato al prezzo, se quel farmaco è reso in
maniera gratuita dal sistema ne faccio un determinato uso, se invece
devo pagare anche il ticket cambia,non lo utilizzo ,quindi un altro
elemento che impatta sulla domanda è il consumo dei farmaci in questo
senso.

Questo è quello che diceva la vostra collega sul miglioramento delle
tecniche, che mi porta a incrementare il ricorso a scopo preventivo per
diagnosticare in maniera precoce eventuali malattie e patologie. Quindi
il problema rilevante diventa adesso quello di gestire questo aumento di
domanda. Il governo della domanda nei servizi sanitari ha questo
obiettivo : cercare di guidare le persone verso un uso responsabile
delle risorse. Questo vale non solo per il consumatore ,ma anche per il
servizio di offerta proprio degli operatori che operano nel sistema,
l'obbiettivo è quello di continuare a rendere il nostro sanitario
pubblico sostenibile, efficiente ed efficace, che non è una cosa
scontata e banale, anche perché sul sistema pubblico nulla vieta di
tornare al passato, il nostro sistema non è più sostenibile, dobbiamo
lavorare tutti affinché questo sistema sanitario resti nazionale, quindi
garantito a tutti .

\emph{Governare la domanda per migliorare la risposta (sottosezione)}

Come possiamo governare la domanda?

Come possiamo fare in modo che ci sia una sostenibilità nel sistema?

Possiamo lavorare innanzitutto sulle persone,sui consumatori ,quindi
prima che il bisogno si trasformi in domanda, attraverso l'educazione
sanitaria,gli stili di vita e cosi via ..oppure sul prendersi cura di se
stessi.

Secondo ruolo importante è giocato dal medico, soprattutto dal medico di
medicina generale perché è lui che ha il ruolo di intermediario e di
filtro tra la domanda e l'offerta, è lui che ti dice vai a casa oppure
vai in ospedale oppure prendi questi farmaci , sicuramente una funzione
chiave ce l'ha il medico di medicina generale ,questo è il sistema anche
di capire e influenzare il comportamento dei medici.

Domanda: ma il nostro sistema è comunque sostenibile se noi eliminiamo
il ruolo del medico di medicina generale? La risposta : dipende un po'
dal sistema che si viene a creare, sicuramente ci sono delle condizioni
che non vanno bene,perché spesso ci si lamenta del medico di medicina
generale ,perché il suo ruolo non è altro che quello che ti indirizza
poi verso lo specialista .Se il suo ruolo è solo quello di indirizzarti
e non riesce a dare una risposta concreta ai tuoi bisogni di salute, in
quel caso non può avere una visione critica.

Una riduzione del numero dei medici di medicina generale può essere una
soluzione per cercare di ridurre il costo legato a questo servizio,a
maggior ragione un servizio considerato dagli utenti come inefficiente,
quindi potrebbe essere una soluzione , sull'eliminarlo completamente
sono un po' critico ,bisognerebbe fare delle valutazioni , so che ci
sono delle riduzioni ,una serie di manovre e altre forme che cercano di
migliorare questo sistema attraverso ,ad esempio, mettendo insieme un
team di medici di medicina generale, non c'è il singolo ma ci sono dei
gruppi, in Inghilterra è molto utilizzato questo sistema, in qualche
modo si integrano delle competenze, quindi c'è una gestione migliore sul
cliente, questa potrebbe essere una soluzione ,di non avere il singolo
ma avere un gruppo di riferimento a cui il singolo può andare, da avere
come punto di riferimento piuttosto che il singolo soggetto.

\emph{L'offerta dei servizi sanitari - variabili e offerta
(sottosezione)}

Ultimo punto è quello di agire sull'offerta e quindi riorganizzare il
servizio.\\
Ci sono degli strumenti quindi ,diretti sul governo della domanda e
strumenti indiretti; quelli diretti vanno sostanzialmente sul
consumatore , e sempre attraverso una educazione sanitaria della
popolazione le liste di attesa vengono create per scoraggiare chi non ha
bisogno di una prestazione, nel fare domanda di quella prestazione.

Cosi come la compartecipazione alla spesa e quindi il ticket che noi
paghiamo è un modo per disincentivare il ricorso alle prestazioni
sanitarie.

Il ticket nasce con questa funzione,non come strumento aggiuntivo per
finanziare l'azienda sanitaria , che poi ultimamente sia diventato una
fonte di entrate rilevante per l'azienda è un discorso, però
storicamente nasce per governare la domanda e cercare di ridurre il
ricorso a questi servizi che abbiamo visto prima ,che la spesa
farmaceutica è legata al fatto che il farmaco viene in qualche modo
pagato direttamente dal soggetto oppure viene rimborsato dal sistema.

Questo discorso ,che una attestazione non è necessaria e io devo pagare
quella prestazione attraverso il ticket ,io posso essere scoraggiato
perché non è necessaria ,se devo pagare io non c'è il problema legato al
terzo pagante.

Strumenti indiretti vanno sul lato dell'offerta, quindi a potenziare la
medicina territoriale, fornire solo quelle prestazioni che si basano
sull'evidenza scientifica e poi definire dei criteri per gestire le
liste di attesa o quello che viene fatto al pronto soccorso, per esempio
viene assegnato un codice colore in base alla gravità, lo si fa proprio
per scoraggiare gli accessi inappropriati .

Esempio : i codici bianchi che arrivano in ospedale che invece
dovrebbero essere gestiti dal medico di medicina generale. Se io do un
ordine di priorità in base alla reale necessità di quella prestazione ,
anche qui posso governare la domanda . Alcuni ospedali applicano i
codici bianchi e i ticket ,anche qui una doppia funzione della gestione
delle priorità.

Innanzitutto ti tratto per ultimo dopo aver trattato tutti i casi più
complessi che hanno bisogno di un intervento urgente da parte della
struttura . Dall'altro , devi pagare il ticket perché è un servizio per
il quale non è necessario l'intervento dell'azienda sanitaria.

\emph{Input output outcome - definizione del prodotto (sottosezione)}

Ultimo aspetto è legato quindi all'offerta, siamo passati dal bisogno
alla domanda di prestazioni, vediamo adesso il dato dell'offerta.

Sul dato dell'offerta ,dipende sostanzialmente dalle risorse messe in
campo dall'organizzazione offerta dal comportamento dei medici ,che
abbiamo visto hanno un ruolo fondamentale nel sistema ,non solo i medici
di medicina generale, e poi il rendimento di efficienza ed efficacia
abbiamo visto essere anche questa una variabile importante nel
comportamento sia degli operatori che dei consumatori.

All'interno del sistema di offerta chi assorbe maggior numero di risorse
anche dal punto di vista economico sono gli ospedali.

Facciamo fuoco sugli ospedali perché assorbono della spesa totale in
campo sanitario tra il 50 e 60\%. Quando parliamo del sistema offerta ci
limitiamo al ruolo degli ospedali. Due considerazioni importanti sugli
ospedali: il primo è che sono delle aziende a forte utilizzo del
personale( labor intensive si contrappone al capital intensive ),sono
aziende che si basano soprattutto sulle persone ,lo abbiamo visto quando
abbiamo visto il bilancio, nel conto economico la voce più importante di
costo era legato al personale.

Sta crescendo molto il costo legato alle attrezzature, macchinari sempre
più costosi però la voce principale resta quella del personale .

Secondo aspetto è che ci sono dei costi di struttura molto rigidi, sono
i cosiddetti costi fissi. I costi fissi sono quei costi che non variano
al variare della produzione. Io pago lo stipendio al medico, lo pago a
prescindere dal numero di pazienti che tratta dal numero di casi che
tratta nell'arco dell'anno o del mese. Se io ho deciso di acquistare una
TAC che costi 10000\euro{}, io quel costo lo sostengo sia se quella TAC
viene utilizzata una volta al mese sia se viene utilizzata cento volte
al mese. I costi fissi sono questi , sono quei costi che non variano al
variare della produzione, mentre i costi variabili sono i costi tipo i
farmaci, i vari consumi sanitari ,quei costi che hanno un rapporto
lineare tra la prestazione e il costo io li sostengo solo se tratto il
caso. Questo cosa comporta avere una struttura di questo tipo con tanto
personale ,che poi il personale è un costo fisso a prescindere dai casi
trattati ,l' infermiere lo pago tutti i mesi con lo stesso stipendio,
l'operatore lo stesso discorso, il medico anche.

Il fatto di avere questa struttura cosi rigida ,cosa comporta secondo
voi?

Questi sono costi che io sostengo a prescindere dalla mia attività , o
tratto un caso o ne tratto 10000 all'anno, io sostengo questi costi ,per
essere in equilibrio il nostro ospedale deve aumentare la produzione ,se
io ho una TAC da 10000 \euro{} e la utilizzo per un caso significa che
per trattare quel caso io ho speso 10000\euro{}. Se io la uso per 10000
casi vuol dire che ogni caso mi costa 1\euro{} sotto il profilo del
consumo del macchinario. Quindi l'ospedale per questa struttura rigida
deve cercare di aumentare la produzione per cercare di distribuire
quest' ammontare di costi fissi su un numero di casi maggiori, per avere
una riduzione del costo unitario nel singolo caso trattato. Questo è il
processo che il nostro ospedale utilizza per creare salute ,alla fine
l'obiettivo dell'ospedale è quello di creare salute a un insieme di
punti che sono le risorse a disposizione del nostro ospedale ,qui
abbiamo macchinari ,personale ,attrezzature ,le strutture,tutto quello
che viene impiegato nel processo produttivo, quindi nel punto uno
abbiamo il processo di produzione per produrre l'out-put, qual è
l'out-put del nostro ospedale?E' la prestazione sanitaria .

Punto uno ,devo impiegare questi \textbf{out-put} in maniera efficiente.

Però l'obbiettivo dell'ospedale è quello di produrre la prestazione
sanitaria fine a se stessa?

No! Ma è quello di avere un impatto positivo su bisogno di salute ,nel
senso che la prestazione sanitaria ha il fine di soddisfare il bisogno
di salute di chi si è rivolto a questo ospedale, quindi quello che
importa è l'\textbf{outcome.}

Il punto due è l'efficacia, nel senso della mia azione per generare un
risultato positivo e quindi una soddisfazione del bisogno.

Qual è poi il problema che in realtà sull'obiettivo ultimo del nostro
ospedale impattano non solo la nostra prestazione sanitaria? Ci sono una
serie di elementi esterni che influenzano l'\textbf{outcome:} il
patrimonio genetico del soggetto ,come i fattori ambientali, culturali,
socio economici ,stili di vita\ldots{}

In realtà valutare l'outcome effettivamente, il risultato prodotto
dall'ospedale, è molto difficile .Anche misurare la produttività del
nostro ospedale è molto difficile ,perché quali sono le problematiche?
Che innanzitutto bisogna capire lo stato di salute prima e dopo.

Pensate solo alle malattie croniche, non c'è effettivamente un
miglioramento dello stato di salute quindi capire esattamente qual è lo
stato di salute prima e dopo il trattamento è veramente molto difficile.
Prima abbiamo detto che è su questo risultato su questo out come,che
impattano fattori esterni all'attività dell'ospedale,per la vita
dell'ospedale per capire qual è il contributo del nostro ospedale sul
risultato ottenuto è difficile isolare la singola componente sulle n
componenti che entrano in gioco , di conseguenza se io non so
esattamente qual è il risultato che ho ottenuto diventa anche difficile
ipotizzare un sistema di finanziamento adeguato per l'ospedale ,tanto
vero che faremo con i DRG ma i DRG vanno a finanziare la prestazione, tu
fai la prestazione e io ti do la tariffa come rimborso ma non vado a
valutare poi effettivamente qual è l' impatto output ottenuto.

Io non posso con esattezza distinguere il contributo della prestazione
sanitaria da una serie di fattori esterni.Quindi diventa difficile
capire effettivamente il contributo della prestazione. Quindi
,concludendo, come faccio a capire qual è la produzione del nostro
ospedale e quindi poi progettare tutto il sistema di finanziamento?Per
capire se il nostro ospedale va bene devo utilizzare delle proxit ,delle
approssimazioni che vanno a toccare o gli imput o gli ouptut ,quindi
vado a vedere effettivamente nell'ospedale quali sono i fattori di
produzione ,gli strumenti utilizzati, personale ,spese
farmaceutiche,altri costi variabili ,i consumi, però chiaro questa non è
la prestazione,vado a capire sulla base delle risorse impiegate quanto
hai prodotto.

Oppure vado sui prodotti ,tante operazioni effettuate quante visite
ambulatoriali, anche qui non ho l'outcome, ho l'ouput, il prodotto
fornito alla fine del servizio, oppure ancora vado sul numero di giorni
dedicati alla cura o il numero dei casi trattati ; questi sono tutti
elementi che possono essere utilizzati sì per valutare la produttività
di un ospedale, ma sono delle proxit che mi aiutano solo in parte a
capire effettivamente quanto ha prodotto questo ospedale .

\end{document}
