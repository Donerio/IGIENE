\documentclass[]{article}
\usepackage{lmodern}
\usepackage{amssymb,amsmath}
\usepackage{ifxetex,ifluatex}
\usepackage{fixltx2e} % provides \textsubscript
\ifnum 0\ifxetex 1\fi\ifluatex 1\fi=0 % if pdftex
  \usepackage[T1]{fontenc}
  \usepackage[utf8]{inputenc}
\else % if luatex or xelatex
  \ifxetex
    \usepackage{mathspec}
  \else
    \usepackage{fontspec}
  \fi
  \defaultfontfeatures{Ligatures=TeX,Scale=MatchLowercase}
\fi
% use upquote if available, for straight quotes in verbatim environments
\IfFileExists{upquote.sty}{\usepackage{upquote}}{}
% use microtype if available
\IfFileExists{microtype.sty}{%
\usepackage{microtype}
\UseMicrotypeSet[protrusion]{basicmath} % disable protrusion for tt fonts
}{}
\usepackage[unicode=true]{hyperref}
\hypersetup{
            pdfborder={0 0 0},
            breaklinks=true}
\urlstyle{same}  % don't use monospace font for urls
\IfFileExists{parskip.sty}{%
\usepackage{parskip}
}{% else
\setlength{\parindent}{0pt}
\setlength{\parskip}{6pt plus 2pt minus 1pt}
}
\setlength{\emergencystretch}{3em}  % prevent overfull lines
\providecommand{\tightlist}{%
  \setlength{\itemsep}{0pt}\setlength{\parskip}{0pt}}
\setcounter{secnumdepth}{0}
% Redefines (sub)paragraphs to behave more like sections
\ifx\paragraph\undefined\else
\let\oldparagraph\paragraph
\renewcommand{\paragraph}[1]{\oldparagraph{#1}\mbox{}}
\fi
\ifx\subparagraph\undefined\else
\let\oldsubparagraph\subparagraph
\renewcommand{\subparagraph}[1]{\oldsubparagraph{#1}\mbox{}}
\fi

% set default figure placement to htbp
\makeatletter
\def\fps@figure{htbp}
\makeatother


\date{}

\begin{document}

Lezione 07-11-2016

Sbobinatore: Maddalena Quarto

Docente: Marco Ferretti (economia)

\textbf{AZIENDA PUBBLICA} (definizione): ordine economico d'istituto in
cui il soggetto economico è pubblico.

\textbf{SEZIONE: ISTITUTO}

Insieme di persone che si unisce per soddisfare un bisogno.

Possiamo definire fondamentalmente 4 grossi gruppi di persone:

\begin{itemize}
\item
  \textbf{La famiglia:} persone che si uniscono per soddisfare più
  bisogni contemporaneamente. Maslow ha creato una piramide divisa in 5
  livelli che individua i bisogni umani. Il più basso è rappresentato
  dai \emph{bisogni primari} (alimentazione, vestiario) poi, la
  \emph{socialità}, la \emph{sicurezza} fino ad arrivare al livello
  apicale: la \emph{leadership} a cui solo alcuni riescono ad arrivare.
  Nella famiglia si arriva a soddisfare i primi 3. La famiglia è un
  tipico \textbf{istituto di consumo} (produce un reddito che in realtà
  può non essere prodotto da tutti i membri della famiglia ma tutti lo
  consumano);
\item
  \textbf{Istituto impresa}: qui alcune persone si uniscono e mettono
  insieme del capitale di rischio puntando ad ottenere un maggiore
  profitto;
\item
  \textbf{Istituto pubblico territoriale:} persone che si uniscono per
  soddisfare una collettività di riferimento (provincia, regione,
  azienda sanitaria). Non è autonomo economicamente ma trae le proprie
  risorse economiche dallo Stato;
\item
  \textbf{Istituto no profit:} è come il precedente ma viene a mancare
  la corretta remunerazione dei prestatori di lavoro per via della
  volontarietà dell'opera dei lavoratori.
\end{itemize}

\textbf{Gli istituti nascono per durare nel tempo ed essere autonomi nel
tempo.} L'autonomia finanziaria permette agli istituti di durare nel
tempo. Essere patologicamente dipendente da terze economie va a segnare
in negativo la durata nel tempo. Es: la famiglia indigente che richiede
al comune un sostentamento di qualunque tipo come una dimora o un
assegno, diventa, se non si rende autonoma economicamente,
patologicamente dipendente dal comune e quindi la loro sopravvivenza
risulta a rischio se il comune non riesce più a fornire loro tale
servizio.

\textbf{SEZIONE: SOGGETTO ECONOMICO}

Persona o gruppo di persone nel cui prevalente interesse è gestita
l'azienda. Nel caso dell'azienda privata il soggetto economico sono i
prestatori di lavoro e la direzione. Nel caso di azienda pubblica:

\begin{itemize}
\item
  la collettività (domanda reale-domanda potenziale)
\item
  lo Stato, la Regione
\item
  prestatori di lavoro
\item
  direzione (direttore sanitario e amministrativo)
\item
  fornitori
\end{itemize}

\textbf{SEZIONE: AZIENDA = Ordine economico d'istituto}

Scambi economici tra i membri dell'istituto e l'ambiente esterno. Es:
l'azienda famiglia, tipicamente di consumo, attraverso un lavoro ha
delle entrate che vengono ``investite'' all'esterno per ricevere beni e
servizi (cinema, ristorante.)

Nell'istituto pubblico di servizio, il soggetto economico scambia beni
economici sia all'interno sia all'esterno diventando un'azienda
pubblica.

Esistono di 3 tipi di azienda pubblica:

\begin{itemize}
\item
  \textbf{di consumo} (azienda sanitaria: fornendo i vaccini consuma un
  determinato tipo di risorse con lo scopo di influenzare e diminuire la
  domanda futura, quindi si cerca di scongiurare l'insorgenza di una
  malattia con la vaccinazione; questa strategia è detta esternalità
  positiva)
\item
  \textbf{di produzione} (Enel e il trasporto pubblico locale: erogano
  un servizio che viene ripagato)
\item
  \textbf{composta} (azienda ospedaliera: avviene un consumo di risorse
  attraverso l'erogazione di prestazioni a titolo gratuito ma anche con
  con pagamento di ticket)
\end{itemize}

Nelle SDO (scheda di dimissione ospedaliera) sono contenuti i
\textbf{DRG = gruppi omogenei di diagnosi che vanno a riclassificare
tutte le possibili diagnosi in aggregati che variano da regione a
regione.} Ad ogni DRG è associata una tariffa sulla base
dell'assorbimento di risorse impegnate (\textbf{sistema isorisorse).}

Il paziente contribuisce indirettamente al sistema attraverso il
pagamento di tasse che possono essere delle più disparate (ad esempio
l'accisa sulla benzina o l'IVA) e non direttamente pagando un ticket
alla dimissione. Quindi esistono degli scambi economici che possono
essere facilmente o difficilmente individuabili. Questi scambi vengono
registrati su quello che viene chiamato \textbf{libro giornale.} Ad ogni
data vengono segnate le scritture di fornitori che hanno portato merce,
quindi che devo pagare, e dei compratori che hanno acquistato, quindi
che devono pagare me. Giornalmente vengono messi in ordine gli
accadimenti all'interno dell'istituto.

\textbf{SEZIONE: ECONOMICITA'}

Definizione: equilibrio dinamico nel tempo tra i costi e i ricavi
dell'azienda che ne garantisce una condizione di vita duratura.

VARIABILI per definirla:

\begin{itemize}
\item
  \textbf{EQUILIBRIO FINANZIARIO}: capacità di un'azienda di far fronte
  in qualsiasi momento a una qualsiasi tipologia di uscita. Come fa ad
  assicurarsi le risorse per far fronte alle uscite? Produce e vende
  beni e servizi in modo tale da creare un margine di guadagno rispetto
  ai costi che ha sostenuto per produrli. \emph{Quindi le entrate sono
  corrette}. Ad esempio un'azienda paga il personale con delle entrate
  corrette per pagare il personale, quando un'azienda fa un investimento
  (costruisce un nuovo padiglione, compra una Tac) andandosi ad
  indebitare per cui avremo entrate corrette per quella tipologia di
  uscita\textbf{\emph{. Sono in equilibrio finanziario quando utilizzo
  delle entrate corrette per avere una determinata tipologia di
  uscita.}}
\item
  \textbf{EQUILIBRIO MONETARIO}: è il salvadanaio. Metto dentro soldi e
  li tolgo. Mi fa capire se nel maialino ci sono + e quindi posso far
  fronte ad una spesa, se ci sono dei -- ovviamente questo non è
  possibile. \textbf{\emph{Sono in equilibrio monetario quando posso far
  fronte ad una determinata uscita in qualsiasi momento, perché ho avuto
  delle entrate corrette per farlo}}.
\item
  \textbf{EQUILIBRIO ECONOMICO}: \textbf{\emph{sono in equilibrio
  economico quando i costi sono uguali ai ricavi}} (quello che ho
  prodotto). Non c'è utile e non c'è perdita. Per un'azienda sanitaria
  non è ``bello'' chiudere con utili perché vuol dire che hai fatto
  pagare delle risorse non consumate ai cittadini, i quali hanno
  contribuito al sistema con l'imposizione fiscale, quindi l'azienda non
  ha fatto propriamente il loro interesse. Un piccolo utile bisogna
  farlo essendoci delle prestazioni a totale consumo; da qualche parte
  quindi devi avere delle prestazioni con ricavo maggiore del costo,
  perché se perdi su tutto anche il tuo bilancio perderà.
\end{itemize}

L'azienda sanitaria fattura alla Regione con fatture trimestrali.
L'ultima fattura dell'anno è Ottobre-Novembre-Dicembre 2016 che viene
rendicontata e fatturata a Gennaio 2017. Nel 2016 ho avuto costi\% e
ricavi\% (quello che ho prodotto e non quello che entra nel
salvadanaio). Il salvadanaio è nello stato patrimoniale non nel conto
economico. Le entrate monetarie vanno nel salvadanaio. Se si fa una
fattura a gennaio 2017 è impossibile che siano entrate risorse nel
salvadanaio del 2016 quindi, ho consumato e ricavato 100 nel 2016 ma gli
effetti finanziari, cioè le entrate e le uscite di cassa, li avrò nel
2017. I costi li ho sostenuti durante l'anno (ho pagato il personale
tutti i mesi, i farmaci, i fornitori) i ricavi arrivano dopo perché
prima devo fare la fattura e poi la Regione mi paga a 60 giorni. Quindi
i soldi relativi all'ultimo trimestre del 2016 arrivano a marzo 2017 ma
nel frattempo sostengo delle spese per diversi mesi (ho 6 mesi di buco)
senza avere un'entrata in cassa effettiva.

ALTRO ESEMPIO: compro una macchina con costi per 10000 euro e chiedo un
finanziamento. Il mio esborso monetario iniziale è 2000 con flussi in
uscita mensili per i prossimi 36 mesi. Quindi compro un bene, consumo
risorse pari a 10000 con esborso monetario oggi di 2000. Lo sfasamento
entrate-uscite è l'EQUILIBRIO MONETARIO. Però, quando si parla di
economicità, bisogna guardare anche l'EQUILIBRIO FINANZIARIO, in altre
parole il fatto che vengano correttamente utilizzate delle fonti di
finanziamento per determinate spese.

\begin{itemize}
\item
  \textbf{EFFICIENZA}: è il rapporto tra risorse consumate e risultati
  ottenuti. Esempio: ho degli \emph{\textbf{input}} (3 infermiere, 3
  batuffoli di cotone, 3 siringhe, 3 provette, carta e toner stampante
  ecc) che sono risorse, fattori produttivi (in questo caso sono beni
  sanitari, beni non sanitari, personale) e produco un
  \emph{\textbf{output}} (prelievo del sangue). Se si riduce ad 1
  infermiere, 1 provetta, 1 batuffolo ecc per produrre lo stesso output,
  allora si parla di efficienza. A parità di input però, posso essere
  più efficiente anche aumentando l'output, ovvero aumentando il numero
  di prelievi effettuati con le stesse risorse\textbf{\emph{. Si parla
  di efficienza quando, dato l'output riduco l'input oppure quando a
  parità di input aumento l'output}}.
\item
  \textbf{EFFICACIA}: è il rapporto tra il risultato ottenuto e
  l'obiettivo previsto. Nell'esempio di prima l'obiettivo è la diagnosi
  precoce di una patologia. Tanto più raggiungo l'obiettivo finale,
  tanto più velocemente ho fatto una diagnosi precoce, tanto più come
  azienda sanitaria mi scongiuro dei costi per curare una patologia che
  se presa tardivamente avrà dei costi sicuramente maggiori.
\end{itemize}

\begin{quote}
Se si stressa molto l'efficienza, si va ad intaccare l'efficacia. Se ad
esempio metto l'infermiere senza la siringa, la provetta ecc.. questo
non sarà in grado di fare il prelievo, non si avrà la diagnosi precoce e
quindi non si raggiunge l'obiettivo.
\end{quote}

\begin{itemize}
\item
  \textbf{PATRIMONIO}: è quello su cui si può andare ad impattare per
  cercare di raggiungere i primi 3 equilibri.
\end{itemize}

\emph{Domanda studente: Differenza tra patrimonio ed equilibrio
monetario:}

L'equilibrio monetario è il salvadanaio.

Il bilancio è fatto da 3 documenti:

\begin{itemize}
\item
  \textbf{Conto economico} (costi e ricavi di 1 anno solare)
\item
  \textbf{Conto patrimoniale} (tutto il patrimonio dell'azienda. È come
  il suo cv. Ogni anno dei valori devono essere messi nelle attività (+)
  e passività (-) e fanno la storia dell'azienda. Nelle + c'è il
  salvadanaio)
\item
  \textbf{Nota integrativa} è un file word che spiega i precedenti file
  excel
\end{itemize}

Quando si parla di entrate ed uscite monetarie, si parla di entrate ed
uscite dal salvadanaio; quando si parla di patrimonio si parla di tutta
la parte di patrimonio netto messo in un cassettino, ad esempio un
immobile che posso smobilizzare. Le immobilizzazioni possono essere
materiali e immateriali (brevetti, invenzioni). Smobilizzando il
patrimonio posso far fronte a delle uscite se non ho altro tipo di
entrate. Se si chiude in utile ad esempio, questo si mette nel
patrimonio e non nel salvadanaio.

\emph{CARRELLATA VELOCE SLIDE:}

AZIENDA = ordine strettamente economico di un istituto

AZIENDA DI PRODUZIONE = Corretta remunerazione dei prestatori di lavoro
ma soprattutto dei conferenti il capitale di rischio

AZIENDA PUBBLICA = produzione e consumo di beni pubblici e la corretta
remunerazione dei prestatori

ISTITUTO = insieme di persone, energie e beni coordinati per raggiungere
un obiettivo comune.

ECONOMICITA'= quando esiste un equilibrio dinamico nel tempo tra le
risorse impiegate nell' azienda e i beni e servizi prodotti dall'
azienda stessa per la collettività di riferimento. Esprime la condizione
di vita duratura.

EQ. ECONOMICO = i ricavi devono coprire i costi della produzione.
Nell'azienda privata i ricavi devono essere maggiori dei costi, in
quella pubblica devono essere uguali.

Devo dimostrare che lavoro in efficienza, quindi devo parlare di costi e
input /output.

Se sto parlando di efficacia metto in relazione i risultati, quindi gli
output, e gli outcome, quindi gli obbiettivi che mi sto dando (mission).
La mission di un'azienda ospedaliera è la tutela della salute della
collettività, mentre la vision, che è il modo in cui raggiungo la
mission, corrisponde a fare prevenzione.

Se efficacia ed efficienza portano all'economicità, devo tener conto che
se sono efficiente ho costi e ricavi, sto parlando di input e output, di
analisi dei costi. Se parlo di efficienza vado all'equilibrio economico,
ovvero costi e ricavi. Però devo tener conto anche dell'equilibrio
finanziario: il pagamento delle mie fatture non è immediato e nel
frattempo devo sostenere delle spese non facendo affidamento sul
pagamento che mi perviene dalla Regione. Il vero problema è quindi
l'equilibrio finanziario ovvero la disponibilità monetaria. Ultima
variabile è il patrimonio che è collegato sia all'equilibrio finanziario
che monetario. Gestendo le mie risorse riesco ad avere risorse per
gestirmi entrambi gli equilibri, lo uso come una levetta per mettermi in
equilibrio.

\end{document}
