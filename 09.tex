\section{Infezione pneumococcica, H. Influenzae tipo B, infezioni respiratorie acute}

La prima prova in itinere verterà su :
\begin{itemize}
\item Concetti generali (parte di igiene trattata nel corso integrato di
semeiotica del terzo anno, ad es. notifica, isolamento, disinfezione
ecc.)
\item Metodologia epidemiologica
\item Epidemiologia e profilassi delle malattie infettive (programma fino al
30 ottobre)
\end{itemize}

\subsection{Infezione pneumococcica}

\subsubsection{Generalità}

L'agente eziologico è \emph{Streptococcus pneumoniae}, ancora più
pesante rispetto al meningococco ed ospite frequente delle vie aeree di
soggetti normali.

La condizione di \textbf{portatore naso-faringeo} o di commensalismo è
FREQUENTISSIMA, molto più di quanto non sia la ricaduta in termini di
malattia; questo perché l'ospite, che è l'uomo, nella sua struttura ha
delle grandi difese naturali che sono: il riflesso epiglottico, il muco,
le ciglia vibratili e i macrofagi, che riescono a mantenerlo
generalmente a livello di commensale -> figura del portatore
naso-faringeo. Diventa patogeno conclamato importante quando ci sono
condizioni particolari, definite CAUSE PREDISPONENTI, in presenza delle
quali possiamo avere numerose possibili infezioni da \emph{S.
Pneumoniae.}

Nonostante queste infezioni siano molto numerose, quando sentiamo
parlare di \emph{S. Pneumoniae} pensiamo subito alla polmonite lobare
franca, la più tipica, ma in realtà ne abbiamo molte altre, classificate
in forme invasive e non invasive.

Forme INVASIVE :

\begin{itemize}
\item
  Meningite
\item
  Osteomielite
\item
  Batteriemie / sepsi
\item
  Polmonite con batteriemia
\end{itemize}
  
  Forme NON INVASIVE :

\begin{itemize}
\item
  Otiti medie, che giocano un ruolo importantissimo nel bambino
\item
  Bronchite
\item
  Polmonite non batteriemica
\item
  Sinusite
\end{itemize}
  Le due fasce d'età più implicate sono quella dei bambini e quella
  degli anziani.

  Nel bambino è l'otite media (spesso ripetuta) ad avere la massima
  incidenza, seguita poi da polmonite, batteriemia, e poi meningite, che
  rappresenta la punta della piramide (è la meno frequente) ma che è
  molto grave. La gravità della malattia è quindi inversamente
  proporzionale all'incidenza; meno grave la forma delle otiti, poi
  progressivo aumento di gravità fino ad arrivare alla meningite,
  espressione di patogenicità molto seria.

\subsubsection{Eziologia: Streptococcus pneumoniae}

  Si tratta di un diplococco (classica forma a fiamma di candela,
  lanceolato) capsulato -> vecchia dicitura: \emph{diplococcus
  pneumoniae}. Sulla capsula è importante sottolineare la presenza
  dell'\textbf{antigene capsulare} che:

\begin{itemize}
\item
  definisce il SIEROTIPO -> sono oltre 90 i sierotipi documentati, di cui
  solo un 10-25\% sono responsabili dell'80\% dei casi delle forme
  invasive.
\item
  è IMMUNOGENO -> quindi fa produrre anticorpi protettivi e battericidi
  complemento-dipendenti, cioè che svolgono la loro attività in funzione
  della perfetta efficienza della cascata del complemento.
\end{itemize}
  Questo antigene è il principale responsabile della patogenicità (ma
  non è l'unico), grazie alla sua attività anti-fagocitaria; essendo
  intracellulare può avere una capacità di offesa verso l'ospite molto
  importante. Il batterio è un abituale colonizzatore del naso-faringe
  dell'uomo, ma se da qui riesce, in virtù di fattori di rischio che
  elencheremo, a superare la barriera mucosa, può dar luogo alle due
  serie di forme citate prima:
\begin{itemize}
\item[1.] non invasive -> resta a livello locale dando luogo a otiti,
  sinusiti, polmoniti non batteriemiche, ecc.
\item[2.] invasive -> con batteriemia (invasione del torrente circolatorio),
  può arrivare sia alle meningi che al polmone
\end{itemize}

\subsubsection{Fattori di rischio o cause predisponenti}

  Il passaggio da portatore a malattia è favorito da una serie di
  fattori di rischio, molto numerosi:

\begin{itemize}
\item
  Tutte le malattie che alterano la formazione e produzione di anticorpi
\item
  Fattori che intervengono nel perfetto funzionamento della cascata del
  complemento (che è l'elemento che consente l'azione di difesa)
\item
  Fattori che alterano la fagocitosi
\end{itemize}
  Abbiamo poi anche fattori di tipo AMBIENTALE che incrementano il
  normale stato di portatore:

\begin{itemize}
\item
  Fumo (attuale o pregresso) che, intaccando l'albero bronchiale,
  facilita l'adesione all'epitelio
\item
  Traumi toracici -> spesso non ci si pensa, ma alla base di una
  polmonite ci può essere anche uno spostamento meccanico di sede dalle
  alte vie, in cui \emph{S. Pneumoniae} svolge un'azione commensale,
  alle basse vie, dove diventa un chiaro patogeno
\item
  Inalazione di sostanze tossiche quali anestetici, gas o polveri
\item
  Ospedalizzazione (l'ambiente ospedaliero di per sé non favorisce la
  difesa)
\item
  Alcolismo
\item
  Droghe
\item
  Malnutrizione
\item
  Condizioni di sovraffollamento
\item
  Cattiva igiene
\item
  Carenza di ventilazione
\end{itemize}
  Tra i fattori di rischio abbiamo poi tutto ciò che crea sofferenza o
  ferita dell'epitelio respiratorio, che porta quindi alla rottura
  dell'equilibrio tra la normale popolazione stanziale e lo pneumococco.

\begin{itemize}
\item
  Infezioni virali respiratorie : è documentato che in periodo epidemico
  da influenza, lo pneumococco è uno tra i primi agenti di sovra
  infezione batterica che possono dare origine a patologie gravi
\item
  Malattie sistemiche debilitanti:
\begin{itemize}
\item neoplasie
\item diabete (grosso fattore di rischio)
\item scompenso cardiaco
\item stasi polmonare
\item ricoveri prolungati a letto
\item broncopneumopatie sia di tipo chimico che infettivo
\item infezioni da HIV -> il soggetto HIV + presenta un rischio almeno 10
  volte maggiore di polmonite e batteriemia
\item insufficienza renale
\item incidenti vascolari
\item incidenti cerebrali
\item splenectomia
\item agammaglobulinemia
\item neutropenia
\end{itemize}
\end{itemize}

\subsubsection{Epidemiologia}

  Rappresenta la più comune causa di polmonite batterica nel mondo; fino
  agli inizi del secolo scorso (in epoca pre-antibiotica) era la
  principale causa di morte per polmonite.

  Secondo le stime dell'OMS ancora oggi le infezioni pneumococciche
  colpiscono oltre 1.600.000 persone portandole a morte, e di queste da
  0.7 a 1 milione colpisce i bambini al di sotto dei 5 anni -> quindi
  ancora gravità importante.

  Dati USA: in termini di mortalità e incidenza, le polmoniti sono molto
  frequenti, almeno 600.000 casi all'anno (comprensivi di tutte le età);
  le otiti medie però sono sempre le infezioni che vanno per la
  maggiore. Il peso di queste nell'età infantile è molto importante e
  può portare a complicanze suppurative, riduzione o perdita dell'udito
  e anche a ritardo nell'apprendimento, fino ad arrivare alla meningite.

  Una fetta importante della mortalità è espressa negli anziani oltre i
  65 anni: l'anziano è oggi il target più importante. La gravità è anche
  rappresentata dalle \textbf{sequele}, cioè da ciò che può rimanere
  dalla malattia pneumococcica. Secondo studi americani nella
  popolazione generale possono rimanere uno o più difetti neurologici,
  ritardo mentale, perdita dell'udito, convulsioni e paralisi. Per
  quanto riguarda le sequele in età pediatrica: è responsabile di circa
  l'80\% dei casi di batteriemia (con un numero rilevante di infezioni
  al di sotto dei 5 anni) e di 1/4 dei casi di polmonite. E' il
  principale responsabile di malattia INVASIVA al di sotto dei 2 anni.

  Massima incidenza in assoluto nel bambino tra 0 e 2 anni, poi inizia
  dopo i 35 e progressivamente sale raggiungendo un secondo picco di
  massima incidenza dopo i 65.

\begin{itemize}
\item
  Frequenza : Massima frequenza (60-70\%) e massima gravità nel bambino
  sotto ai 2 anni, ma il rischio permane fino ai 5. 30-40\% oltre ai 60
  anni.
\item
  Andamento: sporadico (non dà quasi mai epidemie)
\item
  Via di trasmissione: aerea
\item
  Stagionalità: inverno-primavera
\item
  Sesso: prevalenza nel maschio (2:1)
\item
  Colonizzazione: naso-faringe, che può avvenire a carico di più
  sierotipi anche contemporaneamente
\item
  Persistenza: mesi, anche molti (condizione di portatore sano)
\end{itemize}
  La condizione di portatore prevale in termini di frequenza nel bambino
  (40 \% circa), mentre negli adulti normalmente ci si attesta entro il
  10 \%. Il bambino infatti è particolarmente favorito dagli aspetti
  ambientali: la socializzazione, la scuola, gli asili nido, gli
  ambienti confinati ecc.

  \emph{La presenza di anticorpi non elimina la condizione di
  portatore.}

  La sorgente di infezione è rappresentata prevalentemente dai portatori
  (condizione di commensalismo), e anche il malato ovviamente è
  infettante (ma ha un peso epidemiologico minore, perché è noto).

  La modalità di trasmissione aerea può essere diretta o semi-diretta:
  pneumococco è molto fragile, ricorda molto meningococco, che tende ad
  estinguersi nell'ambiente, dove la possibilità di sopravvivenza è
  modica. I fattori di rischio condizionano il passaggio da portatore a
  malattia.

\subsubsection{Farmaco resistenza}

  Si è andata stabilendo progressivamente un'incrementata resistenza
  agli antibiotici, sia ai macrolidi come l'eritromicina sia alla
  penicillina. Per quanto riguarda l'espressione della farmaco
  resistenza nei confronti dei macrolidi in generale l'Italia si pone al
  quarto posto, una posizione importante. Addirittura possiamo valutare
  anche la farmaco resistenza nell'ambito dei vari ceppi; ci sono circa
  una ventina di sierotipi importanti, e vengono valutate le varie
  resistenze, di cui le più importanti sono comunque verso penicillina
  ed eritromicina.

  Anche la penicillina, seppur non a livelli stratosferici in termini di
  incidenza, ha una buona quota di farmaco resistenza. L'eritromicina
  invece ha moltissimi ceppi farmaco resistenti.

  Da dati abbastanza recenti possiamo vedere come l'eritromicina
  raggiunga livelli estremamente rilevanti, di circa il 50 \%; la
  penicillina invece rimane a livelli abbastanza stabili. In ogni caso è
  importante evidenziare come, anche se lievemente, la resistenza alla
  penicillina sia incrementata dal 2003 al triennio successivo.

  Dal 2011 al 2015 (dati incompleti) incidenza in Italia: da 700 a 1000
  casi/anno per la malattia invasiva.

  Divisione a livello nazionale delle patologie: la \% è sempre 2/3
  forme di sepsi, 1/3 forme di meningite, abbastanza omogenee nel tempo.

  Nel 2010: fra le forme batteriche invasive \emph{S. Pneumoniae} ha la
  fetta maggiore; le analisi di dati di due quinquenni dimostrano che
  pneumococco non è in calo, ma anzi è in lieve aumento.

  Incidenza in età adulta, Emilia-Romagna: cresce dai 15 anni in avanti,
  ma nella fascia dai 25 ai 64 anni il peso è rilevante.

\subsubsection{Prevenzione: profilassi}

\begin{itemize}
\item
  Misure di tipo GENERALE -> agire sui possibili "fattori di rischio"
  ambientali, modifica dello stile di vita e della qualità di vita:
  fumo, sovraffollamento, carenza di ventilazione, ambienti chiusi,
  inalazione di tossici, polveri ecc.
\item
  Misure SPECIFICHE -> immunoprofilassi attiva (vaccino)
\end{itemize}
  Vaccino profilassi: offerta attiva e GRATUITA (hanno un
  costo-beneficio enorme sia come risparmio di vite sia economico, in
  termini di cure da destinare) a tutti i bambini e anziani (dopo i 50
  anni il rischio aumenta) ENTRAMBI QUANDO sono affetti da patologie che
  espongono ad alto rischio di infezione invasiva (predisposizione al
  passaggio da colonizzazione a malattia).

  In più c'è l'offerta non obbligatoria, ma fortemente consigliata, al
  nuovo nato (entro il primo anno di vita).

  VACCINI DISPONIBILI (la capsula è polisaccaridica):

\begin{itemize}
\item
  \textbf{Vaccino 23-valente} antico, il primo in assoluto, a
  "polisaccaridi nudi". Efficacia BUONA nei soggetti di età superiore ai
  2 anni, quindi va bene nell'adulto. Si fa una dose intramuscolo e
  l'immunità dura 5 anni. E' bene non anticipare né eccedere nelle
  somministrazioni perché ci sono degli effetti collaterali.

\item
  \textbf{Vaccino eptavalente} a "polisaccaridi coniugati" (contiene un
  antigene legato ad una proteina carrier: il tossoide difterico). E' il
  vaccino per l'infanzia, buona efficacia nella prevenzione delle forme
  invasive (meningite, polmonite, sepsi), meno efficace in quella delle
  non invasive. Ricordiamo che il vaccino polisaccaridico NON funziona
  nel bambino < 2 anni, non stimola le cellule della memoria.
\item
  \textbf{Vaccino 10-valente}, che però è stato superato dal vaccino
  successivo 13-valente
\item
  \textbf{Vaccino 13-valente} a "polisaccaridi coniugati" (anch'esso
  contenente un antigene legato ad una proteina carrier tossoide
  difterico). E' un vaccino per l'adulto, protegge dall'80 \% delle
  infezioni pneumococciche nel bambino al di sopra dei 5 anni. Non è che
  non funziona nel bambino piccolo, ma non ci sono ancora prove
  sufficienti che lo dimostrino. (non è stato ancora sufficientemente
  testato nei bambini < 5 anni)
\end{itemize}

  OBIETTIVI DEI VACCINI CONIUGATI:
\begin{itemize}
\item indurre una risposata immunitaria protettiva nella popolazione
  infantile (dal 2\textsuperscript{o} mese di vita)
\item indurre una memoria immunologica, cioè uno stimolo a successive
  infezioni
\item produrre un'immunità a livello locale, dove c'è l'impianto, che
  vuole quindi ridurre lo stato di portatore, una difesa anche a livelli
  pubblici perché riduce questa condizione.
\end{itemize}
  INDICAZIONI ALLA VACCINAZIONE -> raccomandata dal Piano di Prevenzione
  Nazionale dal 2005-2007.

  > Adulti:

\begin{itemize}
\item
  Anziani oltre i 65 anni, ma già dai 50 l'epidemiologia suggerisce che
  aumenti il rischio di esposizione a condizioni che portino alla
  malattia grave
\item
  Particolare attenzione: coloro che usufruiscono di strutture sanitarie
  e/o strutture geriatriche -> la vita in comunità è già di per sé un
  fattore di rischio, perché c'è una commistione di ceppi e di sierotipi
  che qui albergano e spesso i soggetti presentano condizioni di
  fragilità che consentono un più facile instaurarsi della malattia.
\item
  Tutti i soggetti a rischio di patologie. Il piano di prevenzione
  nazionale '12-'14 si è posto l'obiettivo di \emph{"raggiungere e
  mantenere la copertura immunitaria di oltre il 95\% di nuovi nati e
  adolescenti nei confronti dell'infezione pneumococcica".} Le patologie
  sono quelle che troviamo nella circolare n\textsuperscript{o}11 del Min. della Sanità
  2001:
\begin{itemize}
\item Talassemia e anemia falciforme
\item Asplenia funzionale o anatomica
\item Broncopneumopatie croniche (asma esclusa)
\item Immunodepressione secondaria
\item Infezione da HIV
\item Diabete
\item Insufficienza renale e sindrome nefrosica
\item Alcune immunodeficienze congenite
\item Malattie cardiovascolari croniche
\item Malattie epatiche croniche
\item Perdite di liquido cerebro-spinale
\item Impianto di sensori nell'orecchio interno
\end{itemize}
\end{itemize}

  > Bambini:

\begin{itemize}
\item
  Tutti i nuovi nati
\item
  Bambini di età \textless{} 5 anni che presentano condizioni sanitarie
  che più o meno sono quelle che abbiamo già citato
\end{itemize}
  Quindi \textbf{indicazione per età e per patologia}. Non è
  obbligatoria né nell'anziano né nel bambino, ma il fatto che sia
  fortemente consigliata indica che si potrebbe raggiungere un elevato
  grado di copertura immunitaria che porterebbe ad una netta riduzione
  delle malattie gravi. Facciamo un confronto in Emilia Romagna dei
  periodi pre- e post- introduzione del vaccino (2006-2011): vediamo una
  riduzione del 55 \% del tasso di incidenza nel bambino tra 0 e 4 anni,
  mentre son rimasti abbastanza stabili i tassi di incidenza complessivi
  in adulti e anziani. Questo perché certamente il bambino è più
  facilmente "aggredibile" dal vaccino, cosa che risulta più difficile
  nell'anziano.

 \subsection{Haemophilus Influenzae}

\subsubsection{Generalità}

  \emph{H. Influenzae} è uno dei 3 grandi responsabili di gravi
  patologie in età pediatrica, in particolare pesa in modo rilevante
  sulla meningite. La meningite da \emph{H. Influenzae di tipo B,} anche
  se trattata correttamente, presenta ancora una mortalità del 5 \% e
  può dare problemi neurologici in 1/4 dei casi. In letteratura sono
  riportati i seguenti dati (prevalentemente americani) riguardo alle
  conseguenze:
\begin{itemize}
\item deficit visivi 4\%,
\item deficit uditivi nel 10 \%
\item disturbi del linguaggio
\item deficit ai test mentali dopo i 2 anni con QI \textless{} 70\% in 1/4
  dei pazienti infetti
\end{itemize}

\subsubsection{Eziologia: Hib (Haemophilus Influenzae di tipo B)}

  Questo è quello prevalentemente implicato nelle forme invasive,
  sistemiche. Deve il suo nome al fatto che prima dell'isolamento del
  virus influenzale, siccome era spesso correlato all'influenza, c'era
  l'erronea convinzione che fosse lui il responsabile dell'influenza
  epidemica (fino al 1933), mentre oggi si sa che è una sovra infezione.

  Si tratta di un coccobacillo (corto e tozzo), microaerofilo, gram
  negativo, immobile, asporigeno, spesso capsulato (la capsula anche in
  questo caso favorisce l'impianto e la virulenza), esigente
  nutrizionalmente (difficile da coltivare, comporta arricchimento del
  terreno con fattore X e V).

  Conosciamo 6 tipi di antigeni capsulari, dalla \emph{a} alla \emph{f}
  ; è l'antigene capsulare che conferisce patogenicità e fra questi è il
  \emph{b} quello più importante nell'uomo. Non esistono serbatoi
  animali, quindi lo riteniamo specifico dell'uomo.

 \subsubsection{Patogenesi}

  E' in grado di colonizzare il naso faringe. Molto frequente il ruolo
  di portatore, ma non tanto quanto il precedente. In alcuni soggetti è
  in grado di invadere il circolo e causare infezioni a distanza.
  Precedenti infezioni delle alte vie respiratorie (URI, "upper")
  possono favorire l'infezione; quindi una sofferenza dell'albero
  respiratorio offre una maggiore opportunità di adesione anche all'
  \emph{H. Influenzae.} Può causare le seguenti patologie:

  > \emph{Forme invasive}, sostenute per il 95\% da
  \emph{Hib}:

\begin{itemize}
\item
  meningite
\item
  sepsi
\item
  polmonite
\item
  cellulite
\item
  epiglottite
\item
  artrite settica
\item
  osteomielite
\item
  pericardite
\item
  ascessi
\end{itemize}

  > \emph{Non invasive}, spesso sostenute da ceppi non
  tipizzabili:

\begin{itemize}
\item
  otite
\item
  congiuntivite
\item
  sinusite
\item
  bronchite
\item
  polmonite
\end{itemize}

  In America in epoca pre-vaccinale la meningite era dovuta a \emph{Hib}
  nel 50 \% dei casi. Poi a seguire: polmonite, epiglottite, artrite e
  cellulite. In Italia lo abbiamo scoperto tardi, tant'è che è stato
  prodotto anche abbastanza lentamente il vaccino.

  Guardando l'incidenza delle varie patologie nel mondo, in Italia la
  più frequente e la più evidente è la meningite, seguita dalla sepsi
  (che peraltro è presente in tutti i paesi). In Australia ad esempio
  meningite, sepsi ed epiglottite sono più o meno analoghe. Noi siamo
  più simili ai dati europei.

  -> \textbf{Italia: meningite e sepsi +++}
\begin{itemize}
\item Sorgente di infezione: malato e portatore sintomatico
\item Via di trasmissione: goccioline e secrezioni respiratorie, via aerea
\item Stagionalità: due picchi, settembre-dicembre (quando c'è l'ingresso
  dei bambini nelle comunità) e marzo-maggio (fine inverno - primavera)
\item Diffusione: in genere abbastanza contenuta, se non nelle
  \textbf{comunità}, dove ci sono persone che vengono da vari territori
  e possono scambiarsi microrganismi patogeni.
\item Età: massima gravità per noi è \textless{} 5 anni
\end{itemize}
  Dati di alcune nazioni europee per quanto riguarda l'età: la meningite
  è dominante nel primo anno, rimane costante fino alla seconda infanzia
  e viene poi sovrastata dalla sepsi nel bambino oltre i 15 anni. Poi
  rimane consistente, un 60\% nelle altre età.

  Dal '97 in Italia è partita la sorveglianza attiva di \emph{H.
  Influenzae} e ciò ha portato ad una riduzione e ad un contenimento
  della diffusione, grazie ovviamente anche alla vaccinazione.

  Nel 2015 (dati incompleti) al di là delle fasce d'età, stranamente ha
  prevalso la sepsi (di solito è la meningite).

  FATTORI DI RISCHIO PER LE FORME INVASIVE:

\begin{itemize}
\item
  Comunità chiuse (es. collegio, camerate di militari, istituti per
  anziani, per disabili)
\item
  Ambienti affollati (es. discoteca)
\item
  Asili nido, scuole materne e ambiente scolastico in generale
\item
  Basso livello socio-economico (reddito economico, livello di
  scolarizzazione, tipo di casa, dimensioni della casa, applicazione
  delle norme di igiene, la conoscenza di queste ultime, ecc.)
\item
  Malattie croniche
\end{itemize}

\subsubsection{Profilassi immunitaria - vaccinazione}

  Per prevenire l'infezione si può in primis agire sulle condizioni
  ambientali favorenti (vedi fattori di rischio) poi esiste la
  profilassi attiva specifica per Hbi

  Già dagli anni '70 si era reso disponibile un vaccino di I generazione
  (provato in Finlandia e in America) costituito da polisaccaridi, cioè
  dall'antigene capsulare purificato; ma come sappiamo il polisaccaride
  non funziona nei bambini \textless{} 2 anni, quindi fu studiato un
  altro vaccino (in America c'era già una documentazione e
  un'epidemiologia nota della patogenicità di \emph{H. Influenzae}). E'
  nato quindi il vaccino di \textbf{II generazione} formato da
  oligosaccaridi capsulari (ottenuti mediante depolimerizzazione
  chimica) coniugati con un carrier proteico (es. tossoide difterico)
  che permettono di attivare una risposta dei linfociti T-helper con
  aumento dell'immunogenicità -> efficacia anche nel bambino piccolo e
  anche nel neonato, praticabile dal 2\textsuperscript{o} mese di vita in poi. Alla fine
  degli anni '80 sono state studiate diverse tipologie di vaccini e di
  prodotti diversi a seconda nelle nazioni, ma sempre coniugati.

  Studi che valutano l'efficacia del vaccino dal 2\textsuperscript{o} mese di vita a
  distanza di un anno dimostrano una riduzione dell'incidenza dell'80\%
  in un anno e si è visto che nelle regioni del Nord-Europa (Finlandia,
  Gran Bretagna e Islanda) si è avuta una riduzione dell'incidenza della
  malattia del 99\% nell'arco di breve tempo -> efficacia notevole.

  \emph{Efficacia = produzione di IgG, attività battericida mediata dal
  complemento.}

  Gli studi sono stati condotti per valutare l'efficacia non solo nei
  paesi industrializzati, ma anche in quelli in via di sviluppo, con
  componenti di tipo ambientale e sociale più pesante. Ad esempio studi
  in Gambia e in Cile confermano la capacità del vaccino di proteggere
  dalla meningite e anche dalla polmonite.

  Questi e altri dati hanno portato l'OMS a \emph{"includere il vaccino
  coniugato contro Hib nell'immunizzazione routinaria di tutti i
  neonati"} (1998) e da qui è partito il programma vaccinale
  nell'infanzia. Dal '97 al 2006 c'è stato quindi un incremento delle
  vaccinazioni in molti paesi.

  In Italia dal '97 ad oggi (in realtà dal '98 è partito effettivamente
  il vaccino) è evidente come l'incidenza sia inversamente proporzionale
  all'intensità della copertura vaccinale. Ricordiamo che, non essendo
  obbligatorio, l'applicazione non è sempre ottimale.

  In Emilia-Romagna l'offerta è ATTIVA e GRATUITA a partire dal 1996.

  Popolazione target:

\begin{itemize}
\item
  Bambini nel primo anno di vita
\item
  Soggetti di qualunque età appartenenti alle categorie a rischio
\end{itemize}
  EFFETTI COLLATERALI

\begin{itemize}
\item
  Reazioni locali transitorie di poco peso (dolore, indurimento), dal 5
  al 30\% dei casi
\item
  Reazioni avverse, rarissime
\item
  Reazioni sistemiche (febbre ecc.), rare
\end{itemize}
 
  CONTROINDICAZIONI: non vaccinare soggetti che

\begin{itemize}
\item
  Hanno dimostrato reazioni allergiche a componenti del vaccino alla
  prima vaccinazione
\item
  Presentano malattie febbrili in atto (moderate o gravi), questo per
  seguire le linee guida generali
\item
  Hanno età \textless{} 2 mesi (quindi sempre dal 3\textsuperscript{o} mese in avanti)
\end{itemize}
\subsection{Infezioni respiratorie acute (ARI)}

\subsubsection{Generalità}
\begin{itemize}
\item Impatto sulla sanità Pubblica a livello GLOBALE -> secondo stime
  dell'OMS queste patologie sono estremamente importanti in quanto
  responsabili di:
\begin{itemize}
\item non meno del 20 \% delle cause di morte in bambini \textless{} 5
  anni per polmoniti, bronchioliti e bronchiti
\item almeno il 13\% delle cause di morte nei soggetti oltre i 55 anni
\end{itemize}
\item Impatto sulla sanità pubblica a livello NAZIONALE (in Italia) ->
  patologie responsabili di:
  \begin{itemize}
  
\item almeno il 25 \% delle visite mediche -> 1/4 dell'impegno di un medico
  è legato a patologie dell'albero respiratorio (influenza, bronchite,
  faringite, tonsillite, otite, sinusite, ecc.)
\item un 30 \% delle assenze dal lavoro -> giorni persi di lavoro,
  importante peso economico
\item il 75 \% delle prescrizioni di antibiotici.
\end{itemize}
\end{itemize}
  Complessivamente si ritiene che un 75 \% degli interventi di un medico
  di medicina generale nella stagione invernale (ottobre-marzo) sia
  dovuto a malattie legate a infezioni dell'albero respiratorio.

  Dal punto di vista clinico suddividiamo le ARI in due gruppi:
\begin{itemize}
\item[1.] ALTE vie respiratorie (URTI, upper) più frequenti nel bambino
\item[2.] BASSE vie respiratorie (LRTI, lower) prevalgono col crescere
  dell'età, più frequenti nell'anziano
\end{itemize}

\subsubsection{Eziologia}

  Abbiamo una gamma enorme di varietà di agenti eziologici. Senza
  considerare i batteri, già solo in ambito virale abbiamo molte
  famiglie virali, all'interno di cui abbiamo moltissimi virus diversi,
  suddivisi a loro volta in tipi, sottotipi e anche varianti.

\begin{itemize}
\item
  \emph{\textbf{Influenza virus} (RNA)}, il gruppo principale di tutti
  questi virus.
\item
  \emph{\textbf{Rhinovirus} (RNA)}: dà rinite e raffreddore. Abbiamo
  oltre 100 sierotipi riconosciuti; il raffreddore non è così banale, ad
  esempio in un lattante o in altre condizioni può essere molto serio.
\item
  \emph{\textbf{Coronavirus} (RNA)}: da cui sono nati molti virus mutati
  (SARS), che ci hanno dato anche un timore di pandemia, come appunto
  per la SARS nel 2000. Ne abbiamo almeno 4, quelli noti tradizionali,
  quello della SARS e il coronavirus mutato di ora.
\item
  \emph{\textbf{hMPV},} ovvero \emph{metapneumovirus (RNA):} fam.
  Paramyxoviridae (stessa del morbillivirus e del virus della
  parainfluenza), due sono i gruppi maggiori: incrementati in studi
  recenti nella seconda infanzia in cui c'erano tante forme orfane di
  agente eziologico.
\item
  \emph{\textbf{Parainfluenza virus} (RNA)}: fam. Paramyxoviridae, di
  cui conosciamo 4 tipi importanti (dall'1 al 4) e anche alcuni
  sottotipi. Ad esempio il tipo 3 può dare bronchite, polmonite e
  laringite seria, sempre prevalentemente nel bambino.
\item
  \emph{\textbf{RSV},} ovvero \emph{virus respiratorio sinciziale
  (RNA)}: virus molto impegnativo, principale responsabile di
  bronchiolite nel lattante (nei primi due anni di vita), che all'epoca
  della sua scoperta aveva causato il cosiddetto "male oscuro" a Napoli,
  con un'epidemia in un gruppo di bambini morti per una causa
  sconosciuta, che poi si è rivelata essere questo virus.
\item
  \emph{\textbf{HPEV},} ovvero \emph{human parechovirus (RNA)}: fam.
  Picornaviridae, di nuova scoperta, sappiamo pochissimo.
\item
  \emph{\textbf{hBoV}}, ovvero \emph{human bocavirus (DNA)}: abbastanza
  recenti, documentati nella popolazioni infantile
\item
  \emph{\textbf{Adenovirus} (DNA):} i più noti, antichi come conoscenza,
  da sempre conosciuti come responsabili di infezioni delle tonsille,
  dei tessuti delle alte vie respiratorie, ma anche di patologie
  gastro-enteriche. Ci sono oltre 47 sierotipi diversamente distribuiti
  in termini di patologie e implicati in fasce d'età diversificate.
\end{itemize}
  Tutti questi sono in grado di dare un ventaglio di patologie
  dell'apparato respiratorio, con maggiore o minore incidenza, che vanno
  dal raffreddore più lieve alla polmonite.
\\\\
  Ci troviamo quindi in una miriade di problematiche perché abbiamo:
\begin{itemize}
\item[I.] Stessa sintomatologia, ma agente eziologico diverso (e qua stiamo
  parlando solo di virus senza contare i batteri)
\item[II.] Stesso agente in diverse patologie
\end{itemize}
  Dal punto di vista diagnostico è complicato, e di conseguenza anche la
  scelta terapeutica è complessa. Ecco perché oltre all'elevata
  contagiosità valida per tutti i virus, abbiamo anche delle conseguenze
  economiche, per l'impegno del medici, per l'impegno dell'ospedale nei
  casi gravi, per la spesa dei farmaci, per la giornate di lavoro perse,
  e così via.

\subsection{Influenza virus}

  Di tutti questi agenti eziologici, sicuramente il capofila è il gruppo
  dei virus influenzali.

  \emph{INFLUENZA = malattia respiratoria acuta ad eziologia virale ad
  andamento epidemico. }

\begin{itemize}
\item
  \emph{è virale}
\item
  \emph{è respiratoria}
\item
  \emph{è epidemica}, cosa invece abbastanza particolare nei virus che
  abbiamo descritto, che classicamente non danno epidemie
\end{itemize}

  Si chiama così perché è sostenuta dai virus dell'influenza; ma una
  parte rilevante di tutte le forme respiratorie febbrili che vengono
  definite "influenza" sono causate da tutti gli altri agenti eziologici
  elencati prima. Tutto questo incide sempre sulla possibilità
  diagnostica e di intervento. Si esprime sempre con focolai epidemici,
  più o meno grandi.

  L'andamento epidemiologico è giustificato dalle sue caratteristiche:
\begin{itemize}
\item elevata \textbf{contagiosità}
\item Elevata variabilità, cioè \textbf{instabilità}: il virus cambia.
  Questo è dovuto al fatto che il virus

\begin{itemize}
\item
  ha un acido nucleico a RNA, che per sua caratteristica crea frequenti
  errori di trascrizione che danno luogo a produzione di antigeni
  diversi.
\item
  non dispone di un correttore di bozze, quindi gli antigeni sono
  soggetti a modifiche, e se la modifica cade a carico di un settore
  caldo (es. sito di aggancio del virus) allora offre una chance in più
  di diffusione a quel virus rispetto agli altri.
\item
  è soggetto ad un RIASSORTIMENTO, fattore puramente fisico, ovvero ad
  uno scambio di acido nucleico da un virus all'altro (ha RNA
  segmentato).
  \end{itemize}
\end{itemize}
  Tutto questo determina un'elevata instabilità genetica che si traduce
  in:

\begin{itemize}
\item
  FOCOLAI EPIDEMICI: piccole espressioni in un nucleo contenuto di
  persone che si infettano contemporaneamente.
\item
  EPIDEMIE: più consistenti, possono coinvolgere dal 5 al 30 \% della
  popolazione (l'anno scorso l'epidemia dell'influenza ha coinvolto
  5.000.000 di persone, e non è stata la più grossa in assoluto).
\item
  PANDEMIE: 60-80 \% della popolazione del pianeta. Altro non è che
  un'epidemia estesa a livello mondiale, perché quando c'è un agente
  nuovo che si esprime, la popolazione è tutta suscettibile -> il
  rapporto infezione:malattia è praticamente = 1.

  L' influenza é un grosso problema di Sanità Pubblica per 2 grandi
  motivi:

\begin{itemize}

\item[1.]
  Impatto pesante a livello di salute sulla popolazione -> la malattia
  può comportare complicanze frequenti e gravi al punto di mostrare un
  eccesso di mortalità in periodo epidemico in soggetti ad alto rischio.
\item[2.]
  \textbf{Aspetto economico}, da non trascurare. I soldi per la Sanità
  derivano dal PIL, influenzato fortemente dall'attività lavorativa.
  Abbiamo due fronti: la non-produttività da un lato e la necessità di un intervento economico per le cure dall'altro.
\end{itemize}

  Costi DIRETTI per le cure, le visite mediche, i farmaci, le
  ospedalizzazioni. Costi INDIRETTI per l'assenza da lavoro sia per il
  malato che per chi deve curare il malato (anziani, lattanti, bambini
  piccoli, che non possiamo lasciare da soli) quindi assenteismo
  (genitori per i figli o qualcuno pagato per badare al malato, e di
  conseguenza altri costi), interruzione del servizio di persone che
  svolgono attività di pubblica utilità (postini, ferrovieri, medici,
  infermieri, ecc.) con disagi da non sottovalutare è l' interruzione
  dei servizi in periodo epidemico. Tutto ciò ovviamente ha dei costi,
  perché bisogna sopperire alle varie mancanze di personale e di
  servizi.
\end{itemize}

\subsubsection{Eziologia: Virus influenzale}

  Ha una struttura classicamente rotondeggiante, anche se a volte può
  essere tubulare o allungata. Partendo dal centro ci sono gli acidi
  nucleici: caratteristica particolare e anche abbastanza rara di questo
  virus è quella di avere un acido nucleico a \textbf{RNA segmentato},
  con 7-8 segmenti a seconda del tipo, ognuno dei quali produce un
  prodotto virus-specifico, salvo uno che produce da solo le due
  polimerasi. Quindi un segmento produce la proteina M, uno un antigene
  di superficie, uno l'envelope e così via. L'acido nucleico è protetto
  e rivestito dalla \emph{matrice} e dalla \emph{proteina M}, due
  proteine antigeniche, ovvero che possono stimolare il sistema
  immunitario, ma NON tutte immunogene. Sopra alla proteina M abbiamo un
  \emph{peplos o pericapside} di natura in parte virale e in parte della
  cellula in cui si è replicato il virus -> la parte tipica del virus è
  rappresentata dalle \emph{spicole} in superficie, funzionalmente e
  strutturalmente diverse: la \textbf{neuraminidasi} (funghi come forma
  identificativa) e la \textbf{emoagglutinina} (bastoncelli).
\begin{itemize}
\item
 L'\emph{EMOAGGLUTININA (H)} è 5 volte più numerosa sulla particella
  rispetto alla neuraminidasi e questo perché è da lei che dipende la
  possibilità di aggancio del virus alla cellula. Ha due ruoli:
  \begin{itemize}
  
\item
  Aggancio alla cellula sensibile e creazione di un contatto stabile,
  primo passo dell'infezione.
\item
  Stimolo del sistema immunitario nella produzione di anticorpi
  anti-infettivi, perché se c'è l'anticorpo questo si lega
  all'emoagglutinina, e quindi il virus non la ha più a disposizione per
  legarsi alle cellule dell'albero respiratorio -> azione protettiva
  diretta mirata.
\end{itemize}
  Il nome "emoagglutinina" deriva dal fatto che questo recettore trova
  l'anti-recettore a cui legarsi su moltissimi tipi di globuli rossi,
  come quelli umani, di cavia, aviari (uccelli, anitra, oca, pollo). Si
  lega ai recettori presenti su tutta la superficie del globulo rosso
  facendo in modo che il virus si distribuisca su tutta la superficie,
  ma in alcuni casi il virus si viene a trovare a contatto con due o più
  globuli rossi: in questo caso li aggancia entrambi formando un ponte
  (AGGLUTINAZIONE) e tenendoli così spaziati tra loro, non più liberi di
  sedimentare e formare un pacchettino omogeneo di globuli rossi (che
  tornerebbero in sospensione normalmente se noi sbattessimo la
  provetta), ma agganciati e distribuiti con una trama direzionale di
  superficie nel contenitore. E' un fenomeno che in natura serve ad
  agganciare gli anti-recettori e legarsi alle cellule dell'ospite, ma
  che in laboratorio sfruttiamo molto per identificare la presenza del
  virus nel materiale biologico e per riconoscerlo fra tanti altri,
  proprio perché il fenomeno dell'emoagglutinazione fa sì che in
  presenza di virus e di globuli rossi (che sono il sistema rivelatore),
  questi non sedimentino al fondo del pozzetto della provetta, ma
  vengano distribuiti in modo omogeneo e uniforme, dal momento che sono
  tenuti spazialmente legati e separati dall'emoagglutinina -> grande
  utilità per la ricerca e per la diagnostica.

\item La \emph{NEURAMINIDASI (N)} è meno numerosa ed ha una funzione
  enzimatica, ovvero è in grado di idrolizzare l'acido sialico:
  componente dei muco-peptidi presenti sulle cellule dell'albero
  respiratorio, ma anche del muco stesso.

  Sfruttando questa capacità, svolge così due funzioni principali:

\begin{itemize}
\item
  Sganciare il virus dalla cellula ospite infettata e favorirne l'uscita
  dalla cellula, così da permettere la diffusione del virus ad altre
  cellule.
\item
  Fluidificare il muco (idrolizza l'acido sialico del muco), riducendo
  la capacità di difesa dell'ospite ed esponendo così altre cellule al
  virus, intensificando la diffusione e la gravità.
\end{itemize}
  Ha un'azione diretta di liberazione e di diffusione, è immunogeno e di
  conseguenza fa produrre anticorpi, che ci proteggono indirettamente,
  perché evitano la liberazione del virus e quindi ne riducono la
  diffusione -> importante nella terapia, infatti i farmaci moderni sono
  anti-neuraminidasi.
\end{itemize}
  Il processo di infezione avviene in virtù dell'emoagglutinina, che si
  aggancia ai recettori cellulari formati da muco-peptidi contenenti
  acido sialico, che nell'uomo sono al 99\% recettori specifici per i
  virus UMANI (solo una piccolissima quantità è anche per altri virus).
  Una volta legati, si viene a formare una vescicola, il virus si sveste
  e libera il suo acido nucleico, che raggiunge le sedi proprie per la
  replicazione. Si replica e produce in sedi diversi i suoi prodotti:
  gli acidi nucleici e gli antigeni virus-specifici (emoagglutinina,
  neuraminidasi, ecc.), i quali vengono portati e inseriti a livello di
  membrana. Si viene quindi a formare il virus dell'influenza sfruttando
  parte della cellula: il peplos è infatti in parte di origine virale,
  in parte cellulare. Dopodiché la neuraminidasi favorisce il distacco,
  il virus è libero e va ad invadere altre cellule, a meno che non ci
  siano degli anticorpi, che a quel punto bloccano l'attività di
  diffusione.

\subsubsection{Classificazione dei virus influenzali}

  Si tratta di un virus dotato di componenti geneticamente stabili ed
  altre fortemente instabili. Quelle stabili sono costituite dal core
  del virus, cioè dalla ribonucleoproteina e dalla proteina M di
  membrana. Queste componenti possono essere di 3 tipologie: \emph{A},
  \emph{B} e \emph{C} -> abbiamo 3 virus dell'influenza in funzione di
  queste componenti stabili. L'impatto grosso è dato dai tipi \emph{A} e
  \emph{B}, il ruolo epidemiologico del \emph{virus C} è modestissimo
  perché non è epidemico, è sporadico e infrequente.

  I virus di tipo \emph{A} hanno la caratteristica di suddividersi in
  sottotipi, perché cambiano gli antigeni variabili, che sono quelli di
  superficie, ovvero l'emoagglutinina (H) e la neuraminidasi (N).

  Prendono dei numeri diversi a seconda della tipologia di H o di N.
  Quelli noti per l'uomo sono:

\begin{itemize}
\item
  H3/N2
\item
  H1/N1
\item
  H2/N2 (questo è stato il grande virus pandemico del '57 ed è oramai
  scomparso dalla scena epidemiologica)
\end{itemize}
  Sia l'H sia la N del \emph{virus A} e del \emph{virus B} possono
  andare incontro a dei piccoli arrangiamenti chiamati VARIANTI, che
  danno luogo ad un numero infinito di virus.

  -> sottotipi solo per il \emph{virus A}

  -> varianti sia per \emph{A} che per \emph{B}

\paragraph{Nomenclatura dei virus influenzali}
\begin{itemize}
\item Tipo: \emph{A} o \emph{B}, in base alle componente stabile del virus
\item Luogo geografico in cui è stato isolato quel ceppo virale (Moscow,
  Salomon, ecc.)
\item Numero di isolamento -- ceppo - (se ne isolo 40, 50 o 100, ognuno ha
  un numero, sono tutti diversi)
\item Anno di isolamento
\end{itemize}
  In più per tutti i \emph{virus A} è indispensabile la
  sottoclassificazione, cioè il sottotipo (identificazione di H e N).

  es: tipo A / città isolamento: Moscow / ceppo :21 / anno di
  isolamento: 99 / sottotipo: H3N29

  Il cuore del problema dell'influenza è l'\textbf{instabilità}. Abbiamo
  due tipi di variazioni:

\begin{itemize}
\item
  \emph{SHIFT}: variazioni maggiori. Possibili solo nel tipo \emph{A.}
  Si tratta di una sostituzione completa o di uno o di entrambi gli
  antigeni di superficie, che cambiano totalmente, dando un virus nuovo,
  responsabili di SHIFT o "cassure" (in francese, rottura). Causano
  quindi PANDEMIE, perché la popolazione mondiale non ha mai incontrato
  questo tipo di virus ed è tutta suscettibile.
\item
  \emph{DRIFT}: variazioni minori, legate ad un'instabilità genetica,
  derivate da errori di replicazione. Possono esserci sia nel tipo
  \emph{A} che nel tipo \emph{B}. Sono piccoli cambiamenti,
  insignificanti di per sé (prevalentemente a carico di H, ma anche N),
  ma che se avvengono in punti caldi possono giocare a favore del virus,
  nel senso che può agganciarsi a recettori diversi. Danno luogo a
  FOCOLAI EPIDEMICI ed EPIDEMIE, tanto più epidemie quanto più è
  significativa questa variazione minore. Le variazioni minori sia di
  tipo A che B portano alla nascita di un virus con antigeni di
  superficie lievemente modificati rispetto al parentale. Questo è
  dovuto a mutazioni puntiformi nei geni dell'H e della N, che di
  conseguenza producono antigeni lievemente modificati, ma che possono
  offrire un vantaggio nella diffusione del virus in una popolazione
  estesamente immunizzata. Questo si traduce annualmente nelle epidemie
  stagionali, in cui un virus leggermente mutato si inserisce in una
  popolazione (protetta contro il virus degli anni precedenti) e a
  seconda del grado di variazione colpisce tutti o no. L'anno successivo
  succederà la stessa cosa, gli anticorpi pregressi copriranno gli
  analoghi di cui avranno già visto gli antigeni, ma i pochi virus
  variati troveranno la possibilità di esprimersi. E così andremo avanti
  con periodi INTERPANDEMICI, che intercorrono da una pandemia
  all'altra.
\end{itemize}
  In senso generale possiamo parlare di "malattie simil-influenzali"; in
  tutte le stagioni, tutti gli anni, poco o tanto, in misura lieve,
  modesta o grave, abbiamo episodi epidemici.

  Il risultato è sempre un cambiamento di struttura del virus, ma
  l'origine del cambiamento può essere diversa:

\begin{itemize}
\item
  Una CO-INFEZIONE (prevalente), cioè una tecnica di
  \textbf{riassortimento} di materiale \textbf{genetico} del virus. Lo
  shift è favorito dalla presenza di virus segmentato (un segmento
  produce H, un altro N, ecc.) e anche dalla disponibilità di un AMPIO
  serbatoio animale per i virus ti tipo A. Se nel contempo la cellula è
  stata infettata da due virus, e se di specie diverse ancora peggio, ci
  sono il doppio di segmenti diversi che nell'atto della formazione del
  virione nuovo possono per errore essere costituiti da materiale
  genetico proveniente dalle due replicazioni separate che si associano,
  con un mix di segmenti. Gioca molto il tipo di segmento che si
  scambia, perché se si scambiano ad esempio i segmenti della proteina
  M, potrò avere prodotti abortivi, mentre se si scambiano quelli di H
  e/o N il risultato sarà produttivo. Il riassortimento non è dovuto
  solo al fatto che i virus sono costituiti da tanti segmenti, ma anche
  dal fatto che il virus, e SOLO il virus di \emph{tipo A}, ha un
  serbatoio animale enorme (dove li abbiamo cercati li abbiamo trovati):
  volatili, animali acquatici, vertebrati, cavallo, suino, rettili, ecc
  -> e da qui possibilità di comparsa e riassortimento semplice. Il
  serbatoio animale è anche alla base del salto di specie, che è
  eccezionale, perché la barriera di specie è sempre molto forte, ma è
  comunque possibile.

\item
  \textbf{Un salto di specie}, cioè un passaggio diretto animale-uomo
\item
  Infrequente, ma che può esserci, \textbf{ricomparsa a distanza} di
  generazioni di un virus antico, non nuovo in senso assoluto, ma nuovo
  per quella popolazione, che non ha un'immunità in grado di
  contrastarlo.
\end{itemize}

\subsubsection{Ecologia dei virus influenzali}

  Abbiamo un grande serbatoio animale del virus, che vede come
  componente prioritaria nella diffusione gli \textbf{uccelli
  migratori}, i quali si infettano dall'ambiente in cui vivono o dagli
  animali selvatici infetti del territorio, che sono prevalentemente
  aviari (anitre, polli, uccelli vari), che normalmente si infettano ma
  non si ammalano. Gli uccelli migratori possono così trasferire
  l'infezione ad altri territori, in cui non solo infettano altri
  aviari, ma possono infettare anche aviari "domestici", che rientrano
  negli animali da cortile, in primis galline e anitre poi anche cigni,
  ma anche vertebrati come il suino (importantissimo) e il cavallo. Se
  condividiamo l'habitat con questi animali, la possibilità di
  assortimento diventa massima. Questo aspetto di condivisione era molto
  presente nel passato, ma nei paesi del Sud-Est asiatico c'è ancora
  l'abitudine di comprare l'animale vivo e di tenerlo a casa, accudirlo
  e poi sacrificarlo, con grande possibilità di contagio, di liberazione
  di virus in tutti i momenti, compreso quello dell'uccisione, della
  spennatura ecc., che possono dar luogo anche al salto di specie.
  Questo è il circuito ambientale dei virus influenzali: nascita di un
  nuovo virus, epidemia, diffusione nella popolazione umana. Non a caso
  i ceppi pandemici nascono quasi sempre nel Sud-est asiatico (Hong
  Kong, Singapore), perché qui c'è questo habitat ideale e questa
  possibilità di cambiamento.
