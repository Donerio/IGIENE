\documentclass[]{article}
\usepackage{lmodern}
\usepackage{amssymb,amsmath}
\usepackage{ifxetex,ifluatex}
\usepackage{fixltx2e} % provides \textsubscript
\ifnum 0\ifxetex 1\fi\ifluatex 1\fi=0 % if pdftex
  \usepackage[T1]{fontenc}
  \usepackage[utf8]{inputenc}
\else % if luatex or xelatex
  \ifxetex
    \usepackage{mathspec}
  \else
    \usepackage{fontspec}
  \fi
  \defaultfontfeatures{Ligatures=TeX,Scale=MatchLowercase}
\fi
% use upquote if available, for straight quotes in verbatim environments
\IfFileExists{upquote.sty}{\usepackage{upquote}}{}
% use microtype if available
\IfFileExists{microtype.sty}{%
\usepackage{microtype}
\UseMicrotypeSet[protrusion]{basicmath} % disable protrusion for tt fonts
}{}
\usepackage[unicode=true]{hyperref}
\hypersetup{
            pdfborder={0 0 0},
            breaklinks=true}
\urlstyle{same}  % don't use monospace font for urls
\IfFileExists{parskip.sty}{%
\usepackage{parskip}
}{% else
\setlength{\parindent}{0pt}
\setlength{\parskip}{6pt plus 2pt minus 1pt}
}
\setlength{\emergencystretch}{3em}  % prevent overfull lines
\providecommand{\tightlist}{%
  \setlength{\itemsep}{0pt}\setlength{\parskip}{0pt}}
\setcounter{secnumdepth}{0}
% Redefines (sub)paragraphs to behave more like sections
\ifx\paragraph\undefined\else
\let\oldparagraph\paragraph
\renewcommand{\paragraph}[1]{\oldparagraph{#1}\mbox{}}
\fi
\ifx\subparagraph\undefined\else
\let\oldsubparagraph\subparagraph
\renewcommand{\subparagraph}[1]{\oldsubparagraph{#1}\mbox{}}
\fi

% set default figure placement to htbp
\makeatletter
\def\fps@figure{htbp}
\makeatother


\date{}

\begin{document}

16/01/2017

Docente: Dr Angelo Campanini (MMG)

Argomento: storia della medicina, parte 1

STORIA DELLA MEDICINA (sezione)

Cercheremo di capire cos'è la medicina generale a partire dai
presupposti culturali della medicina fin dagli esordi.

La medicina in senso lato, quella che veniva definita medicina teorica,
era un'arte filosofica, e quindi era basata sul ragionamento, sulla
silloge, sulla logica, non era basata assolutamente sul visitare il
paziente.

L'assistenza primaria era concepita agli esordi in maniera molto diversa
da quella di oggi, che nasce da concetti della filosofia illuminista che
la fiondò direttamente dentro alla dichiarazione dei diritti dell'uomo e
che non ha niente a che vedere con quello che veniva fatto nei secoli
passati.

Alcuni concetti che spesso siamo portati a considerare come sinonimi
sono la salute o sanità e la medicina. Attengono ovviamente allo stesso
campo, però la salute è un fenomeno naturale, sociale, ha delle
connessioni economiche e ha nella biologia e nell'ambiente i suoi
elementi in parte fondativi, in parte di disturbo, che causano un
disequilibrio. La medicina invece è l'epifenomeno che si modella su
quello che trova, sulla mancanza di salute e cerca in qualche modo di
operare per riportare lo stato di salute.

Le variabili storiche sono di tipo individuale e di tipo collettivo. Il
medico di medicina generale ha sì una responsabilità individuale nei
confronti di ciascuno dei pazienti, ma in realtà abbiamo una
responsabilità di tipo collettivo, (ad esempio abbiamo finito la
campagna vaccinale antinfluenzale).

I rapporti di questi due mitili semantici sono poi immersi in un
contesto culturale, che è la cultura di tutti i giorni.

Il concetto di salute è stato fissato da rivoluzionari, massoni
all'interno del processo della Rivoluzione Francese ed è stato piazzato
nel 1794 dentro la dichiarazione dei diritti con uno scopo pratico:
quello di garantire una istruzione dei medici basata non più su aspetti
di tipo filosofico ma piuttosto di tipo pratico. Qui sono state
inventate le \emph{ecoles de salut.} Quindi il legislatore che era un
Giacobino ritorna all'antico insegnamenti ippocratico e questo è
importante perché allora molte delle facoltà di medicina erano non
controllate ma comunque indirettamente influenzate dalla Chiesa, non
solo quella cattolica romana: infatti molti dei riformisti, ad esempio
calvinisti a Ginevra, hanno mandato al rogo alcune personalità.

L'insegnamento ippocratico aveva negato l'interferenza divina nella
genesi delle malattie (è solo uno dei tanti aspetti ippocratici).
Ippocrate considerava inscindibile il discorso di salute e malattia
perché l'ambiente per lui era importantissimo: poteva causare anche per
eventi storici o politici malattie nella popolazione.

Si può ragionare sulla storia della medicina, ma anche analizzare ciò
che accade oggi nella medicina attraverso un sistema di assi ortogonali
``logici''. Sono degli assi che vanno immaginati in movimento, non sono
statici e si possono incrociare.

Gli assi rappresentano:

-epidemiologia: che cosa succede, gli elementi che determinano delle
turbe nel benessere dell'individuo e della società (nel tempo si sono
viste peste, lebbra, sifilide ecc)

-sistema storico-antropologico nel quale ci si muove, che poteva essere
la scuola salernitana allora, o il sistema di accoglienza di Ippocrate a
Kos, la nascita delle università, la nascita degli ospedali (ospedale
deriva da \emph{hospitalitas}, non aveva in sé il concetto di cura),
strutture di controllo, le accademie

-tecnologia. Molto importante è la dimensione temporale, c'è una
evoluzione costante

La teoria un tempo era una cosa fantasmagorica, con delle costruzioni
mentali straordinarie, che ci stupiscono per il rigore razionale, ma la
prassi era pari a zero (vedi cosa facevano con la peste).

Studiando la storia della medicina, ci si accorge che essa non è
progressiva come direbbe Croce: se la rappresentiamo su un grafico si
vede che cerca di salire, e sale progressivamente ma ogni tanto ha dei
tonfi che la riportano indietro.

Un esempio di partenza è il periodo della peste, quello che impatta sul
periodo che è diventato uno dei più fulgidi per la storia italiana, che
è il Rinascimento. La peste era terrificante, con periodo di incubazione
pari a zero, trasmissione immediata, propagazione micidiale, acuzie del
quadro clinico, prognosi fatale. Peste e morte erano sinonimi.
L'aspettativa di vita era di 30 anni (1300), il secolo dopo
l'aspettativa di vita cade a 20 anni: le cose non stavano andando bene,
se c'era tutta questa spettacolare costruzione filosofica ma si moriva a
20 anni. L'obiettivo era quello di accertare una discrasia, cioè un
venir meno dell'equilibrio dell'organismo. Sono termini che usiamo
ancora oggi (es. edemi discrasici). Erano discrasici in quanto si
alterava l'equilibrio degli umori. Tutta questa teoria non aveva nessun
impatto pratico. Questa emorragia di vite portò via il 35\% della
popolazione europea.

{[}Mostra alcune rappresentazioni della peste e alcune descrizioni di
essa in varie epoche{]}

Il primo che iniziò a pensare a qualcosa di innovativo sulla peste fu
Fracastoro, il quale però ebbe intuizioni meravigliose che non si
tradussero in nulla di pratico (ad esempio prescriveva un pasticcio di
millepiedi).

La teoria che cade è quella di Ippocrate e Galeno, conosciuti al tempo
nella versione di Avicenna, che era il più importante traduttore del
tempo. Dicevano che ci sono tre spiriti: lo spirito animale, che ha sede
nel cervello, sede delle sensazioni e dei movimenti, (era stata creata
anche una giustificazione sul piano anatomico grazie alla \emph{rete
mirabilis}, che era a cavallo tra le carotidi e il cervello, Galeno
aveva studiato l'anatomia sui macachi); lo spirito vitale, lo
\emph{zòticon} (notare come questi temine si sono trasferiti nel
linguaggio comune: lo zotico è qualcosa di poco intellettuale, legato al
fisico), ha sede nel cuore e regola il calore del corpo; infine lo
spirito naturale, che risiede nel fegato e ha il compito di
metabolizzare, è il centro del ricambio.

Il pensiero di Ippocrate fu enorme perché puntò l'attenzione su delle
cause naturali e non divine delle malattie. E queste alterazioni si
possono studiare osservando, sia il paziente sia l'ambiente (ad esempio
lui dà molte colpe all'aria).

Il C\emph{orpus Hippocraticum} era composto di nove tomi, quindi non è
facile riassumerlo.

Le regole si basavano su:

-teoria degli umori: quanti e quali erano. Vi era associata tutta una
semeiotica particolare

-condizionamenti esterni e interni sugli umori

-c'è una corrispondenza costituzionale sia delle aree e degli umori che
compongono il soggetto

-distribuzione geografica: sapevano che in alcune aree si ammalavano di
più, inoltre avevano visto che alcune malattie avevano una cadenza
periodica, una certa stagionalità attribuita a congiunzioni astrali.
Questa cosa condizionò la medicina fino al `600, in cui si credeva che
se un particolare infuso veniva fatto nel momento sbagliato perdeva la
sua efficacia.

Secondo questa teoria degli umori le malattie si originavano da uno
squilibrio tra sangue, flegma, la bile bianca, la bile nera. I quattro
umori determinano salute se sono in equilibrio, oppure malattia se si
squilibrano, questo era il concetto.

La meteorologia incideva, il mescolamento di questi umori era viziato
dall'aria, perché se un male si diffonde così rapidamente e colpisce
tutti, allora doveva dipendere da qualcosa che era in contatto con
tutti, cioè l'aria. Questa teoria ``aerista'' della genesi di queste
malattie va avanti fino a quasi il `700. Nonostante qualcuno abbia avuto
delle intuizioni diverse, l'attaccamento alle \emph{auctoritas} non ha
prodotto cambiamenti.

Il fatto che le cose potessero andare male in relazione al clima sono
state descritte benissimo anche da Polibio, che fu il caposcuola a Kos
dopo Ippocrate.

C'era inoltre il concetto che gli umori corrotti devono essere eliminati
dagli organi emuntori (bocca, naso, sfinteri). Il bubbone infatti nella
peste era considerata la strada che aveva l'umore corrotto di essere
eliminato. Il concetto dell'aria come causa della peste la ritroviamo
anche in Manzoni, bisogna arrivare all'800 nella ultima grande epidemia
di peste cinese nella quale Yersin da una parte e Kitasato dall'altra
scoprirono, indipendentemente l'elemento causale, la Yersinia Pestis.

Galeno ci ha messo del suo: era un anatomista venuto a Roma dalla
periferia dell'Impero. Era un filosofo di scuola aristotelica,
appassionato di coltelli e che lavorava per la scuola dei gladiatori,
quindi di strumenti per tagliare ne aveva. In più, siccome c'era il
mercato degli animali per il circo, aveva circa 200 scimmie all'anno da
sezionare. È un'anatomia comparata. Un altro prima di Galeno fu Celso,
che con la nascita della stampa fu pubblicato anche prima di Galeno.
Galeno non faceva il medico del popolo ovviamente, ma era il medico di
Marco Aurelio, e lo aveva reso un tossicodipendente in quanto curava i
suoi dolori con vino fermentato, oppio. Quindi capite che l'Impero in
quel tempo è stato diretto da uno così.

Il pensiero di Ippocrate e Galeno ritorna violentemente sulla scena,
attraverso gli interpreti arabi in un signore che si chiamava Gentile da
Foligno, definito il ``divino principe dei medici''. Studia ovviamente
sotto l'egida di Vicenna. Riprende i concetti dell'umore corrotto che
deve essere eliminato attraverso i bubboni, retroauricolari, ascellari,
inguinali. Si chiudeva perfettamente la descrizione, stando dentro ai
principi di Ippocrate e Galeno.

E il medico che faceva? Si mettevano un cappotto incerato, per far sì
che l'aria scivolasse, scarpe alla polacca che potevano essere
raccordate con i calzoni di pelle, camicia di pelle perché ritenevano
che non avesse pori e poi il bastone per evitare di toccare il malato. E
poi avevano il becco d'uccello (quello che si vede ancora nelle maschere
al Carnevale di Venezia) che aveva in punta una spugna aromatizzata per
non sentire il fetore. Inoltre mettevano dei cristalli sugli occhi per
paura che l'aria penetrasse anche attraverso gli occhi. Si limitavano a
dare una spiegazione del perché avvenisse la patologia. Ma non facevano
nulla se non l'ispezione. Le terapie erano: fuggire in luoghi dove
l'aria era sana, buona dieta, buona condotta perché ciò che poteva
determinare una discrasia era il peccato.

Marsilio Ficino tra i consigli che dava per curare questa malattia
diceva di aiutare il malato a eliminare gli umori corrotti: quindi
consigliava sostanzialmente una purga.

Come medicine davano: terra, pietre perché erano secche e umide, quindi
era il contrario di quello con la febbre che si stava asciugando. Poi
andava di moda il bolo armeno di amatite e argilla, che altro non è un
composto che, ancora oggi, si usa per dorare gli oggetti. Anche la terra
sigillata si usava, ma era molto sofisticata perché doveva essere
scavata a Lemno (arcipelago di Kos) e funzionava solo se si prendeva in
determinati periodi per via delle congiunzioni astrali. Anche molti
animali si usavano, tra cui la più importante era la \emph{Triaca
Maggiore,} un intruglio composto almeno di 50 alimenti che più si
lasciava lì e più era efficace. L'elemento più importante era il veleno
di vipera e un trito di vipera essiccata.

La trattatistica non aveva portato a nulla di buono dunque. Tutti quelli
che avevano avuto qualche idea diversa hanno fatto una brutta fine.

Dal momento che c'era una nosografia sempre uguale, la malattia si
riproduceva uguale, a Milano, città leggermente autoritaria, il Duca
decise di sbarrare le case dei malati di peste, che venivano sbarrati
dentro casa. Coloro che dicevano chi era appestato e chi no erano i
dottori, ma la corruzione presente tra essi fece comunque sì che molti
casi non fossero segnalati (i ricchi li pagavano per non farsi
segnalare).

Siamo ancora lontani dall'anatomia sul malato. Il fallimento della
medicina era totale, ma i medici si ``salvavano'' in quanto detenevano
le teorie.

Qualcosa inizia a cambiare con la nascita dell'Università di Bologna,
fondata nel 1088, che in termini concorrenziali surclassa Salerno, che
fino ad allora era il centro per la cultura medica e insieme a Padova
diventa il centro più importante a livello europeo per la formazione
medica.

Ma cosa si faceva all'epoca nel corso di medicina a Bologna? Faceva una
\emph{lectio}, che significava prendere un testo di Avicenna, che si
commentava in aula e a casa si imparava. Il grosso vantaggio rispetto
alle altre scuole presenti all'epoca fu questo: generare dei dubbi e
discuterne, cioè introdurre la \emph{questio e la disputatio.} Prima
semplicemnte si imparavano le \emph{auctoritas. }

Ci furono due grossi tentativi di rinnovamento: Taddeo Alderotti che
dice che non basta leggere, ma bisogna osservare e confrontare con
quello che si ha letto: in base a questo disputo. L'altro era Pietro
Davalo, che andò alla fine del '200 a Padova, che era una affiliazione
di Bologna, che dice che bisogna prendere in considerazione le cose che
si vedono. È una considerazione universale: vedere, ascoltare il malato,
dare ragione al malato fino a prova contraria. Questi riformano il
metodo medico. Nel frattempo il chirurgo era considerato di bassa lega.
Il primo a cambiare qualcosa fu un certo Mondino. Egli fa dei
cambiamenti fondamentali nella modalità di lezione di anatomia: il
paziente viene messo supino su un tavolo. Questa è una rivoluzione
perché una volta li appendevano. Durante la lezione quindi c'era un
\emph{lector}, che dall'alto dell'aula leggeva il testo di Avicenna, uno
che sezionava il corpo mentre il lector leggeva (\emph{sector}), e un
\emph{ostensor} che aveva un compito difficilissimo: doveva far vedere
con una bacchetta o una pinza di che cosa stesse parlando il
\emph{lector} dall'alto. Ma ci fu ancora un'occasione perduta perché in
quell'epoca Guglielmo da Saliceto (l'ospedale di Piacenza ha questo
nome), Rugerius Parmensis, che probabimente ha una discendenza tedesca
perchè era arrivato con l'esercito dell'imperatore Federico II, e poi si
è fermato a Parma e ancora tutta la scuola di Parma e Bologna continuata
poi da Lanfranco alla fine del `300 avevano già delle basi molto valide
costruite \emph{in bellum}, cioè durante la guerra, di anatomia
chirurgica e le portarono in giro per le varie università, però ancora
il concetto della corrispondenza era il pallino dei medici alla fine del
`300, quindi i concetti elencati prima. Qualche passo avanti comunque si
era fatto.

I medici come erano articolati? C'erano i professori della scienza,
quindi quelli che facevano della filosofia, della medicina teorica, e un
gradino sotto quelli che facevano la medicina pratica, sotto c'erano gli
empirici, che erano quelli che venivano a contatto tutti i giorni coi
malati che venivano presentati in varie situazioni, nei conventi, nei
ricoveri a comunali. Gli empirici ``rompevano le scatole'', c'era una
lotta all'esercizio abusivo della professione. I chirurghi erano
considerati tali, ma anche i flebotomi, che erano quelli che eseguivano
il salasso su indicazione del medico, che invece non li faceva. E poi
c'erano i barbieri che avevano funzioni varie. Quindi c'era un sistema
medico, uno paramedico e uno inframedico. Il sistema medico dominava
tutto, perché con quella costruzione teorica si era in grado di spiegare
tutte le malattie, che poi non fossero in grado di fare nulla sul piano
pratico è un altro discorso. Il sistema era fallimentare, questo portò
gli amministratori a mettere degli ufficiali responsabili della salute
pubblica: nel 1370 Barnabò Visconti (ducato Milano, il più grande del
nord Italia) ordinò di limitare l'accesso ai sospetti ai confini.
Inoltre nacquero gli elenchi dei malati, dove ogni parrocchia doveva
avere il suo elenco. Inoltre con gli altri Ducati avevano messo in piedi
un notiziario della salute, che era governato in questo modo: Genova e
Venezia chiudono il porto a navi che provengono da alcune località,
Ragusa inventa la chiusura del porto per le navi che vengono dal Medio
Oriente e le tiene fuori porto 40 giorni: inventa la quarantena. Ma in
realtà ancora una volta ritorna Ippocrate, che diceva che se una
malattia è tale per cui in 40 giorni il paziente non è morto, vuol dire
che la malattia è una malattia cronica. Venezia crea dei lebbrosari,
prende un'isola e ci fa un muro e all'interno mette gli appestati. Chi
controlla tutto ciò? Il magistrato di salute pubblica, che sono
amministrativi, lo speziale, il medico e un cerusico.

Nel `400, terza grande epidemia, si comincia a dire che gli ospizi fatti
fino ad allora non vanno bene, c'è un problema igienico. Di conseguenza
Francesco Sforza, duca di Milano, prende filarete, medico e architetto
fiorentino e gli fa costruire la Ca Granda di Milano, con una serie di
polemiche infinite perché venivano sottratti fondi per gli altri
ospedali. La Ca Granda era meravigliosa: aria, finestre, raccolta dei
liquami che andavano a finire in un posto certo. Per la prima volta un
albergo dei poveri diventa la prima fabbrica della salute. Gli italiani
furono invidiati da tutti gli europei per secoli per questi ospedali che
secondo molti non aveva neanche il re di Germania.

In questa epoca si sono gettate le basi che hanno portato alla polizia
medica del `700 e all'igiene dell'800. Nel `500, quando cominciavano ad
arrivare testi originale, accadde che nel tentativo di conciliazione tra
chiesa ortodossa e chiesa romana arriva Giovanni Ottavo Paleologo che
determina il ritorno dei testi di Ippocrate e Galeno.

Qualche cosa potrebbe cambiare a Bologna, dove accade una cosa
straordinaria: Monaldo descrive la sifilide, che si andava sempre più
diffondendosi in Italia, andando a vedere tutti i termini che trovava in
Ippocrate, Galeno, Avicenna: fa un lavoro di pulizia etimologica e si
accorge che i diversi autori danno significato diverso alle parole. Ma
soprattutto Gian Battista del Monte detto Montano a Bologna dice che gli
studenti devono andare in reparto e visitare i malati. Giovanni Rasori,
250 anni dopo lo onora come colui che per primo portò gli studenti al
letto del malato. Se due personaggi come Vesalio e Montano si fossero
incontrati, visto che erano entrambi a Bologna, si sarebbe fatto un
incredibile passo in avanti, visto che si sarebbero incontrate la
perfetta conoscenza dell'anatomia e l'inizio della medicina clinica. M
ciò non accade. Vesalio pubblica un'opera con una \emph{pars destruens},
in cui smentisce gli errori di Galeno, e una \emph{pars construens}: la
lezione di anatomia non viene fatta più con \emph{sector, lector e
ostensor}, ma egli è allo stesso tempo colui che seziona e spiega agli
studenti. La trattazione anatomica è destinata a \emph{medicinam
prfessionem}, senza la conoscenza dell'anatomia il discorso elaborato
dalle \emph{auctoritas} sacre non aveva alcun valore.

Per quanto riguarda il concetto di contagio, il primo a pensare che la
sifilide potesse essere contagiosa fu il già citato Fracastoro, ma
sbagliò il vettore, in quanto riteneva che fosse la \emph{pulex}.
Fracastoro era colto, fa una descrizione della Sifilide in versi. Rende
anche a omaggio a Lucrezio nel \emph{De Rerum Natura}.

L'ufficio di sanità pubblica, non il medico singolo si riafferma nel
1590 con il compito di vigilare sui luoghi pubblici, le strade, le
derrate, gli ospedali, i cimiteri. E poi anche alle porte delle città
cominciano ad essere controllati i venditori ambulanti, i viaggiatori,
gli accattoni e le prostitute. E poi si potenziano anche i sistemi di
informazione, cosa che permette di frenare un po' la circolazione della
malattia. Durante una celebrazione con le spoglie di San Carlo Borromeo
però, dai 35 casi sospetti di peste si è passati a 3500 morti nelle due
settimane successive. Un medico della sanità pubblica quindi afferma che
è evidente che questa aggregazione ha favorito il contagio.

Altro esempio spettacolare: la malaria. Malattia dei poveri, con fuga
costante dei nobili che potevano dalla pianura alla montagna. Di ritorno
dalla conquista delle Indie tornò la corteccia di china, usata dai
peruviani già da tempo per curare le febbri e che la moglie del vicerè
del Perù usò con grande beneficio per curare le febbri. Si pensò di
utilizzarlo per la malria, ed effettivamente sulla terzana funzionava
splendidamente. Funzionava talmente bene che iniziò un commercio molto
prospero gestito dalla Compagnia di Gesù, dai gesuiti. I gesuiti avevano
la più grande azienda farmaceutica dell'epoca e vendevano china in tutto
il mondo. L' è successo di tutto, perché per esempio a Roma e nell'Agro
Pontino funzionava perfettamente, nella pianura del ravennate e del
ferrarese no, e allora siccome non ce n'era poi tanta di corteccia di
china ci grattavano dentro tutta la china, quindi con una concentrazione
del farmaco attivo estremamente più bassa, ma ci guadagnavano lo stesso.
Un anticipo di una prima truffa di quelle clamorose. Il problema qual è?
Che la china guarisce la febbre, non c'è dubbio. Il paziente non suda,
non vomita, non ha diarrea, non saliva. Quindi nessun umore peccante
esce dl corpo, quindi non può essere una cura questa. E infatti per il
mantenimento della solidità dei riferimenti galenici nessun medico degno
di tal nome la prescriverà. In realtà poi la prescrivevano lo stesso
però non entrò nella didattica ufficiale.

La classe medica aveva medici di classe medio alta, medici con forti
basi pratiche, retrocessi dalla medicina dotta al rango di praticoni,
medici con dottrina che erano ipervalutati scientificamente perché
raccontavano le storie di Galeno, Ippocrate e Avicenna, che però erano
al top della catena sociale, erano quelli che andavano a curare il duca
di Milano. Poi c'era la parte medio bassa, di diversa ispirazione perché
c'era anche Paracelso in quell'epoca che se forse non avessero
bruciacchiato tanto avremmo una medicina probabilmente un po' diversa. E
soprattutto c'erano i praticanti. Allora la gerarchia medica fissata
dagli uffici di sanità era questa nel 1740 a Milano. Medici fisici,
chirurghi maggiori, chirurghi minori, norcini e ciurmatori, che erano i
cosiddetti ciarlatani, quelli che vendevano gli unguenti. Per esercitare
la professione bisognava essere matricolati nei collegi e pagare. E
allora l'arte cerusica va ad attaccare le cose insanabili, i bubboni, le
apostene, le cataratte ecc. Poi c'era la norcineria vera, di classe
alta, che faceva le ernie, e trattava la malattia della pietra, quindi
la calcolosi delle vie urinarie era di competenza della norcineria
maggiore. La norcineria maggiore era quella che castrava i maiali da
ingrasso, e che ha reso famose le carni della pianura padana in tutto il
mondo, ma che ha reso famosi anche tanti fanciulli che venivano
sottoposti alla castrazione per il canto. Poi c'erano i norcini minori,
norcini e barbieri, che erano quelli che eseguivano il salasso. Cavavano
sangue, ma in presenza del medico che diceva se andava bene. Tutte cose
che venivano fatte anche negli ospizi gestiti dalla chiesa, poi a un
certo punto la Chiesa disse: \emph{ecclesia aborret sanguinem}, quindi
non hanno più concesso ai monaci di specializzarsi.

Nel `600 cambia qualcosa: si incontrano questi personaggi Paolo Sarpi,
ecclesiastico tollerantissimo, consulente della serenissima, che lascia
entrare tutti gli eretici del mondo nella Repubblica di Venezia, con
grandissimo vantaggio per scienza, mercato, economia ecc.

Galileo Galilei lo conosciamo, Gerolamo Fabrizio anche lui grande
metodologo e matematico sulla scia di Galileo Galilei. Un altro
personaggio fantastico era questo, Sartorio Sartori, un istriano che
comincia dire che ha ragione Galileo e quindi se c'è lo zoticon, lo
psichico e l'animale e io mando dentro della roba e ne butto fuori
dell'altra, voglio vedere com'è il mio bilancio metabolico. Si fa fare
una bilancia, ci si mette su e per mesi mangia, beve, dorme su essa. E
nota che in realtà tra l'introduzione dei cibi e quello che ha espulso
manca sempre qualcosa: fu il primo a parlare di \emph{perspiratio
insensibilis}. Poi inventa un sacco di altre robe: il termometro. Nel
frattempo inizia anche un sistema specialistico infermieristico,
soprattutto ad opera di San Camillo che in realtà era Camillo De Lellis,
che dà tutta una serie di cose che devono fare gli infermieri, ma non
solo, nel grande ospedale di Milano a un certo punto i dirigenti hanno
detto che gli infermieri dovevano saper leggere e scrivere. Non solo,
tutti devono essere iscritti alla scuola di anatomia e sapere di cosa
parlano, quindi vedete che sotto l'egida della pubblica sanità il paese
in qualche modo cresce.

Quale eredità dal `600? Un sacco di cose bellissime, perché William
Harvey con la dinamica della circolazione fa un bel passo avanti,
Borelli con la dinamica muscolo scheletrica, Gaspare Aselli sui cani ha
trovato la circolazione chilifera, Sartorio il ricambio materiale e
altri strumenti, Marcello Malpighi ha descritto la respirazione
polmonare fin nei minimi dettagli, Lorenzo Bellini si è accorto della
filtrazione renale, Francesco Redi si fa mandare da un tessitore
olandese famoso che aveva inventato il microscopio per guardarsi le tele
e perché era un maniaco psicopatico ipocondriaco si guardava quello che
veniva fuori dai denti e studia la \emph{generatio equivoca}. Tutto dal
punto di vista fisiologico cambia da questo momento in avanti, però c'è
un muro: la cultura è cresciuta enormemente, ma l'inerzia della
traduzione di queste cose è lentissima, nel `600 non comparirà
assolutamente. E quanti erano i medici che applicavano queste cose in
giro per l'Italia? La Toscana aveva già allora una distribuzione
estremamente alta di medici: uno ogni 5000 abitanti. Lo stesso in
Sicilia, non si hanno molti dati sul Vaticano, che avendo avuto le opere
della carità era un mondo un po' a parte. Il medico gode
complessivamente di una buona reputazione dal punto di vista culturale,
ma per sua sfortuna nascono queste due figure: Mollier, che poteva
vedere i medici come un pugno negli occhi, e l'altro era Goldoni, che
fortunatamente era figlio di un medico. Mollier diceva che i medici non
avevano modificato nulla sull'approccio terapeutico. Carlo Goldoni
prende in giro i medici nelle sue commedie, però essendo figlio di
medico riconosce che qualche cosa di buono fanno. Si preparava comunque
un medico un po' di più osservante della natura e disposto a osservare.

Ramazzini è quello che inizia la medicina del lavoro; si accorge che
determinate classi di popolazione hanno sempre le stesse malattie e fa
una classificazione di 60 tipi di lavoro associati a 60 malattie. Non si
accontenta, dà anche delle indicazioni alle commissioni igieniche degli
uffici di salute per esempio per quelli che lavorano in miniera.
Quest'opera fu fondamentale, ma poi successe che la popolazione cominciò
a crescere, e ai dirigenti importava meno preservare le vite dei molti
contadini, lavoratori, in quanto ce n'erano molti. Il discorso di
Ramazzini muore. Ma Ramazzini si era accorto anche che avevano ragione
gli antichi quando dicevano che bisognava lavarsi col vino e con l'acqua
di rose, perché l'acqua nella maggior parte dei casi era un elemento che
favoriva i contagi, e quindi si comincia a pensare di organizzare gli
ambienti in modo diverso per favorire e mantenere la salute.

\end{document}
