\section{LEZIONE INTRODUTTIVA AL CORSO INTEGRATO DI IGIENE E POLITICHE DELLA SALUTE}


Questa lezione è una lezione introduttiva in cui vengono spiegate oltre
alle modalità d'esame anche la gestione complessiva della didattica.

Questo corso è costituito da lezioni frontali a cui però, per la parte
di MEDICINA DI FAMIGLIA, si possono aggiungere delle visite ad alcune
strutture, per esempio alle case della salute come accaduto l'anno
scorso. ( le CASE DELLE SALUTE sono una forma aggregativa legata alla
medicina di famiglia e alla gestione soprattutto delle patologie cronico
degenerative).

\subsection{IL PROGRAMMA}


\textbf{1)EPIDEMIOLOGIA E PROFILASSI GENERALE DELLE MALATTIE INFETTIVE}(
questa parte si da per scontata che sia stata già fatta)
\begin{itemize}

\item misure di profilassi diretti e indiretti

\item vaccinoprofilassi, chemioprofilassi
\end{itemize}
\textbf{2)LA PREVENZIONE , SORVEGLIANZA E CONTROLLO DEI RISCHI SANITARI}
\begin{itemize}

\item concetti generali di prevenzione e sanità pubblica

\item metodologia epidemiologica

\item indicatori salutari e sistemi di sorveglianza

\item emergenze sanitarie e medicina delle catastrofi

\item rischi sanitari del viaggiatore internazionale

-educazione sanitaria e promozione della salute
\end{itemize}
\textbf{3)MALATTIE PREVEDIBILI CON VACCINI E ALTRE PATOLOGIE INFETTIVE}
\begin{itemize}

\item malattie vpd dell'infanzia. Il PNPV.

\item malattie vpd dell'adolescenza, dell'adulto, dell'anziano

\item calendari e strategie vaccinali

\item  malattie trasmissibili non vpo

\item  tossinfezione alimentare e HACCP

\item infezioni correlate all'assistenza
\end{itemize}
\textbf{4)ORGANIZZAZIONE SANITARIA ECONOMIA SANITARIA E MEDICINA DI
FAMIGLIA}
\\\\
\textbf{5)DETERMINANTI DI SALUTE E MALATTIA}
\begin{itemize}

\item epidemiologia malattie ad alto impatto socioeconomico

\item fattori di rischio modificabili e non modificabili

\item medicina preventiva e medicina predittiva, screening

\item incidenti, infortuni, e patologie comportamentali
\end{itemize}
\textbf{6)AMBIENTE E SALUTE}
\begin{itemize}

\item rischi ambientali e sviluppo sostenibile

\item inquinamenti atmosferici

\item ciclo idrico integrato

\item Urbanistica e salute e healthy buildings
\end{itemize}
\subsection{INTRODUZIONE}


Ricordiamo che per fare sanità pubblica non servono solo medici ma a
seconda degli approcci, servono inoltre diverse categorie, molto spesso
``decisori'' ovvero dal ministro della salute in giù, cioè chi prende
decisioni di politica sanitaria.

In tutto il corso di igiene ci sono 2 punti fermi:

\begin{enumerate}
\def\labelenumi{\arabic{enumi})}
\item
  Le priorità di sanità pubblica
\item
  Cosa si può fare in termini di prevenzione
\end{enumerate}

  Non a caso è stata mostrata una diapositiva che riguarda la campagna
  pubblicitaria per ridurre il sodio , che ha un suo ruolo come fattore
  di rischio delle malattie cerebrovascolari. Questo è un tema
  importante e prioritario in sanità pubblica poiché le morti per
  malattie cardiovascolari sono la prima causa di morte in Italia. Se
  noi agiamo sul fattore dieta possiamo fare molto in termini di
  riduzione di incidenza di mortalità da malattie
  cerebrovascolari.Perciò in tutto il corso cercheremo di identificare
  le priorità e di identificare dove la prevenzione può arrivare.

  I medici non hanno più quella piena e completa autonomia e devono fare
  i conti con patti, piani e nuove normative. L' Italia ha un servizio
  sanitario nazionale, che affronteremo come argomento a Novembre, lo
  sentiamo sempre con lo slogan ``la libertà di cura'' cioè ciascun
  cittadino per alcune prestazioni ( le più importanti) ha le cure
  gratuite o pagando un ticket di entità bassa rispetto alla prestazione
  che riceve. Questa cartina ci dice che siccome c'è libertà di
  circolazione, ovviamente quando si tratta di un emergenza il paziente
  tenderà ad andare nell'ospedale piu vicino, ma quando l'intervento
  chirurgico è programmato, quando il ricovero per far diagnosi è
  programmabile e dilazionabile , li ci sono i movimenti per l'italia

  Le regioni segnate in rosso sono le regioni che hanno una quantità di
  pazienti che migrano in altre regioni superiori al 5\% dei ricoveri.
  La Liguria ad esempio essendo circoscritta da diverse regioni verdi (
  sono verdi le regioni che accettano molto) tende ad avere un flusso
  alto verso le altre regioni. Al centro- sud invece vediamo un
  agglomerato di regioni rosse, qui evidentemente c'è un problema grosso
  di flusso di pazienti verso le regioni verdi ad esempio Emilia
  Romagna, Veneto, Lombardia significa che le strutture non sono buone o
  ritenute tali o c'è qualcosa che non va nell'organizzazione sanitaria
  di quelle regioni. ( questo argomento verrà comunque trattato più
  avanti). Il legislatore sanitario in questo caso deve fare delle
  riflessioni profonde riguardo alle ``migrazioni regionali'' perché non
  è che se la regione Sicilia perde il20\% dei ricoveri vada a
  risparmiare, anzi! Perché il meccanismo è tale per cui il ricovero del
  siciliano è pagato dal servizio sanitario della Sicilia anche se va a
  farsi curare fuori, con un contributo economico da parte della ragione
  Sicilia alla struttura alla quale il paziente si è rivolto secondo
  precisi tariffari. Perciò laddove c'è ``la fuga'' c'è un grave
  problema di dissesto economico in queste regioni, poiché i medici
  assunti ,quelli sono e quelli rimangono, avremo posti letto vuoti ,
  costi uguali e spese maggiori perché devono anche pagare il DRG ,
  ovvero la tariffa che viene pagata per la prestazione di ricovero alle
  regioni dove queste persone si sono rivolte per curarsi. Questo è un
  flash per far capire come la sanità in realtà sia più complicata del
  previsto.

  La tabella di sinistra ( grafico a barre: in questo caso la lunghezza
  della barra corrisponde alla percentuale del PIL che viene speso nella
  sanità nei paesi Europei). Possiamo notare che l' Italia è a metà
  classifica, ma considerando che dentro a quel valore di 9.2 prcento
  del PIL (che è una cifra spaventosa , siamo attorno a 150 miliardi di
  cui 120 miliardi spesi dal pubblico e i 30 restanti dal circuito delle
  prestazioni private. Per capire quanti siano questi 150 miliardi : lo
  stato spende in media 1800 euro per ogni cittadino all'anno!). Anche
  questo è un flash , sarà un argomento trattato più avanti.

  Parlando di \textbf{METODOLOGIA EPIDEMIOLOGICA} , che è una parte di
  corso ,noi illustreremo:

\begin{itemize}
\item
  Che cos'è l'epidemiologia
\item
  Quali sono gli strumenti di lavoro
\item
  Fonti dei dati, misure , lettura e interpretazione dei dati
\item
  C'è un'aggiunta che riguarda l'epidemiologia clinica: cioè
  l'applicazione dei metodi epidemiologici a quegli studi che
  sostanzialmente validano le terapie, anche gli approcci preventivi e
  vanno a far maturare le esigenze e le evidenze scientifiche.
\end{itemize}

  Vedremo inoltre la parte delle MALATTIE INFETTIVE, e qui emerge la
  gestione dell'evento infettivo quando ci sono situazioni di epidemie
  di scala mondiale, ma leggendo i giornali oggi si parla di un caso di
  epidemia di Legionellosi a parma , che è un epidemia su piccola scala
  ma che ha acceso i riflettori su una provincia , su una patologia e
  anche qui servono interventi e prese di posizione sul fronte della
  gestione.
\\\\
  Ci dedicheremo alla questione delle vaccinazioni e cercheremo almeno
  in una lezione di non trascurare un fenomeno esistente, ovvero ci sono
  vaccini sempre più efficaci e sempre meno gente che si vaccina ( un
  po' come la questione dello screening) e ciò è un po' la frustrazione
  di chi si occupa di prevenzione, che però qualcosa può fare ovvero
  cercare di capire ciò accada.

  Vedremo poi i determinanti di malattia qui c è un grafico che mostra
  in che modo i fattori esterni possono influenzare l'incidenza e lo
  sviluppo di malattia.
\\\\
  Ha peso del 5\% la genetica, ci sono il 5\% di malattie dove noi
  possiamo fare poco in realtà, possiamo al più fare diagnosi precoce.

  Su tutto il resto ci sono fattori esterni su cui noi possiamo lavorare
  molto bene a volte. L' ambiente è considerato non solo come ambiente
  fisico ma anche sociale,in quanto i luoghi di vita e le condizioni di
  vita in generale possono avere influenza sull'andamento di alcune
  malattie. Poi ci sono gli `` health behaviors'' che sonoi
  comportamenti individuali , e i ``medical care'' che sarebbe il
  servizio sanitario in grado di far fronte ai bisogni della
  popolazione. Mettendo tutto insieme facciamo il 100\%. Possiamo
  lavorare su quasi tutti questi ambiti cercando di ridurre l'impatto
  delle malattie.
\\\\
  Il tema DELL'AMBIENTE è esemplificato da 3 fotografie che non sono
  casuali ( manca diapo) una fotografia classica che parla
  dell'inquinamento tradizionale dell'aria ( tema rilevante in ITALIA e
  soprattutto in Pianura Padana). Settimana scorsa è uscito un rapporto
  che classificava la salute degli italiani , mettendo alcuni aspetti
  positivi, uno di questi al servizio sanitario nazionale che comunque è
  universalistica e fa fronte alle principali esigenze sanitarie e
  mettendo in evidenza inoltre le debolezze nazionali principali tra cui
  ricordiamo l'inquinamento atmosferico riferito soprattutto al nord
  Italia poiché è più inquinato per motivi climatici e geografici .Anche
  questo tema verrà esplicato più avanti.
\\\\
  A questo punto il prof ha introdotto due concetti base:

\paragraph{lo stato di salute}
  

  Definizione : completo stato di benessere fisico, sociale , mentale e
  non soltanto assenza di malattia.

\paragraph{La sanità pubblica}
  

  Definizione: è l'organizzazione che mobilita risorse scientifiche,
  tecniche e professionali ed economiche per far fronte ai problemi
  sanitari delle popolazioni cercando di garantire loro il miglior stato
  di salute.

  Da ciò capiamo che la strategia della sanità pubblica non è uguale
  dappertutto, cioè se faccio sanità pubblica in Uganda non ho gli
  stessi problemi dell' Italia per cui ne consegue che l'approccio sarà
  completamente diverso. In Uganda l'epidemiologia ci dice che ci sono
  malattie infettive ad incidenza altissima, pochi anziani e tanti
  bambini, quindi dovrò commisurare tutti gli interventi a quella
  situazione demografica e sanitaria.

  Cerchiamo di vedere la principale differenza tra la sanità pubblica e
  l'attività medica clinica:

  la medicina clinica salva una vita per volta , la sanità pubblica
  dovrebbe arrivare a salvare milioni di vite (soprattutto tramite la
  PREVENZIONE . La prevenzione lavora a lungo termine si basa su
  evidenze scientifiche tende ma l operazione è in corso a lavorare sul
  consenso e sulla responsabilizzazione individuale senza quello la
  prevenzione fallisce. Ci stiamo ad esempio adoperando per uno studio
  che riguarda il vaccino per lo zoster, e sappiamo che molti non sono a
  conoscenza dell'esistenza del vaccino e ciò sarebbe perdonabile da
  parte di un cittadino anziano, la cosa grave è che nemmeno i medici
  sanno molte volte dell'esistenza del vaccino. Per cui bisogna
  informare e responsabilizzare.

  Esistono diversi tipi di prevenzione:
  \begin{itemize}

\item
  La primaria che cerca di rimuovere i rischi e proteggere gli esposti
  ai fattori di rischio ( es. tramite i vaccini)
\item
  La secondaria che fa diagnosi precoce e identifica situazioni a
  rischio
\item
  Prevenzione attiva che fa interventi sulla persona ( ex. Vaccini e
  screening)
\item
  Prevenzione collettiva
  \end{itemize}
\subsection{COME SI SVOLGERà L'ESAME DI IGIENE E SANITA'  PUBBLICA}

Innanzitutto è un esame di 12 crediti. 
Come si svolge l’esame: per tutto l’anno viene garantita la possibilità di svolgere un quiz di 36 domande a scelta multipla su tutto il programma più orale finale. Invece per la prima fase dell’anno fino a febbraio , si possono svolgere i 3 compiti a quiz , suddividendo così la materia , più sempre un orale. Il voto finale dipenderà molto dalla media dei 3 scritti, si arriverà all’orale con un certo range di voto ( ad esempio tra il 26 e il 28) e l orale quindi non modificherà più di tanto l’esito finale.
La prima prova verterà su 2 argomenti( saranno 12 domande e il primo appello sarà tra la metà e la fine di novembre, la seconda possibilità di darlo sarà a Dicembre e altre fino a fine Febbraio):
\begin{itemize}

\item epidemiologia e prevenzione delle malattie infettive
\item metodologia epidemiologica
\end{itemize}
La seconda parte comprende argomenti trattati da inizio Novembre e termina a Natale( l’appello sarà dal 15 Dicembre in poi) :
\begin{itemize}

\item economia sanitaria e organizzazione sanitaria
\item le altri parti le stiamo valutando
\end{itemize}
Tutto ciò che manca verrà messo nel terzo e ultimo compito a Gennaio/Febbraio

Superando i diversi compitini che avranno un giudizio ( sufficiente, discreto, buono, ottimo) si avrà un giudizio finale che corrisponde a un range di voto e si accede con questo poi all’orale.

Ci sono alcune parti in cui il materiale delle lezioni è sufficiente come per esempio Bioetica, o la parte di medicina generale , per altre invece va fatta  integrazione soprattutto per questi argomenti:
\begin{itemize}

\item Organizzazione sanitaria
\item Metodologia epidemiologica
\item Profilassi malattie infettive
\item Alimenti e ambiente
\end{itemize}