
\section{INFEZIONI DELL'APPARATO RESPIRATORIO}


\subsection{INFLUENZA}

\subsubsection{Ecologia del virus dell'influenza.}


Il virus di tipo A ha molti serbatoi animali, sia domestici quindi
anatre, oche, galline ma anche cavalli e suini che possono essere
infettati da altri territori in cui c'è un' ampia endemia negli animali
aviari.

Negli animali c'è una via di trasmissione sia aerea (come negli umani)
ma soprattutto fecale quindi c'è una contaminazione dell'ambiente e
delle acque e una acquisizione dell'animale dell'infezione che però non
è mortale, a parte alcuni ceppi e quindi in seguito allo spostamento
dell'animale in altre zone abbiamo una trasmissione in altri territori.

Il cerchio si chiude quando vengono infettati animali domestici e
stabiliscono un rapporto stretto con l'uomo che può quindi contrarre a
sua volta l'infezione.

Nell' habitat umano, negli animali che convivono con l'uomo, sono
presenti virus diversi con emoagglutinina diversa che sta ad indicare
che le tipologie di antigeni negli animali sono numerosissime, favorendo
la formazione di questi nuovi virus che poi possono infettare l'uomo.

Il cavallo per esempio presenta H3, H7, H9; negli aviari ci sono tutti,
nel suino ci sono H1 e H3 ma piano piano si scoprono nuove potenzialità
di antigeni.

Oggi abbiamo riconosciuto 18 tipi della spicola di Emoagglutinina, che è
l'anti-recettore virale che si aggancia alle cellule sensibili e 10 tipi
di Neuraminidasi. Gli aviari le hanno tutte, sia H (emoagglutinina) che
N (neuraminidasi) mentre gli altri ospiti ne hanno alcuni, che però si
stanno allargando, per esempio l'H5 che è tipicamente aviario ma che è
stato in grado di infettare l'uomo pur non essendone tipico.

\subsubsection{Pandemie}


L'influenza è l'ultimo grande morbo infettivo, responsabile di pandemie
mentre gli altri sono stati combattuti e sconfitti.

La grande pandemia che per noi rappresenta la massima espressione di
rischio è quella del 1918 chiamata la Spagnola (perché isolata dai
soldati Spagnoli) che presenta come antigeni l'H1 e N1.

Nel secolo scorso si sono presentate tre grandi pandemie:
\begin{itemize}
\item 
 1918 che è stata storicamente documentata ma non è stato isolato il
virus perché il primo virus è stato isolato nel 1933. Tuttavia nel 2005
è stato isolato il virus in un soldato della grande guerra congelato e
noi sappiamo che è stato un salto di specie, cioè che quel virus prima
non era tipico dell'uomo.
\item 
 1957 l'Asiatica. Cambiamento sia di H che di N infatti era una H2N2.
\item 
 1968 chiamata Hong Kong che è un H3N2.
\end{itemize}
Nel 2000 era attesa una pandemia che è arrivata invece nel 2009 (quindi
capiamo che sono ad andamento molto lento) che venne chiamata
``Messicana'' o ``suina'' a seconda che si indichi il posto in cui si è
espressa prima o che si indichi l'animale che ha trasmesso il virus.

La pandemia si ha quando abbiamo un veloce trasferimento del nuovo virus
alla popolazione umana che è vergine non essendo mai venuta a contatto
con il patogeno e non è immunizzata. Mentre nel passato ci metteva 12-14
mesi per coinvolgere tutto il pianeta, nell'ultima sono bastati 3 mesi
per determinare l'allerta pandemica, questo legato anche ai flussi
migratori di popolazione che oggi sono molto rapidi e continui.
\\\\
Il nuovo virus che ha causato l'ultima pandemia, chiamato ``H1N1
variato'' è dotato di elevatissima contagiosità ma bassa patogenicità e
la letalità globale è uguale o inferiore a quella del virus stagionale.

In Italia sono stati stimati 4000 casi, dei quali 453 gravi (cioè che
richiedono un'assistenza respiratoria) con circa la metà di questi che
sono morti, 12 dei quali nella nostra regione.

\subsubsection{Storia Naturale}


Presenta un'incubazione breve, massimo 4 giorni ma con una media di 2
giorni. successivamente si ha un esordio improvviso con una febbre
elevata \textgreater{}38.5, sintomi respiratori e sintomi articolari.
\\\\
La malattia virale è benigna nel giovane adulto ma richiede comunque
attenzione medica. Dura mediamente una settimana ma lascia una profonda
astenia per molto tempo.

Non ha dei sintomi specifici, infatti si parla di sindromi influenzali.

La sindrome, per definizione, prevede tre parametri clinici e un
indicatore epidemiologico.

Parametri clinici sono:
\begin{itemize}

\item Febbre \textgreater{} 38\textsuperscript{o} C

\item Un sintomo sistemico: cefalea, astenia, mialgie.

\item  Un sintomo respiratorio: tosse, congestione nasale o faringodinia.
\end{itemize}
Il fattore epidemiologico vuole dire la contemporaneità di comparsa, per
esempio il bambino che torna a casa da scuola e si infettano i familiari
o viceversa si ammala a casa e una volta tornato a scuola anche i
compagni di classe sviluppano la malattia.

\subsubsection{Vie di trasmissione}


Si trasmette per via inter-umana per via diretta (mediante affettuosità)
e per via semidiretta (emissione del virus come goccioline di flugge
nell'aria e assunzione a breve distanza da parte di un altro soggetto).

Una parte si può depositare sulle superfici e quindi essere trasmessa
successivamente, ma sempre in tempi molto brevi perché il virus non
resiste a lungo nell'ambiente esterno. Molto importanti sono i veicoli
come i fazzoletti.

Una volta entrato penetra nel rinofaringe, aderisce ed entra nelle
cellule epiteliali dotate di recettori come acido sialico e all'interno
di queste porta a termine la replicazione a cui consegue la necrosi
estesa dell'epitelio respiratorio, con desquamazione dello stesso ed
esposizione ad altri agenti patogeni che possono causare sovrainfezione,
spesso di origine batterica.

L'eliminazione è molto intensa anche dopo 5- 7 giorni anche se la fase
più acuta è attorno alle 24-72 ore dall'infezione. Un aspetto importante
dal punto di vista epidemiologico è che il soggetto è già infettante
anche un giorno prima dell'espressione della malattia conclamata, per
cui il bambino che va a scuola con una situazione di malessere ma senza
la febbre è già infettante e quindi da lì parte il focolaio epidemico.

\subsubsection{Complicanze}


Perchè l'infezione da virus dell'influenza può essere grave?

La gravità è legata a:
\begin{itemize}
\item 
  Patologie preesistenti: in soggetti che presentano patologie
cronico-degenerative (diabete, patologie respiratorie o cardiovascolari)
l'infezione da virus dell'influenza va ad aggravare la patologia di base
e può portare anche a morte.
\item 
  Sovrainfezioni batteriche: la disepitelizzazione della mucosa
respiratoria favorisce l'aggancio di batteri, soprattutto Streptococcus
Pneumoniae ma anche Staphilococcus Aureus o Haemophilus Influenzae.
\item 
  Abbassamento dell'infezione a livello polmonare con sviluppo di
polmoniti virali, rare ma molto gravi. Possono esserci anche
localizzazioni eccezionali come a livello dell'encefalo che causano
meningo-encefaliti.
\end{itemize}
Nel periodo epidemico c'è un eccesso di mortalità per cause respiratorie
dovute all'influenza che si misura come morti per polmoniti nel periodo
epidemico e si registra sempre questo eccesso di morti superiore agli
attesi per patologie respiratorie.
\\\\
Le classi più a rischio di morte sono bambini \textless{}5 anni e gli
anziani, ad eccezione della ``Spagnola'' in cui venivano colpiti anche
giovani tra 20 e 35 anni.

C'è da dire che la spagnola era un virus si molto grave, ma è anche
capitato in un periodo storico difficile con popolazione defedata e
malnutrizione .

\subsubsection{Terapia}

I farmaci sono di diversi tipi:
\begin{itemize}
\item 
  Farmaci sintomatici : antipiretici, antinfiammatori, antitosse,
espettoranti.
\item 
  Farmaci specifici: Amantadina, Rimantadina e Ribavirina di prima
generazione e poi dei farmaci innovativi di seconda generazione come gli
anti-neuraminidasi.
\end{itemize}
Attenzione: gli antibiotici NON vanno usati per la cura dell'influenza,
al massimo vanno usati per le sovrainfezioni batteriche.

Ricordiamo inoltre che nella popolazione giovane e sana la patologia
viene generalmente sconfitta dalle difese immunitarie del soggetto
quindi non bisogna abusare di farmaci.

\subsubsection{Prevenzione}


Il virus è molto fragile, variabile, si diffonde per via aerea, è
altamente contagioso e la contagiosità precede i sintomi. Questi sono
gli assunti su cui sono state formulate le regole per le campagne di
prevenzione. Abbiamo a disposizione diversi tipi di prevenzione:
\begin{itemize}
\item 
1. PREVENZIONE GENERALE che comprende:
\begin{itemize}
\item 
  Notifica :essenziale, perché è soggetta a sorveglianza internazionale.
\item 
  Vaccinazione.
\end{itemize}
\item 
2. PREVENZIONE INDIVIDUALE:
\begin{itemize}
\item 
  Lavaggio delle mani per evitare di scambiare materiale tramite mani
contaminate.
\item 
  Igiene respiratoria: coprirsi bocca e naso quando si tossisce o
starnutisce e lavarsi le mani successivamente.
\item 
  Isolamento volontario: se un soggetto ha dei sintomi respiratori
dovrebbe rimanere il più possibile isolato rispetto alle altre persone,
per esempio dovrebbe restare a casa senza andare al lavoro oppure a
scuola perché da un lato deve curarsi e dall'altro deve evitare di
contagiare altre persone, oppure se proprio deve stare a contatto con
persone che rischiano il contagio, dovrebbe usare la mascherina
protettiva almeno in ambiente sanitario.
\end{itemize}
\end{itemize}
\paragraph{Vaccino}


Il vaccino antinfluenzale è disponibile dagli anni '40.

Adesso abbiamo molti tipi di vaccini:
\begin{itemize}
\item 
  Vaccino inattivato come lo Split che è usato dai 3 anni in su (è un
vaccino derivato dalla disgregazione e successiva purificazione del
virus e quindi essendo molto purificato è molto poco reattogeno per cui
lo si usa nei bambini).
\item 
  Vaccino inattivato a subunità, per cui ho solo Emoagglutinina e
Neuraminidasi ed è il meno reattogeno in assoluto perché ho solo questi
due antigeni immunogeni.

Viene infatti destinato ai bambini molto piccoli (6 mesi).
\item 
  Vaccino Adiuvato (che sfrutta come adiuvante lo Squalene, che è un
intermedio della via di sintesi del colesterolo, più due Surfactanti)
che ha grande capacità immunogena sia verso il ceppo circolante che
verso ceppi ``vicini'' a quello circolante. È molto immunogeno e un po'
più reattogeno ma non eccessivamente.
\item 
  Intradermico, che ha delle caratteristiche particolari perchè la nuova
via di somministrazione prevede l'utilizzo di una dose molto ridotta di
vaccino ma nello spazio del derma troviamo una grande quantità di
cellule dendritiche che favorisce molto la formazione di anticorpi e
quindi non necessita dell'adiuvante. Si può quindi fornire anche a
soggetti anziani.
\item 
  Vaccino attenuato, che in Europa è stato accettato nel 2014 e che
dovrebbe esserci anche in Italia anche se non è molto diffuso.
\end{itemize}
Ci sono vaccini diversi a seconda che si tratti di un vaccino per la
pandemia o un vaccino per il periodo cosiddetto inter-pandemico.

Il vaccino pandemico è un vaccino monovalente perché è costruito solo
per il ceppo mutato.

Nel periodo inter-pandemico, in cui non circola solo un tipo di virus ma
ce ne sono diversi, abbiamo invece un vaccino trivalente e da un paio
d'anni abbiamo anche il quadrivalente (che protegge da quattro tipi di
virus di cui due A e due B), perchè abbiamo rilevato che nella
popolazione circolano due virus di tipo B e quindi la copertura per uno
solo lascerebbe scoperto il soggetto nei confronti dell'altro virus tipo
B.

il quadrivalente quindi protegge nei confronti di due virus di tipo A
(H3 e H1 che sono i più importanti) e due di tipo B.

Quello che conta è che si tratti di ceppi aggiornati cioè che si tratti
delle varianti osservate nella stagione precedente e che hanno più
probabilità di impiantarsi nel soggetto.

L'efficacia protettiva è tanto più alta quanto più è alta la
corrispondenza tra gli antigeni presenti nel vaccino e il virus
circolante e secondo dallo stato sanitario, infatti la risposta massima
la da il giovane sano, che potrebbe sembrare paradossale perché chi è
più grave è l'anziano però vediamo che oltre all'incidenza, il vaccino
riduce molto la gravità della malattia cioè che riduce di molto le
ospedalizzazioni e la mortalità, quindi l'anziano magari si infetta ma
si riduce di molto la gravità e la letalità della malattia.

Affinché si abbia la massima corrispondenza tra il virus circolante e
gli antigeni del vaccino ci si avvale di una sorveglianza virologica per
isolare il maggior numero di ceppi nel pianeta e confrontarne le
caratteristiche così da capire il grado di variazione e capire che
possibilità ha quel virus di penetrare all'interno della popolazione.

Quindi identifico i ceppi mutati e costruisco un vaccino il più efficace
possibile ogni anno.

Il sistema di sorveglianza riguarda 83 Paesi con laboratori nazionali e
decentrati sul territorio. Ognuno ha il suo centro di riferimento, per
l'Europa è Londra; per l'Italia è l'istituto superiore di sanità e per
la nostra regione è Parma.

Questa sorveglianza serve sia per evidenziare il virus stagionale ma
anche per evidenziare la presenza di un virus totalmente mutato che può
essere responsabile di una pandemia.

La scelta del ceppo vaccinale deriva da uno studio probabilistico: OMS
si riunisce a Febbraio di ogni anno e valuta le caratteristiche di tutti
i virus isolati in tutte le parti del mondo, valutandoli su base
genetica, epidemiologica e virologica e valutando anche l'efficacia del
vaccino contro quel ceppo di produrre immunità in quella popolazione.

In base a questo vengono scelti i ceppi e dati, a marzo, alle case
farmaceutiche che si occupano della produzione del vaccino. Esse avranno
solo 3 mesi per la produzione in larga scala e non è un tempo molto
lungo.

A Luglio devono essere consegnati i rapporti sulle caratteristiche del
vaccino ai sistemi di sicurezza affinché siano sottoposti ai controlli
biologici e affinché si abbia l'autorizzazione alla vendita
(autorizzazione data normalmente ad Agosto) per poter partire con la
produzione in massa e poter successivamente iniziare la campagna di
vaccini ad Ottobre.

Ogni anno troviamo ceppi più o meno mutati per cui bisogna cambiare il
vaccino.

Per esempio nella stagione 2008-09 è stato H1N1, nella stagione
successiva è stato cambiato solo il ceppo B mentre nel 2011 sono stati
cambiati sia il ceppo A che il ceppo B. Nell'annata 2015-16 ha cambiato
l'H3 e il B e c'è stata l'introduzione del quadrivalente.

Quest'anno il vaccino sarà costituito, per quanto riguarda il virus A,
dal virus Hong Kong H3N2 e per il vaccino contro il virus B si è
utilizzato il ceppo Brisbane del 2008.

Rimane il ceppo pandemico H1N1.

\subparagraph{Obiettivi della vaccinazione}
\begin{itemize}

\item 
1. prevenire la malattia
\item 
2. prevenire le conseguenze gravi
\item 
3. attenuare le complicanze
\item 
4. ridurre la mortalità
\item 
5. ridurre la circolazione del virus nella popolazione e quindi impedire
la formazione di ceppi mutati.
\end{itemize}
Chi vaccinare?
\begin{itemize}
\item 
  Per motivi medici :
\begin{enumerate}

\item  Tutti i soggetti con età \textgreater{}64 anni perché è la
popolazione più fragile dal momento che il loro sistema immunitario è
meno efficiente, oltre al fatto che in questa fascia di età è facile che
ci siano delle comorbidità, anche non note.

\item  Soggetti di qualsiasi età che presentino patologie
cronico-degenerative come BPCO, altre patologie respiratorie,
cardiovascolari, renali, metaboliche.

\item  Donne che nel periodo epidemico di trovano nel II o III trimestre di
gravidanza perché il virus può essere pericoloso sia per la madre che
per il feto come testimonia un aumento dei nati prematuri e degli aborti
in periodo epidemico.

\item  Bambini sottoposti a trattamento a lungo termine con Acido
Acetilsalicilico

\item  Soggetti istituzionalizzati, che sono spesso anziani ma possono
essere anche altre categorie di persone in cui la comparsa della
malattia è più facili.\end{enumerate}
\item 
  Per motivi Sociali:
\begin{enumerate}

\item Tutti quelli che sono addetti a lavori di pubblica utilità dovrebbero
venire vaccinati.

\item Soggetti assistenti di anziani e bambini

\item Soggetti che lavorano con animali per evitare le commistioni con
virus animali.\end{enumerate}
\end{itemize}
\subparagraph{Eventi avversi}


Sono rari. Soprattutto locali, specialmente se usiamo un vaccino
adiuvato.

Qualunque vaccinazione ha delle controindicazioni.

Per il vaccino ucciso sono poche: allergie potenziali a proteine
dell'uovo perchè ancora oggi il vaccino viene istituito in uova
embrionate di pollo, quindi tracce di proteine del materiale d'origine
sono presenti.

Altre controindicazioni sono lo stato febbrile, la malattia grave in
atto (come in tutti i vaccini).

\subparagraph{Polemiche sul vaccino}


Spesso si sente parlare in ambiti non medici di come il vaccino
antinfluenzale sia inefficace.

I limiti ci sono, ma dobbiamo anche dare atto alla sua efficacia.

Perchè si dice così? In una stagione influenzale, cioè da Ottobre ad
Aprile, noi incorriamo in più episodi respiratori, che siano raffreddori
o faringiti o malessere; ma i virus in grado di provocare sindromi
respiratori o quelle cosiddette sindromi influenzali sono tantissimi
come il VRS, adenovirus, metapneumovirus, bocavirus, parainfluenzali.
Tutti si possono esprimere in modo simile all'espressione del virus
dell'influenza. Inoltre abbiamo anche molti batteri che possono dare
forme respiratorie febbrili che possono essere confuse se si presentano
in questa stagione.

Inoltre è necessario ricordare che il vaccino contro il virus
influenzale protegge solo nei confronti di questo virus, non protegge da
tutti gli altri!
\\\\
È possibilissimo che un soggetto vaccinato presenti una sindrome
influenzale, che verosimilmente è stata causata da un altro virus o
agente patogeno. Inoltre un soggetto può anche essere infettato da un
virus dell'influenza, semplicemente perché il vaccino non copre al 100\%
nei confronti del patogeno, c'è sempre una (seppur ridotta) possibilità
di non avere una copertura immunitaria.

Non è un vaccino perfetto, ma ricordiamo che ha una grande importanza
nel ridurre le complicazioni gravi e la mortalità, soprattutto nei
pazienti anziani.

\paragraph{ Chemioprofilassi}


Abbiamo dei farmaci antivirali molto efficaci che sono l'Oseltamivir e
Zanamivir che sono inibitori della Neuraminidasi, hanno un meccanismo
d'azione simile, sono attivi su A e B e l'efficacia deriva dalla
tempestività con cui vengono utilizzati.

La Neuraminidasi è l'enzima con cui il virus si sgancia dalla cellula
infettata e ne esce. Quando si libera, se trova delle altre cellule che
possono riceverlo, attacca anche loro e perciò è il fattore che causa
poi problemi perchè il virus si sparge lungo l'albero respiratorio.
Impedire la sua uscita vuol dire confinare il virus all'interno delle
cellule, però vuol dire anche dover partire molto presto perché se
partiamo quando la diffusione è già ampia allora non potrò fare molto.

Viene usato sia per profilassi (bambini o anziani a rischio di sviluppo
della malattia in seguito ad esposizione) sia a scopo terapeutico, anche
in questo caso fondamentale è la tempestività di azione.

\subsection{TUBERCOLOSI}


Definita in epoca attuale una malattia ri-emergente, la Tubercolosi
(TBC) è considerata una delle tre malattie infettive che causano il
maggior numero di morti a livello mondiale. Le altre due sono la malaria
e AIDS.

\subsubsection{Contesto attuale}


Viene definita \textbf{ri-emergente} perché:
\begin{enumerate}

\item c'è stato un cambio della società che ha visto l'aumento
dell'infezione da HIV che causa una condizione di immunodepressione che
favorisce decisamente l'infezione da TBC e favorisce anche la
transizione da semplice infezione tubercolare a malattia conclamata.

\item cambiamento della società per la forte migrazione di soggetti che
derivano da zone endemiche per la TBC. Abbiamo, a causa di questi flussi
migratori, la possibilità di introdurre sia soggetti infetti sia
soggetti più fragili e suscettibili a sviluppare la malattia.

Queste prime due condizioni insieme hanno favorito la formazione di
sacche di povertà e di emarginazione sociale nella nostra popolazione e
quindi in condizioni favorevoli per l'impianto di questo microorganismo.

\item  per la comparsa di ceppi che sono multi-farmacoresistenti oppure
semplicemente farmacoresistenti.

\item limitatamente all'Italia, ha anche contribuito anche la trascuratezza
del problema della tubercolosi; cioè la malattia era stata affrontata in
modo eccellete nel secolo scorso con anche il controllo delle infezioni
in ambito veterinario, invece più recentemente diverse misure di
protezione sono venute a mancare in quanto sono calate notevolmente le
infezioni, la malattia si è quasi annullata grazie a farmaci e diagnosi
tempestiva e quindi si è sottovalutato un problema che è poi riemerso.
\end{enumerate}
Inoltre a causa dell'iperspecializzazione, la malattia tubercolare che
prima era vista da un'unica figura adesso vede l'intervento di tante
figure che spesso sono poco coordinate tra loro.

Bisogna comunque pensare che la TBC ha una morbosità che è stata si
ridotta, ma non annullata. Ricordiamo che c'era una legge che imponeva a
tutti gli studenti di medicina di essere vaccinati e ora non c'è più,
quindi forse si è sottovalutato il problema e la malattia ha ripreso in
parte piede.

\subsubsection{Epidemiologia}


Nel 2016, l'ultimo report dice che ci sono 10.000.000 di infetti/anno in
tutto il pianeta con 1.800.000 di morti per tubercolosi, 2.000 sembra
che siano bambini sotto i 5 anni e inoltre si contano 400.000 morti per
malattia tubercolare in soggetti HIV-positivi.

Sembra che 1/3 della popolazione mondiale sia infetta e che il 5-10 \%
di questi possa sviluppare la malattia conclamata nell'arco della vita e
quindi capiamo che non tutti gli infetti si ammalano, ma è una
percentuale consistente. 3/4 dei casi di infezione sono concentrati nei
Paesi in via di sviluppo.

Nel 1993 OMS l'ha dichiarata emergenza globale e ha stabilito un piano
globale per l'abbattimento della TBC che ha coinvolto circa 400.000
gruppi e organizzazioni, con l'obiettivo -> stop alla tubercolosi, cioè
di renderla allo stato di sporadicità ovvero di un caso per milione di
abitanti.

\subsubsection{Il micobatterio tubercocolare}


È un parassita stretto dell'uomo ma può colpire anche gli animali.

È fortemente aerobio, infatti si posiziona a livello alveolare polmonare
ed ha sempre bisogno di ossigeno per replicare. Una sua importante
caratteristica è che ha un tempo di replicazione lunghissimo: quando un
batterio qualsiasi impiega in media 20 minuti, il micobatterio
tubercolare ci mette anche 20 ore e questo è un limite in termini
diagnostici perché per giudicare un analisi colturale come negativa sono
necessarie 12 settimane.

È in grado di poter sopravvivere e replicare all'interno dei macrofagi
inibendo la propria distruzione (mediante il fattore cordale che
impedisce la fusione del fagosoma con le vescicole lisosomiali per la
formazione del fagolisosoma).

Altra caratteristica è che soffre molto alla presenza di raggi UV e al
calore ma resiste molto ai disinfettanti. Solo alcuni disinfettanti
abbastanza potenti riescono a sconfiggere il micobatterio.

\subsubsection{Patogenesi dell'infezione}


A seguito della penetrazione inizia l'infezione tubercolare che dal
punto di vista diagnostico è documentato dal test tubercolinico. Nella
stragrande maggioranza dei casi (90\%) l'infezione resta asintomatica ed
è quella che viene chiamata tubercolosi latente.

Nel restante 10\% dei casi si sviluppa una TBC attiva nel primi 5 anni
in un 5\% e nel restante 5\% si sviluppa nel resto della vita.

Il suo potere patogeno si sviluppa quando arriva a livello alveolare e
viene fagocitato dai macrofagi all'interno dei quali non solo non viene
inattivato ma riesce anche a replicare. Il destino in termini di danno è
legato alla capacità replicativa che dipende sia dal batterio che dal
soggetto e dalle sue capacità di difesa immunitaria. Se viene bloccata
la sua replicazione allora la malattia si manifesta come infezione
latente, cioè si sviluppa la lesione granulomatosa e vediamo il
cosiddetto ``viraggio tubercolinico'' cioè il test della tubercolina
diventa positivo, anche se il paziente non ha né sintomatologia, né
tanto meno la liberazione del patogeno.

Se invece la moltiplicazione aumenta e diventa attiva, il processo
evolve e va verso una lesione essudativa che porta a una caseificazione,
con distruzione dell'essudato, morte di parte dei batteri e che può
concludersi con la calcificazione, seguita da morte ancora più
significativa dei batteri (anche se qualcuno può sopravvivere).

Invece se si verifica il rammollimento della materia caseosa diventa
malattia tubercolare in quanto viene consentita una attiva replicazione
del micobatterio e un'eliminazione tramite i bronchi delle particelle
caseose con i micobatteri.

A livello polmonare andremo a trovare quelle che sono definite caverne
tubercolare e che ci permettono di fare diagnosi di malattia
tubercolare.

Solo in questo momento il paziente è infettante, mentre negli altri
momenti l'infezione è pericolosa solo per il paziente e non per gli
altri.
\\\\
Possiamo trovarci davanti a un'infezione latente cioè il batterio ha
infettato l'ospite, abbiamo il test tubercolinico positivo ma in termini
di patogenesi e di clinica non abbiamo niente, cioè abbiamo una
condizione di equilibrio tra la replicazione del micobatterio e l'azione
del sistema immunitario.
\\\\
Questa condizione di latenza può rimanere tale oppure può dare luogo
alla malattia tubercolare che può essere definita come primaria oppure
post-primaria:
\begin{itemize}
\item 
  Primaria quando è la prima infezione che determina il cammino fino
alla malattia conclamata e quindi si verifica nei primi mesi o anni
post-infezione.
\item 
  Post-primaria quando si sviluppa la malattia attiva in soggetti
infetti che hanno un'infezione latente che si riattiva nel corso del
tempo oppure che sviluppano una nuova infezione.
\end{itemize}
\subsubsection{Diagnosi}


I materiali di scelta da analizzare sono l'espettorato ma anche il
broncoaspirato, raccolti con tre campioni successivi ma possiamo cercare
anche altri campioni a seconda dell'organo colpito, per cui potremo
anche ricercarlo nel sangue o nelle urine se viene ad essere colpito il
rene.

L'esame microscopico è l'esame cardine, si può fare diagnosi
sull'espettorato osservato al microscopio con la colorazione
differenziale di Ziehl-Neelsen che sfrutta l'alcol acido resistenza del
batterio. Dal momento che non esistono micobatteri come popolazione
residente nel cavo orale, il riscontro di micobatteri nel campione vuol
dire che c'è l'infezione. In passato si faceva la coltivazione che ora
si fa molto meno in favore della diagnosi rapida che sfrutta la biologia
molecolare che ha come vantaggio la rapidità di esecuzione.

Un elemento chiave della diagnosi è la farmacoresistenza, sia mono che
pluri-farmacoresistenza che è sempre più diffusa. La terapia poi prevede
il trattamento adeguato in base alle resistenze del ceppo.

L'immunità nella TBC è fortemente correlata al concetto di allergia,
infatti anche il test tubercolinico (intradermo-reazione di Mantoux) si
basa su una reazione che abbiamo nei confronti di un antigene che
conosciamo perché siamo stati precedentemente esposti al micobatterio.
Immunità ed allergia vanno di pari passo. Gli sperimenti di Koch
mostravano che inoculando dei batteri in un soggetto sano, inizialmente
non accade niente (per 15 giorni circa) e successivamente si sviluppa
una lesione ulcerativa, mentre se è ammalato ho la lesione necrotica
entro le 72 ore che poi regredisce senza influenzare la malattia.
Infatti questo fenomeno viene chiamato \textbf{ipersensibilità
ritardata}, definita tale perché avviene dopo le 24 h.

\paragraph{Test tubercolinico (o PTD)}


È un metodo dosato e quantificato di somministrazione di una quantità di
un derivato purificato di una proteina tubercolare che viene iniettato
in sede volare dell'avambraccio, inoculato in sede intradermica e deve
essere osservato dopo le 24h ed entro le 72h perchè stiamo parlando di
una sensibilità ritardata. Se il test risulta positivo si mostra con un
indurimento del sottocute legato ad una reazione dell'immunità
cellulo-mediata e che viene misurato in termini di dimensioni della
lesione. La positività denuncia una pregressa esposizione al
micobatterio tubercolare.

È un buon test, ha pochi falsi positivi e negativi:
\begin{itemize}

\item I falsi negativi possono essere legati o ad un test eseguito male o
legato a patologie sottese che possono modificare la risposta, per
esempio possono indurre un'anergia delle cellule immunitarie.

\item  I falsi positivi possono essere legati alla presenza di antigeni
comuni che possono essere presenti in micobatteri atipici; poi anche la
vaccinazione fa risultare il test positivo.
\end{itemize}
\paragraph{Quantiferon test}


E' un nuovo test che viene fatto in vitro e che sfrutta delle proteine
secretorie contenute in determinate parti del genoma del micobatterio.
Si basano sula capacità dei linfociti T effettori circolanti di produrre
interferon-gamma in relazione a determinati stimoli.

Se un soggetto ha già incontrato il micobatterio tubercolare e viene a
contatto con dei suoi antigeni (proteina ESAT-6 e CFP-10), viene
stimolata una grande produzione di interferon-gamma da parte dei suoi
linfociti, mentre se non l'ha mai incontrato avrò una produzione bassa
di inf-gamma.

Vantaggi di questo test:
\begin{enumerate}

\item il test viene fatto in un'unica seduta e il risultato viene rilevato
in modo oggettivo, mentre per la Mantoux abbiamo bisogno di due sedute e
il risultato viene determinato soggettivamente dall'operatore.
\end{enumerate}
Svantaggi:
\begin{enumerate}

\item  abbiamo bisogno di sangue e quindi devo fare un prelievo di sangue al
paziente.

\item l'esame deve essere fatto tempestivamente, cioè deve trascorrere un
tempo breve tra il prelievo e lo svolgimento dell'esame.
\end{enumerate}
La Mantoux rimane comunque il gold standard.

\subsubsection{Epidemiologia e distribuzione dell'infezione}


E' una malattia diffusa su tutto il pianeta, particolarmente presente in
zone ad alta endemia dalle quali può essere esportata.

Tende sempre di più alla farmaco-resistenza (MDR identifica la
resistenza ai due farmaci di prima scelta Isoniazide e Rifampicina, XDR,
NDR, XXDR sono un allargamento della resistenza ai prodotti disponibili
come seconda scelta).

Un altro problema risulta essere l'aumento dell'infezione da HIV che
risulta essere importante perché favorisce impianto ed espressione della
TBC.

Abbiamo i dati del 2012 perché essendo stato l'anno della TBC la
raccolta di dati è stata massima ma sappiamo che ad oggi la situazione
non è variata di molto.

Possiamo suddividere il pianeta in tre aree in base alla morbosità:
elevata, media, bassa.

Principalmente ci preoccupiamo delle zone in cui la malattia è endemica
e che hanno una morbosità elevata. La malattia è più presente nei
giovani in Africa, Asia, Sud America e quindi in una parola sola
possiamo dire nei \textbf{paesi in via di sviluppo.}

Parlando della prevalenza della \textbf{farmaco-resistenza} nei nuovi
casi di TBC quindi la situazione attuale, vediamo che è particolarmente
importante in Europa dell'est quindi vicino al nostro paese.

Per quanto riguarda i paesi di origine troviamo che una parte molto
importante riguarda Africa, Asia (con particolare riferimento a India e
Cina) e poi Europa dell'est.

La mortalità va di pari passo con la prevalenza, anche se questo non è
un dato sempre vero.
\\\\
La farmaco resistenza non è causata da una sola mutazione ma da una
sommatoria di mutazioni. La diffusione di ceppi resistenti compromette
fortemente la terapia ma anche l'efficacia de programmi di controllo
perché la liberazione nell'ambiente del batterio e quindi la
contagiosità dei soggetti infettati da un micobatterio
farmaco-resistente è molto più lunga rispetto alla norma, questo perché
è più difficilmente curabile e quindi le due cose sono strettamente
connesse tra loro.

Sono stati registrati dei fenomeni cumulativi di infezioni tubercolari
sia in Europa che in USA e in particolare è molto rilevante la
co-infezione in soggetti HIV-positivi in cui si associa un aumento della
mortalità.

Non ultimo dobbiamo dire che la cura del malato di tubercolosi aumenta
notevolmente le spese sanitarie.

Il legame tra TBC e HIV è dovuto al fatto che HIV è in grado di
riattivare l'infezione latente. Inoltre il rischio di avere una malattia
attiva subito dopo l'infezione tubercolare è molto più elevata
nell'HIV-positivo e poi favorisce anche le localizzazioni
extra-polmonari.

Anche in questo caso i nuovi casi di TBC in soggetti HIV-positivi si
concentra soprattutto in Africa e nel sud-est asiatico.

\subsubsection{Traguardi imposti}


Nel 2003 ci si è posti (dal momento che in passato una delle più grandi
difficoltà era la diagnosi precoce) di realizzare la diagnosi almeno nel
70\% della popolazione e almeno l'85\% di questo doveva essere curato
(obiettivo entro il 2005).

Entro il 2010 volevano realizzare una riduzione del 50\% di prevalenza e
mortalità rispetto al 2000 e nel 2050 arrivare alla sporadicità, quindi
\textless{}1 caso/1.000.000 di persone.

Nel 2016 è stato fatto un nuovo insieme di obiettivi da raggiungere: ci
si pone la riduzione del 90\% delle morti entro il 2030 e la riduzione
dell'80\% dei casi entro il 2030.

\paragraph{Situazione in Italia}

La situazione in Italia è relativamente buona, l'incidenza si è poco
modificata negli ultimi 20 anni e nella popolazione generale l'incidenza
è abbastanza bassa mentre molti casi si concentrano nelle cosiddette
categorie ad alto rischio e in particolare in alcune fasce di età.

Anche in Italia stanno emergendo i ceppi con multi-farmacoresistenza.

C'è comunque una tendenza al contenimento della malattia, nonostante
abbiamo sempre non meno di 3.000 casi/anno ma con un'incidenza che piano
piano tende a scendere.

Oltre il 70\% di questi casi si concentrano in 5 regioni di cui una è
l'Emilia Romagna. La popolazione immigrata ha un rischio 15 volte
maggiore rispetto alla popolazione italiana, sempre a causa delle
condizioni (affollamento, malnutrizione) in cui vivono questi soggetti,
che rende più facile contrarre l'infezione.

La mortalità della tubercolosi è decisamente scesa grazie all'utilizzo
di farmaci specifici, mentre la morbosità si era in parte ridotta per
poi tornare ad aumentare, fenomeno che abbiamo già chiamato ri-emergenza
della malattia.

In passato erano più colpiti gli anziani mentre adesso abbiamo avuto una
riduzione negli anziani mentre un lieve incremento nei giovani adulti.

Negli ultimi tempi abbiamo assistito ad un aumento della partenza
dall'est-europa mentre abbiamo avuto un calo della provenienza della
malattia dall'Africa.

\subsubsection{Trasmissione}


La sorgente di infezione è legata solo al malato che è contagioso solo
quando ha la malattia attiva con caverne tubercolari aperte, non esiste
il portatore sano.

La trasmissione è prevalentemente aerea con lo scambio di goccioline di
flugge contenenti il patogeno. Però possiamo anche contaminare
l'ambiente con il bio-aerosol e il soggetto può infettarsi stando in
quell'ambiente, quindi questa si tratta di una via semidiretta.

A Roma 2 anni fa c'è stato un episodio di un'infermiera che ha
contagiato un numero altissimo di pazienti perché era eliminatrice del
batterio ma ciò non era stato accertato nè sospettato, ma non è che
fosse portatore sano, è che era un malato misconosciuto! Il paziente con
TBC all'inizio è misconosciuto perché si presenta con febbricola, tosse
leggera, leggera astenia. Se poi aggiungiamo che questo può verificarsi
nel periodo di massima espressione del virus dell'influenza può
tranquillamente essere scambiata per una banale influenza e quindi la
diagnosi ritarda.
\\\\
Si stima che un malato non trattato sia in grado di contagiare 15
persone all'anno, che non è un numero altissimo ma neanche basso.

Il rischio di trasmissione si basa su tre condizioni:
\begin{enumerate}

\item  caratteristiche di contagiosità del patogeno

\item l'ambiente

\item  tipo di contatto (frequenza e vicinanza del contatto)
\end{enumerate}
La massima contagiosità si verifica quando all'espettorato si evidenza
la presenza di micobatteri perché se riusciamo a vederlo
all'osservazione diretta, ha una carica batterica molto elevata.

Una contagiosità minore si evidenzia se non trovo il micobatterio
all'esame microscopico diretto ma lo trovo in coltura.
\\\\
Dopo l'inizio del trattamento farmacologico il paziente resta contagioso
per due settimane, a meno che non si tratti di un ceppo
farmaco-resistente perché abbiamo già visto che il soggetto lo elimina
per più tempo.

L'eliminazione viene aumentata da tosse, da laringite tubercolare e
ovviamente da caverna tubercolare.
\\\\
Tutte le condizioni che favoriscono la concentrazione del micobatterio
nell'ambiente sono condizioni che favoriscono il contagio quindi piccole
dimensioni del locale, assenza di aerazione o malfunzionamento degli
impianti di aerazione, assenza di illuminazione naturale (il
micobatterio viene neutralizzato dai raggi UV).
\\\\
Per quanto riguarda la collettività, più stiamo chiusi in ambienti, più
stiamo a contatto con il batterio eliminato e più siamo a rischio di
contrarre l'infezione e infatti qui capiamo perché soggetti come
rifugiati politici o immigrati trovino negli ambienti in cui stanno
(spesso sovraffollati e chiusi) un rischio molto alto di contrarre
l'infezione.

Il tipo di contatto è tanto più rischioso quanto più lunga è la durata e
cambia in base ai diversi tipi di contatto (livello di intimità tra le
persone). I conviventi sono molto a rischio, così come soggetti che
lavorano in ufficio insieme per molte ore al giorno.

I contatti occasionali sono meno a rischio come per esempio contatti nei
luoghi ricreativi, palestra, club, ambienti comuni.
\\\\
Contatti stretti, stranieri, soggetti senza fissa dimora, operatori
sanitari, bambini, malati cronici, soggetti con AIDS, soggetti
sottoposti a terapia immunosoppressiva, insufficienza renale cronica,
diabete, tossicodipendenti, drastiche perdite di peso, trapiantati sono
le categorie maggiormente a rischio di contrarre l'infezione
tubercolare.

Per costruire una scala che poi ci serve per lo screening e la
prevenzione si distribuisce la popolazione in quattro gruppi a rischio
decrescente:
\begin{enumerate}

\item soggetti provenienti da paesi ad alta endemia, soggetti esposti al
rischio professionale.

\item  senza fissa dimora, detenuti, tossicodipendenti.

\item con patologie concomitanti che favoriscono l'impianto.

\item  soggetti anziani ospiti di case di riposo o reparti di lungodegenza.
\end{enumerate}
\subsubsection{Profilassi}


La profilassi di malattia tubercolare prevede l'isolamento. La
disinfezione mirata è molto importante perché è un batterio resistente a
molti disinfettanti. Sia la disinfezione continua, cioè mentre il malato
si trova ricoverato, sia la terminale cioè il recupero della stanza e
degli oggetti dopo la dimissione del malato perché il batterio può
restare nell'ambiente e andare a contagiare altre persone.

Per la disinfezione usiamo mezzi chimici e fisici e usiamo sia aspetti
generali come la ventilazione e la luce solare, sia aspetti mirati come
incenerimento, che oggi è stato soppiantato dall'autoclave.

Abbiamo composti chimici come composti fenolici e glutaraldeide ma anche
il cloro, a seconda del materiale che vogliamo disinfettare.

La profilassi segue l'inchiesta epidemiologica: se c'è un caso bisogna
visitare i conviventi le altre persone con cui il paziente ha contatti e
verificare il loro stato di salute -> identificazione della fonte di
contagio e protezione di coloro che vi stanno accanto.

\subsubsection{Prevenzione e Terapia}


Per quanto riguarda la prevenzione specifica disponiamo di
un'immuno-profilassi, cioè il vaccino e una chemio-profilassi, cioè
farmaci a scopo preventivo.

\paragraph{Vaccino}


E' ancora il vaccino di Calmette-Guerin che è derivato dal virus
attenuato, che viene somministrato solo a soggetti con test
tubercolinico negativo. Deve passare almeno un mese da altre
vaccinazioni, come con tutti i vaccini attenuati.

La dimostrazione che il soggetto ha risposto al vaccino è rappresentata
dal cosiddetto viraggio della tubercolina, cioè la Mantoux diventa
positiva. Il vaccino garantisce un'immunità molto lunga, mediamente una
decina d'anni.

Posso avere complicanze locali come nodulo arrossato che può
sovrainfettarsi, così come posso avere una linfoadenopatia suppurativa
locale o un ascesso sottocutaneo.
\\\\
Controindicazioni: sono differenti a livello locale e a livello globale:
a livello globale non si vaccinano i soggetti con infezione sintomatica
da HIV. In Italia non si vaccina neanche se c'è il sospetto di
un'infezione da HIV.

La positività alla Mantoux è una controindicazione alla vaccinazione
così come anche le patologie croniche e lo stato di gravidanza.

Obbligo di vaccinazione:
\begin{enumerate}

\item neonati e bambini con età \textless{}5 anni con test tubercolinico
negativo che hanno rapporti stretti con persone con tubercolosi
contagiosa.

\item  personale sanitario, studenti di medicina, infermieri che operi in
ambiente ad alto rischio di infezione da ceppi multi-farmacoresistenti o
che operi in ambiente ad alto rischio di contagio anche da ceppi non
farmaco-resistente ma che abbia delle controindicazioni al trattamento
terapeutico, cioè se un soggetto non può essere sottoposto al
trattamento terapeutico e opera in ambiente ad alto rischio, lo
sottopongo al vaccino.
\end{enumerate}
Ci sono molti studi di vaccini in corso, non solo vivi ma anche
ricombinanti e inattivati.

\paragraph{Chemioprofilassi}


Possiamo usare dei farmaci come prevenzione primaria, cioè per soggetti
sani, soprattutto per bambini a rischio di incontro del batterio che
hanno per esempio un membro della famiglia con tubercolosi attiva. È una
pratica di emergenza.

Possiamo anche usarli nell'ottica di una prevenzione secondaria: vuole
evitare che dall'infezione si passi alla malattia conclamata.

Può essere fatta a bambini \textless{}5 anni, soggetti che vivono in
comunità, soggetti che hanno recentemente documentato un viraggio del
test tubercolinico che è diventato positivo e soggetti che vivono in
ospedali psichiatrici per un periodo abbastanza lungo.

La strategia per il controllo della TBC è lo schema DOTS cioè un
trattamento controllato e verificato ma per un tempo breve, cioè una
pronta individuazione dei casi di tubercolosi contagiosa, poi il regime
terapeutico per 6-8 mesi con trattamento antibiotico direttamente
osservato per almeno i due mesi iniziali perché uno dei grossi problemi
è proprio la non assunzione dei farmaci, soprattutto nei paesi in via di
sviluppo.
